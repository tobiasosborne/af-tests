%% Section 6: Open Problems and the r -> infinity Limit
%% Corresponds to AF Nodes 1.4.4 and 1.4.4.1.  Epistemic status: OPEN.

\section{Open Problems and the $r \to \infty$ Limit}
\label{sec:open}
\afnode{1.4.4}
\openstatus

The preceding sections have established a coherent picture of the $q$-deformed
BLM model across three geometric regimes, with epistemic status ranging from
\emph{established} (Part~I) through \emph{conjectural} (Part~II) to
\emph{mixed} (Part~III).  In this section we collect the principal open problems
that have emerged from the adversarial verification process, and we analyze
the $r \to \infty$ limit in which the root-of-unity model is expected to
recover the original BLM construction of Part~I.

Throughout this section, we work at a root of unity $q = e^{2\pi i/r}$ with
integer $r \geq 5$, and the spin parameter $j$ is an odd integer satisfying
the operative admissibility constraint
\begin{equation}\label{eq:admissibility-recap}
  j \;\leq\; \frac{r-2}{3}\,,
  \qquad\text{equivalently}\qquad
  r \;\geq\; 3j + 2\,.
\end{equation}
Recall from \cref{sec:root-of-unity} that this bound is dictated by the Racah
formula for the $(j,j,j)$ quantum $3j$ symbol: the triangle coefficient
$\Delta(j,j,j)$ involves $\qfact{3j+1}$ in the denominator, and the
condition $r > 3j+1$ ensures that no factor $\qint{n}$ with $1 \leq n \leq 3j+1$
vanishes (since $\qint{n} = 0$ if and only if $r \mid n$).  This Racah bound
is strictly stronger than the representation-admissibility condition
$j \leq (r-2)/2$ (which merely ensures $\qint{2j+1} \neq 0$) and coincides
exactly with the Turaev--Viro level truncation $3j \leq r-2$ for the coloring
$(j,j,j)$.  Concretely: $j = 1$ requires $r \geq 5$; $j = 3$ requires $r \geq 11$;
$j = 5$ requires $r \geq 17$.

As established in \cref{sec:root-of-unity}, the $\mathcal{N} = 2$ SUSY
structure---$Q_q^2 = 0$ and $H_q = \{Q_q, Q_q^\dagger\} \geq 0$---is
preserved at root of unity for all admissible~$j$.  The positivity $H_q \geq 0$
is a tautological consequence of the definition of the Hilbert space adjoint
and holds for \emph{any} operator $Q_q$ on a Hilbert space, regardless of
whether its matrix elements are real or complex.

\medskip

We now state the three open problems and the $r \to \infty$ analysis.

%% ==================================================================
\subsection{Open Problem 1: The BLM-to-TV map}\label{sec:OP1}

\begin{openproblem}[BLM-to-TV correspondence]\label{op:blm-tv}
\afnode{1.4.4.1}
Is there a precise mathematical relationship between the BLM Feynman diagram
expansion at root-of-unity $q$ and the Turaev--Viro topological
invariants~\cite{TV92}?
\end{openproblem}

The BLM model at $q = e^{2\pi i/r}$ and the Turaev--Viro state sum share the
same algebraic building blocks: both are constructed from the quantum $3j$ and
$6j$ symbols of $U_q(\mathfrak{su}(2))$ at root of unity.  However, the two
objects are structurally very different.  An individual BLM Feynman diagram at
fixed spin~$j$ involves a product of quantum $3j$ symbols contracted according
to the diagram topology, yielding a quantum $3nj$ symbol.  The Turaev--Viro
invariant $\TV_r(M)$ of a closed $3$-manifold~$M$, by contrast, is a sum
over \emph{all} admissible colorings of \emph{all} edges in a triangulation
of~$M$, weighted by products of quantum dimensions and quantum $6j$ symbols
over \emph{all} tetrahedra~\cite{TV92}.

A potential approach to bridging this gap is to embed the BLM model into a
Boulatov-type group field theory (GFT)~\cite{Boulatov, Freidel} by promoting
the fixed spin~$j$ to a dynamical variable summed over all admissible spins
$l \leq (r-2)/2$.  This would require three steps, none of which has been
carried out:
\begin{enumerate}[label=(\alph*)]
  \item \textbf{Multi-spin generalization.}
    Define a multi-spin BLM model in which different fermion species carry
    different spin labels.  A fundamental obstacle is that BLM uses
    \emph{fermions}, while the standard Boulatov model uses a
    \emph{bosonic} field on $\mathrm{SU}(2)^{\times 3}$.  A fermionic GFT
    extension would be needed.  The existing literature includes fermionic
    tensor field theories (Ben Geloun--Bonzom~\cite{BenGelounBonzom} for
    radiative corrections in colored bosonic tensor models,
    Ben Geloun--Rivasseau~\cite{BenGelounRivasseau} for renormalizable
    fermionic tensor field theory), but a fermionic GFT specifically
    adapted for BLM embedding---with the correct vertex structure matching
    quantum $3j$ symbols at fixed spin---has not been constructed.

  \item \textbf{Feynman expansion $=$ TV state sum.}
    Show that the resulting GFT Feynman expansion reproduces the Turaev--Viro
    state sum~\cite{TV92} (in the same sense that the Boulatov GFT generates
    the Ponzano--Regge partition function~\cite{Boulatov}).

  \item \textbf{Single-spin sector.}
    Understand what the original single-spin BLM model computes as a
    sector of the full GFT.
\end{enumerate}

A further obstacle concerns triangulation independence.  The topological
invariance of the TV state sum relies on the completeness of the sum over
all admissible representations.  Restricting the admissible spins in the
GFT sum---for instance, to the odd integers $j \geq 1$ used by BLM---would
generically \emph{break} triangulation independence, since the Pachner move
identities that guarantee topological invariance require contributions from
all spins in the admissible set.


%% ==================================================================
\subsection{Open Problem 2: Root-of-unity spectral behavior}\label{sec:OP2}

\begin{openproblem}[Spectral and boundary behavior at root of unity]
\label{op:spectral}
What is the detailed spectral structure of $H_q = \{Q_q, Q_q^\dagger\}$ at
$q = e^{2\pi i/r}$, particularly near the admissibility boundary and as a
function of~$r$?
\end{openproblem}

Although the SUSY algebra ($Q_q^2 = 0$, $H_q \geq 0$) is preserved at root
of unity for admissible~$j$, this does \emph{not} mean the root-of-unity
model is trivially identical to the real-$q$ model.  We identify four
genuine open problems.

\paragraph{(a) Boundary behavior near the Racah bound $j = (r-2)/3$.}
At the Racah boundary $j = (r-2)/3$, the quantum dimension is
\begin{equation}\label{eq:qdim-racah-boundary}
  \qint{2j+1} \;=\; \qint{(2r-1)/3}
  \;=\; \frac{\sin\bigl(\frac{(2r-1)\pi}{3r}\bigr)}{\sin(\pi/r)}\,,
\end{equation}
which is nonzero (and generically not equal to~$1$) for all $r \geq 5$.
The bubble identity normalization $1/\qint{2j+1}$ is therefore perfectly
finite at this boundary.  However, the \emph{next} quantum integer
$\qint{2j+2} = \qint{(2r+2)/3}$ approaches the dangerous point
$\qint{r} = 0$ as~$j$ increases.  More precisely, the genuine complications
arise at the \emph{representation} boundary $j = (r-2)/2$ (outside the
BLM admissibility domain), where $\qint{2j+2} = \qint{r} = 0$.  This
causes the following effects:
\begin{itemize}[nosep]
  \item Recoupling identities involving sums over intermediate spins
    up to $j+1$---such as the Biedenharn--Elliott identity and
    orthogonality relations for $6j$ symbols---encounter vanishing quantum
    dimensions in their summation range.
  \item Quantum $6j$ symbols whose entries involve spins at or beyond the
    truncation bound may have singular Racah formula evaluations,
    since quantum factorials $\qfact{n}$ with $n \geq r$ contain vanishing
    factors ($\qint{r} = 0$).
\end{itemize}
For spins in the range $(r-2)/3 < j \leq (r-2)/2$, the spin-$j$
representation of $U_q(\mathfrak{su}(2))$ exists, but the BLM supercharge
is not well-defined via the Racah formula.  A natural question is whether
there is a well-defined limiting procedure as $j$ approaches $(r-2)/3$
from below, and what the BLM spectrum looks like near this boundary.

\paragraph{(b) Spectral gap dependence on~$r$.}
For fixed admissible~$j$, how does the spectral gap of
$H_q = \{Q_q, Q_q^\dagger\}$ depend on the level~$r$?  In particular:
does the gap remain bounded away from zero for all $r \geq 3j+2$,
or does it close as $r$ decreases toward the admissibility bound?
Since $q \to 1$ as $r \to \infty$, one expects the gap to approach
the $q = 1$ value in this limit; the question concerns the behavior at
small~$r$.

\paragraph{(c) BPS degeneracy at root of unity.}
The Witten index $\Tr\bigl((-1)^F\bigr)$ is a topological invariant of
the SUSY algebra and should be independent of~$q$ (and hence of~$r$).
However, the \emph{detailed} BPS spectrum---not just the index but the
multiplicities of BPS multiplets---may change at root of unity.  Are the
BPS multiplicities at $q = e^{2\pi i/r}$ the same as for real $q > 0$?

\paragraph{(d) Perturbative expansion and reality of $q$-$3j$ symbols.}
At first sight, the root-of-unity $q$-$3j$ symbols appear to be generically
complex.  However, closer inspection reveals that the quantum $3j$ symbols
$C_q^j(m_1, m_2, m_3)$ are in fact \emph{real-valued} for admissible~$j$.
The argument is as follows.  Under the admissibility condition $r \geq 3j+2$,
every quantum integer $\qint{n}$ appearing in the Racah formula satisfies
$1 \leq n \leq 3j+1 < r$.  Since
\begin{equation}\label{eq:qint-positivity}
  \qint{n} \;=\; \frac{\sin(n\pi/r)}{\sin(\pi/r)}
\end{equation}
and $0 < n\pi/r < \pi$ for $1 \leq n < r$, each such quantum integer is
\emph{positive real}.  The quantum factorials, being products of positive
real numbers, are positive real.  The triangle coefficient $\Delta(j,j,j)$
is a square root of a ratio of positive reals and is therefore itself real.
The Racah formula sum, which involves only ratios of such quantum factorials
multiplied by the alternating sign $(-1)^s$, produces real values.

This reality has three consequences.  First, the perturbative expansion is
better behaved than the naive ``complex $3j$'' picture would suggest.
Second, the combinatorial interpretation of individual Feynman diagram
amplitudes is preserved.  Three open sub-questions remain:
\begin{enumerate}[label=(\roman*)]
  \item \emph{Weight modification:} the quantum dimensions $\qint{2j+1}$
    differ from their classical values $2j+1$, modifying the weight of each
    Feynman diagram relative to the $q = 1$ case.  What is the effect on
    the melonic dominance hierarchy?
  \item \emph{Finite truncation effects:} the spin sum in the BLM Feynman
    expansion is truncated at $j \leq (r-2)/3$, whereas the $q = 1$ sum
    extends to infinity.  Do the truncation effects produce qualitatively
    new phenomena (e.g., oscillatory corrections, modified SD equations)?
  \item \emph{Sign structure:} while the $q$-$3j$ symbols are real, they need
    not be positive.  Is the sign structure of the root-of-unity Feynman
    diagrams the same as at $q = 1$, or does it differ?
\end{enumerate}


%% ==================================================================
\subsection{Open Problem 3: Spin content reconciliation}\label{sec:OP3}

\begin{openproblem}[Spin content mismatch]\label{op:spin-content}
The BLM model uses only odd integer spins $j \geq 1$.  The Turaev--Viro
state sum uses all half-integer spins $l = 0, \tfrac{1}{2}, 1, \tfrac{3}{2},
\ldots, (r-2)/2$.  How can a BLM-to-TV connection account for this mismatch?
\end{openproblem}

One natural proposal is to interpret the BLM model at fixed odd integer~$j$
as a \emph{single-coloring sector} of a TV-like state sum, in which all
edge labels in a triangulation equal~$j$.  For this interpretation to be
consistent, the coloring $(j, j, j)$ must be TV-admissible.  The three
TV admissibility conditions are:
\begin{enumerate}[label=(\roman*)]
  \item \emph{Triangle inequality:} $|a - b| \leq c \leq a + b$.  For
    $(j, j, j)$, this reduces to $0 \leq j \leq 2j$, which is trivially
    satisfied.
  \item \emph{Integrality:} $a + b + c \in \Z$.  Since $3j$ is an integer
    for integer~$j$, this is satisfied.
  \item \emph{Level truncation:} $a + b + c \leq r - 2$.  For $(j, j, j)$,
    this requires
    \begin{equation}\label{eq:TV-level-truncation}
      3j \;\leq\; r - 2\,,
      \qquad\text{i.e.,}\qquad
      j \;\leq\; \frac{r-2}{3}\,.
    \end{equation}
\end{enumerate}

\noindent
A key observation is that this level truncation condition \emph{coincides
exactly} with the BLM operative admissibility bound~\eqref{eq:admissibility-recap}.
Consequently, for every BLM-admissible~$j$, the single-coloring $(j, j, j)$
is automatically TV-admissible, and the single-coloring sector interpretation
is consistent at the level of admissibility.

The range of representation-admissible but BLM-inadmissible spins
($(r-2)/3 < j \leq (r-2)/2$) is the regime where the spin-$j$ representation
exists but the BLM supercharge is \emph{not} well-defined via the Racah formula.
\begin{example}
Consider $r = 9$, $j = 3$.  This is an odd integer satisfying
$j \leq 3.5 = (r-2)/2$, so the spin-$3$ representation of
$U_q(\mathfrak{su}(2))$ exists.  However, $3j = 9 > 7 = r-2$, so
$(3, 3, 3)$ is \emph{not} TV-admissible.  Moreover, $\qfact{3j+1} = \qfact{10}$
contains the factor $\qint{9} = \qint{r} = 0$, making $\Delta(3,3,3)$
undefined.  In this regime, the BLM-to-TV question does not arise because the
BLM model itself is not well-defined.
\end{example}

Even within the admissible range $j \leq (r-2)/3$, the single-coloring
sector interpretation faces two caveats:
\begin{enumerate}[label=(C\arabic*)]
  \item \textbf{Topology mismatch.}
    A single-coloring sector of a TV state sum applied to a specific
    triangulation gives a single amplitude.  The BLM model, by contrast,
    produces a perturbative series summing over all Feynman diagram
    topologies.  Any comparison must account for the sum over diagram
    topologies on the BLM side and triangulations on the TV side.

  \item \textbf{No triangulation independence.}
    A TV state sum restricted to a single coloring is \emph{not}
    triangulation-independent.  The topological invariance of the full TV
    state sum relies on the completeness of the sum over all admissible
    colorings.  Therefore, the single-coloring sector interpretation does
    \emph{not} yield a topological invariant.
\end{enumerate}

The precise nature of the BLM-to-TV relationship thus remains open, even
for admissible spins.


%% ==================================================================
\subsection{The $r \to \infty$ limit}\label{sec:r-limit}

As $r \to \infty$, we have $q = e^{2\pi i/r} \to 1$.  In this limit, the
root-of-unity model should recover the $q = 1$ BLM model of Part~I
(\cref{sec:euclidean}).  We now describe the three features of this limit
precisely.

\subsubsection*{(L1) Representation truncation disappears}

The Racah admissibility bound $(r-2)/3 \to \infty$ and the representation
bound $(r-2)/2 \to \infty$, so both constraints become vacuous for any
fixed~$j$.  As $q \to 1$, all $q$-deformed quantities converge to their
classical $q = 1$ values:
\begin{equation}\label{eq:classical-limits}
  \qint{2l+1} \;\longrightarrow\; 2l+1\,,
  \qquad
  \threej{j}{j}{j}{m_1}{m_2}{m_3}_{\!q}
  \;\longrightarrow\;
  \threej{j}{j}{j}{m_1}{m_2}{m_3},
  \qquad
  \sixj{a}{b}{c}{d}{e}{f}_{\!q}
  \;\longrightarrow\;
  \sixj{a}{b}{c}{d}{e}{f}.
\end{equation}
The truncated representation category of $U_q(\mathfrak{su}(2))$ at root of
unity is replaced by the unrestricted classical representation theory of
$\mathfrak{su}(2)$.  In particular, the root-of-unity-specific complications
---boundary effects from $\qint{r} = 0$, level truncation---vanish
identically.

\subsubsection*{(L2) Turaev--Viro regularization is removed}

The Turaev--Viro state sum $\TV_r(M)$ provides a finite regularization of
the (divergent) Ponzano--Regge partition function $Z_{\PR}(M)$.  As
$r \to \infty$, the regularization is removed: the truncation of the spin
sum is lifted, and the quantum dimensions $\qint{2l+1} \to 2l+1$.  The
Ponzano--Regge sum
\begin{equation}\label{eq:PR-divergent}
  Z_{\PR}(\mathcal{T})
  \;=\;
  \sum_{\{j_e\}} \prod_e (-1)^{2j_e}(2j_e + 1)
  \prod_t \sixj{j_1}{j_2}{j_3}{j_4}{j_5}{j_6}_{\!t}
\end{equation}
is formally divergent for closed $3$-manifolds and requires independent
regularization; the TV sum at finite~$r$ \emph{is} the standard such
regularization~\cite{TV92, PonzanoRegge}.

The cosmological constant of the associated $3$-dimensional gravity
is~\cite{MizoguchiTada}
\begin{equation}\label{eq:cosmological-constant}
  \Lambda \;=\; \frac{4\pi^2}{(r-2)^2}
  \;=\; \frac{4\pi^2}{k^2} + O(k^{-4})\,,
\end{equation}
where $k = r - 2$ is the Chern--Simons level in standard TV conventions.
As $r \to \infty$, $\Lambda \to 0$, recovering flat three-dimensional
Euclidean gravity (the Ponzano--Regge regime).

\subsubsection*{(L3) BLM model recovery}

For the BLM model with fixed admissible~$j$, as $q \to 1$:
\begin{itemize}[nosep]
  \item The $q$-$3j$ symbols $C_q^j \to C^j$, the classical (real-valued)
    Clebsch--Gordan coefficients.
  \item The supercharge $Q_q \to Q$, the $q = 1$ supercharge of
    Part~I.
  \item The Hamiltonian $H_q \to H = \{Q, Q^\dagger\}$ of the original
    BLM SUSY quantum mechanics.
\end{itemize}

\begin{remark}[SUSY is not ``recovered'']
\label{rem:susy-all-q}
The SUSY structure $H_q \geq 0$ is present at \emph{all} values of $q$,
including roots of unity (per the admissibility analysis of
\cref{sec:root-of-unity}).  It does not need to be ``recovered'' in the
$r \to \infty$ limit.  What \emph{is} recovered is the specific algebraic
simplifications of the $q = 1$ case: the reality of all $3j$ coefficients
(which, as discussed in \cref{sec:OP2}, already holds at root of unity for
admissible~$j$), the absence of any truncation constraints, and the
polynomial (rather than modified) asymptotics of the $6j$ symbols.
\end{remark}

\begin{remark}[Fixed $j$, no thermodynamic limit]
\label{rem:fixed-j}
The BLM model parameter $j$ remains \emph{fixed} throughout the
$r \to \infty$ limit; it does not scale with~$r$.  The limit therefore does
not involve any thermodynamic or continuum limit in the BLM model itself.
It is the surrounding gravitational interpretation---TV versus PR, positive
versus zero cosmological constant---that changes as $r$ increases.
\end{remark}


%% ==================================================================
\subsection{Summary of open problems}\label{sec:open-summary}

We collect the status of the three open problems and the $r \to \infty$
limit in~\cref{tab:open-summary}.

\begin{table}[ht]
\centering
\caption{Summary of open problems and the $r \to \infty$ limit.}
\label{tab:open-summary}
\begin{tabular}{lp{7cm}l}
  \toprule
  \textbf{Label} & \textbf{Question} & \textbf{Status} \\
  \midrule
  OP1 & BLM Feynman diagrams $\leftrightarrow$ TV invariants &
    Open (requires fermionic GFT) \\
  OP2(a) & Boundary behavior near $j = (r-2)/3$ &
    Open \\
  OP2(b) & Spectral gap dependence on $r$ &
    Open \\
  OP2(c) & BPS degeneracy at root of unity &
    Open (Witten index protected) \\
  OP2(d) & Reality and sign structure of perturbative expansion &
    Reality established; sub-questions open \\
  OP3 & Spin content reconciliation (BLM vs.\ TV) &
    Admissibility coincidence established; \\
       & & caveats (C1)--(C2) unresolved \\
  \midrule
  L1 & Truncation $\to$ vacuous &
    Established \\
  L2 & TV $\to$ PR (divergent) &
    Established \\
  L3 & $q$-BLM $\to$ BLM &
    Established \\
  \bottomrule
\end{tabular}
\end{table}

\begin{remark}[Erratum on the original OP2]
\label{rem:erratum-op2}
An earlier version of node~1.4.4 in the AF proof tree contained an Open
Problem~2 asking whether the $q$-BLM model can be given a consistent
quantum-mechanical interpretation at root of unity, presupposing that SUSY
breaks.  This was based on the false premise that $\{Q_q, Q_q^\dagger\}$
can have negative eigenvalues at root of unity.  As corrected in the AF
ledger (node~1.4.3), the positivity $\{Q_q, Q_q^\dagger\} \geq 0$ is a
tautological consequence of the definition of the Hilbert space adjoint
and holds for \emph{all}~$q$.  The three ``remedies'' proposed in the
original---modifying the inner product, using $Q^2 = 0$ cohomologically
without $H \geq 0$, and restricting to real $q$-$3j$ symbols---were all
addressing a non-problem.  The revised OP2 above identifies the genuine
open problems at root of unity, which are representation-theoretic and
spectral in nature.

Additionally, the original OP2(a) claimed that $1/\qint{2j+1}$ diverges
at $j = (r-2)/2$; this was false since $\qint{2j+1} = \qint{r-1} = 1$
at that boundary.  The corrected OP2(a) identifies the genuine boundary
mechanism: $\qint{2j+2} = \qint{r} = 0$, which affects recoupling
identities and $6j$ symbol evaluations near the truncation bound.
\end{remark}
