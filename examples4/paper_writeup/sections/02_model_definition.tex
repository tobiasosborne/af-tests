%% Section 2: The q-Deformed BLM Model
%% Covers AF nodes 1.1 (well-definedness + SUSY), 1.1.1 (braided fermions), 1.1.2 (vertex normalization)

\section{The $q$-Deformed BLM Model}
\label{sec:model}

\afnode{1.1}

This section defines the $q$-deformed BLM model from scratch.
We begin with the classical ($q=1$) construction of Biggs, Lin, and
Maldacena~\cite{BLM}, then introduce the quantum-group deformation.
The reader is assumed to have a standard graduate physics background
(quantum mechanics, second quantization, Lie algebras) but no prior
exposure to supersymmetry or quantum groups.

%%------------------------------------------------------------
\subsection{Fermionic Fock space}
\label{sec:fock}
%%------------------------------------------------------------

Fix an odd integer $j \geq 1$ and set
\begin{equation}
  N \;=\; 2j+1.
  \label{eq:N}
\end{equation}
The magnetic quantum numbers of the spin-$j$ representation of
$\mathfrak{su}(2)$ are $m \in \{-j, -j+1, \ldots, j\}$, a set of
cardinality~$N$.

\begin{definition}[Fermionic Fock space]
\label{def:fock}
Let $\psi^\dagger_m$ and $\psi_m$ ($m = -j, \ldots, j$) be creation and
annihilation operators satisfying the canonical anticommutation relations
\begin{equation}
  \bigl\{\psi_m,\, \psi^\dagger_{m'}\bigr\} = \delta_{m,m'},
  \qquad
  \bigl\{\psi_m,\, \psi_{m'}\bigr\} = 0,
  \qquad
  \bigl\{\psi^\dagger_m,\, \psi^\dagger_{m'}\bigr\} = 0.
  \label{eq:CAR}
\end{equation}
The \emph{fermionic Fock space} is the exterior algebra
\begin{equation}
  \Hilbert \;=\; \textstyle\bigwedge\nolimits^{\!*}(\C^N)
  \;\cong\; \C^{2^N},
  \label{eq:fock-space}
\end{equation}
generated by acting with creation operators on the vacuum $\lvert 0\rangle$
(annihilated by all $\psi_m$).
The dimension $2^N$ grows exponentially with $j$.
\end{definition}

\begin{remark}[Finite-dimensional system]
Because the index set $\{-j,\ldots,j\}$ is finite, the Fock space is
finite-dimensional.  Every operator on $\Hilbert$ can in principle be
written as a $2^N \times 2^N$ matrix.  There are no
ultraviolet or infrared divergences, and all spectral questions reduce
to finite-dimensional linear algebra.
\end{remark}

%%------------------------------------------------------------
\subsection{The Wigner $3j$ symbols}
\label{sec:3j}
%%------------------------------------------------------------

The coupling coefficients in the BLM model are the Wigner $3j$ symbols of
$\mathfrak{su}(2)$.  We write
\begin{equation}
  C^j_{m_1, m_2, m_3}
  \;\equiv\;
  \threej{j}{j}{j}{m_1}{m_2}{m_3},
  \label{eq:3j-def}
\end{equation}
which is nonzero only when the magnetic quantum numbers satisfy
$m_1 + m_2 + m_3 = 0$.  We collect the properties we need:

\begin{enumerate}[label=(\roman*)]
\item \textbf{Reality.}
  All $3j$ symbols with three equal integer or half-integer spins are real
  (they are given by explicit combinatorial formulas involving only
  factorials and square roots of rationals).

\item \textbf{Symmetry under column permutations.}
  Under permutation $\sigma$ of the three columns,
  \begin{equation}
    \threej{j}{j}{j}{m_{\sigma(1)}}{m_{\sigma(2)}}{m_{\sigma(3)}}
    \;=\;
    (-1)^{j+j+j}\,
    \threej{j}{j}{j}{m_1}{m_2}{m_3}
    \;=\;
    (-1)^{3j}\,
    \threej{j}{j}{j}{m_1}{m_2}{m_3}.
    \label{eq:3j-perm}
  \end{equation}
  Since $j$ is odd, $3j$ is also odd, so $(-1)^{3j} = -1$.
  Thus $C^j_{m_1,m_2,m_3}$ is \emph{totally antisymmetric}
  in $(m_1,m_2,m_3)$.

\item \textbf{Orthogonality (bubble identity).}
  The $3j$ symbols satisfy
  \begin{equation}
    \sum_{m_1, m_2} C^j_{m_1,m_2,m}\, C^j_{m_1,m_2,m'}
    \;=\;
    \frac{\delta_{m,m'}}{2j+1}.
    \label{eq:bubble-classical}
  \end{equation}
  Diagrammatically, two vertices joined by two propagators give a single
  propagator weighted by $1/N$.
\end{enumerate}

%%------------------------------------------------------------
\subsection{The classical BLM supercharge and Hamiltonian}
\label{sec:classical-BLM}
%%------------------------------------------------------------

\begin{definition}[BLM supercharge {\cite{BLM}}]
\label{def:supercharge}
The \emph{supercharge} of the BLM model at coupling $J > 0$ is the
cubic fermionic operator
\begin{equation}
  Q
  \;=\;
  \frac{1}{3!}\,\sqrt{2JN}\;\,
  \sum_{\substack{m_1,m_2,m_3 \\ m_1+m_2+m_3=0}}
  C^j_{m_1,m_2,m_3}\;\,
  \psi_{m_1}\,\psi_{m_2}\,\psi_{m_3}\,.
  \label{eq:Q-classical}
\end{equation}
The factor $1/3!$ accounts for the antisymmetric summation over all
$(m_1,m_2,m_3)$, and $\sqrt{2JN}$ is the vertex normalization
(see \cref{sec:normalization}).
\end{definition}

The fundamental algebraic property is:

\begin{proposition}[$\mathcal{N}{=}2$ SUSY algebra]
\label{prop:susy}
The supercharge satisfies
\begin{equation}
  Q^2 = 0.
  \label{eq:Q-squared}
\end{equation}
The \emph{Hamiltonian} $H = \{Q, Q^\dagger\}$ is non-negative,
\begin{equation}
  H \;=\; \bigl\{Q,\, Q^\dagger\bigr\} \;\geq\; 0,
  \label{eq:H-def}
\end{equation}
and the pair $(Q, Q^\dagger)$ generate an $\mathcal{N}{=}2$
supersymmetry algebra.
\end{proposition}

\begin{proof}
\textbf{Nilpotency.}
Expanding $Q^2$ gives a sum over six indices $m_1, \ldots, m_6$ of
the product $C^j_{m_1 m_2 m_3}\, C^j_{m_4 m_5 m_6}\,
\psi_{m_1}\cdots\psi_{m_6}$.
By property~\eqref{eq:3j-perm}, each $C^j$ is totally antisymmetric
in its indices.  The six-fermion monomial
$\psi_{m_1}\cdots\psi_{m_6}$ is totally antisymmetric under
exchange of any pair of indices (from the CAR).  But the product of
two totally antisymmetric rank-3 tensors, contracted with a totally
antisymmetric rank-6 tensor, must vanish by an elementary
counting argument: the symmetrizer and antisymmetrizer project
onto orthogonal subspaces.  Concretely, exchanging $(m_1,m_2,m_3)
\leftrightarrow (m_4,m_5,m_6)$ gives a factor $(-1)^{3 \cdot 3}
= -1$ from anticommuting six fermions (nine pairwise transpositions, but
only the three cross-set transpositions contribute a sign; more carefully,
the two-$C$ tensor is symmetric under block exchange while the six-fermion
string is antisymmetric), so $Q^2 = -Q^2 = 0$.

\smallskip
\noindent\textbf{Positivity.}
For any state $\lvert v\rangle$,
\begin{equation}
  \langle v \rvert\, H\, \lvert v\rangle
  \;=\;
  \langle v \rvert\, Q\, Q^\dagger\, \lvert v\rangle
  +
  \langle v \rvert\, Q^\dagger Q\, \lvert v\rangle
  \;=\;
  \bigl\lVert Q^\dagger \lvert v\rangle \bigr\rVert^2
  +
  \bigl\lVert Q \lvert v\rangle \bigr\rVert^2
  \;\geq\; 0.
  \label{eq:H-positive}
\end{equation}
This is tautological: no properties of $Q$ beyond linearity are used.
\end{proof}

\begin{definition}[BPS states]
\label{def:bps}
A state $\lvert v\rangle$ is called \emph{BPS} (Bogomol'nyi--Prasad--Sommerfield)
if $H\lvert v\rangle = 0$.  Equivalently, by~\eqref{eq:H-positive},
\begin{equation}
  \ker(H) \;=\; \ker(Q) \,\cap\, \ker(Q^\dagger).
  \label{eq:bps}
\end{equation}
BPS states are annihilated by both supercharges and sit at zero energy.
Their count is a robust quantity protected by supersymmetry.
\end{definition}

%%------------------------------------------------------------
\subsection{Symmetries of the classical model}
\label{sec:symmetries}
%%------------------------------------------------------------

The classical BLM model possesses two important symmetries:

\begin{enumerate}[label=(\alph*)]
\item \textbf{$\mathrm{SU}(2)$ invariance.}
The angular momentum operators
\begin{equation}
  J_a = \sum_{m,m'} (T_a)_{m,m'}\, \psi^\dagger_m \psi_{m'},
  \qquad a \in \{1,2,3\},
  \label{eq:angular-momentum}
\end{equation}
where $(T_a)_{m,m'}$ are the spin-$j$ representation matrices of
$\mathfrak{su}(2)$, commute with $Q$ by the Wigner--Eckart theorem:
\begin{equation}
  [Q, J_a] = 0, \qquad [H, J_a] = 0.
  \label{eq:SU2-symmetry}
\end{equation}
Hence the spectrum of $H$ decomposes into $\mathrm{SU}(2)$ multiplets.

\item \textbf{$R$-charge.}
The fermion number operator $N_\psi = \sum_m \psi^\dagger_m \psi_m$
satisfies $[Q, N_\psi] = -3Q$ (since $Q$ is cubic in annihilation operators),
so $R = N_\psi / 3$ is the $R$-charge of the $\mathcal{N}{=}2$ algebra.
The Hilbert space decomposes into sectors of definite $R$-charge, and $Q$
lowers $R$ by one unit.
\end{enumerate}

%%------------------------------------------------------------
\subsection{Quantum groups and $q$-deformation}
\label{sec:q-deformation}
%%------------------------------------------------------------

We now deform the model by replacing the classical $3j$ symbols with
their quantum-group counterparts.  We first recall the necessary
algebraic background.

\begin{definition}[Quantum integer]
\label{def:q-integer}
For $q \in \C \setminus \{0\}$ with $q \neq \pm 1$, the
\emph{quantum integer} is
\begin{equation}
  \qint{n} \;=\; \frac{q^n - q^{-n}}{q - q^{-1}}\,.
  \label{eq:q-integer}
\end{equation}
When $q$ is real and positive, $\qint{n}$ is real and positive for $n > 0$.
In the limit $q \to 1$, we recover $\qint{n} \to n$.
\end{definition}

\begin{definition}[Quantum factorial and quantum dimension]
\label{def:q-factorial}
The \emph{quantum factorial} is $\qfact{n} = \qint{1}\,\qint{2}\cdots\qint{n}$,
with $\qfact{0} = 1$.  The \emph{quantum dimension} of the spin-$j$
representation is $\qint{2j+1} = \qint{N}$.
\end{definition}

The quantum integers arise naturally from the representation theory
of the quantum group $U_q(\mathfrak{su}(2))$, the one-parameter
deformation of the universal enveloping algebra of $\mathfrak{su}(2)$
introduced by Drinfeld and Jimbo.  The key point is that
$U_q(\mathfrak{su}(2))$ has the same finite-dimensional representations
as $\mathfrak{su}(2)$ (labeled by spin $j$), but the Clebsch--Gordan
coefficients and $3j$ symbols acquire $q$-dependent corrections.

\begin{definition}[Quantum $3j$ symbols]
\label{def:q-3j}
The \emph{quantum $3j$ symbols} $C^{j,q}_{m_1,m_2,m_3}$ are the
Clebsch--Gordan coupling coefficients of $U_q(\mathfrak{su}(2))$,
expressed in $3j$-symbol form.  They satisfy:
\begin{enumerate}[label=(\roman*)]
  \item \textbf{$q$-selection rule:} $C^{j,q}_{m_1,m_2,m_3} = 0$
    unless $m_1 + m_2 + m_3 = 0$.
  \item \textbf{Column-permutation symmetry:} For $j_1 = j_2 = j_3 = j$
    with $3j$ odd,
    \begin{equation}
      C^{j,q}_{m_{\sigma(1)}, m_{\sigma(2)}, m_{\sigma(3)}}
      \;=\;
      \sgn(\sigma)\; C^{j,q}_{m_1, m_2, m_3}\,,
      \label{eq:q-3j-antisymmetry}
    \end{equation}
    i.e., $C^{j,q}$ is totally antisymmetric, exactly as in the
    classical case.  This follows from the general column-permutation
    formula for quantum $3j$ symbols (Groza--Kachurik--Klimyk, 1990).
  \item \textbf{Reality:} For $q > 0$ real, the quantum $3j$ symbols
    are real.
  \item \textbf{Classical limit:} $C^{j,q}_{m_1,m_2,m_3} \to
    C^j_{m_1,m_2,m_3}$ as $q \to 1$.
\end{enumerate}
\end{definition}

\begin{definition}[$q$-Bubble identity]
\label{def:q-bubble}
The quantum $3j$ symbols satisfy the orthogonality relation
\begin{equation}
  \sum_{m_1,m_2}
  C^{j,q}_{m_1,m_2,m}\;\,
  C^{j,q}_{m_1,m_2,m'}
  \;=\;
  \frac{\delta_{m,m'}}{\qint{2j+1}}
  \;=\;
  \frac{\delta_{m,m'}}{\qint{N}}\,,
  \label{eq:q-bubble}
\end{equation}
which is the quantum deformation of~\eqref{eq:bubble-classical},
reducing to it as $q \to 1$ (since $\qint{N} \to N$).
\end{definition}

%%------------------------------------------------------------
\subsection{The $q$-deformed BLM model}
\label{sec:q-BLM}
%%------------------------------------------------------------

With these ingredients, the deformation is straightforward:

\begin{definition}[$q$-BLM supercharge]
\label{def:q-supercharge}
For $q > 0$ real, the \emph{$q$-deformed BLM supercharge} is
\begin{equation}
  Q_q
  \;=\;
  \frac{1}{3!}\,\sqrt{2JN}\;\,
  \sum_{\substack{m_1,m_2,m_3 \\ m_1+m_2+m_3=0}}
  C^{j,q}_{m_1,m_2,m_3}\;\,
  \psi_{m_1}\,\psi_{m_2}\,\psi_{m_3}\,,
  \label{eq:Q-q}
\end{equation}
using the same canonical fermions $\psi_m$ as in the classical model.
The \emph{$q$-deformed Hamiltonian} is
\begin{equation}
  H_q \;=\; \bigl\{Q_q,\, Q_q^\dagger\bigr\}.
  \label{eq:H-q}
\end{equation}
\end{definition}

A central point is that the $q$-deformed model inherits the SUSY
algebra from precisely the same algebraic mechanism as the classical
model:

\begin{proposition}[$\mathcal{N}{=}2$ SUSY at all $q$]
\label{prop:susy-q}
For any $q > 0$ real:
\begin{enumerate}[label=(\roman*)]
\item $Q_q^2 = 0$ \;\emph{(nilpotency)}.
\item $Q_q^\dagger = (Q_q)^*$ is the Fock-space adjoint of $Q_q$
  \;\emph{(Hermiticity)}.
\item $H_q = \{Q_q, Q_q^\dagger\} \geq 0$ \;\emph{(non-negative spectrum)}.
\end{enumerate}
Hence $(Q_q, Q_q^\dagger, H_q)$ form an $\mathcal{N}{=}2$ supersymmetry
algebra for every value of the deformation parameter.
\end{proposition}

\begin{proof}
(i) The nilpotency proof is identical to \cref{prop:susy}:
the quantum $3j$ symbols $C^{j,q}_{m_1,m_2,m_3}$ are totally
antisymmetric by~\eqref{eq:q-3j-antisymmetry} (this is a purely
algebraic identity valid for all~$q$), and the fermions are Grassmann.
The argument that $Q_q^2 = 0$ uses only the total antisymmetry of the
coefficients and the canonical anticommutation relations; it does not
depend on the specific numerical values of the $3j$ symbols.

\smallskip\noindent
(ii) For $q > 0$ real, the quantum $3j$ symbols are real
(\cref{def:q-3j}(iii)), and $\sqrt{2JN}$ is real.  Therefore
$Q_q^\dagger$ is obtained by replacing each $\psi_m$ with
$\psi_m^\dagger$, which is the standard Fock-space adjoint.

\smallskip\noindent
(iii) Identical to~\eqref{eq:H-positive}: $\langle v | H_q | v \rangle =
\|Q_q^\dagger |v\rangle\|^2 + \|Q_q |v\rangle\|^2 \geq 0$ for all
$|v\rangle$.
\end{proof}

\begin{remark}[What changes, what does not]
\label{rem:what-changes}
Under $q$-deformation, the \emph{algebraic structure} (SUSY,
nilpotency, positivity) is preserved exactly, while the \emph{numerical
values} of the matrix elements of $H_q$ change.  In particular:
\begin{itemize}
\item The spectrum of $H_q$ is $q$-dependent: energy levels shift,
  degeneracies may split.
\item The count of BPS states ($E=0$ ground states) may change with $q$.
\item The large-$j$ asymptotics of Feynman diagrams change dramatically
  (this is the subject of \cref{sec:euclidean,sec:hyperbolic,sec:root-of-unity}).
\end{itemize}
\end{remark}

%%------------------------------------------------------------
\subsection{Open problem: braided fermions and $U_q$ covariance}
\label{sec:braided-fermions}
%%------------------------------------------------------------

\afnode{1.1.1}

The $q$-supercharge~\eqref{eq:Q-q} combines quantum $3j$ symbols
(which belong to the representation theory of $U_q(\mathfrak{su}(2))$)
with \emph{ordinary, undeformed} fermions satisfying the canonical
anticommutation relations~\eqref{eq:CAR}.  This is a deliberate choice,
and we pause to explain its mathematical status.

The classical BLM model is $\mathrm{SU}(2)$-invariant: the supercharge $Q$
is an intertwiner, meaning it commutes with the $\mathrm{SU}(2)$ action
(\cref{eq:SU2-symmetry}).  One might ask whether the $q$-deformed model
is similarly $U_q(\mathfrak{su}(2))$-covariant.  The answer is
\emph{no, not with ordinary fermions}, for the following reason.

The quantum group $U_q(\mathfrak{su}(2))$ is a Hopf algebra with a
\emph{non-cocommutative} coproduct~$\Delta$.  In the tensor product
of representations, the correct notion of ``antisymmetric subspace'' is
defined using the \emph{braiding} $c = \tau \circ R$ (where $\tau$ is
the flip map and $R$ is the universal $R$-matrix), not the naive flip
$\tau$.  The ordinary fermionic Fock space $\bigwedge^*(\C^N)$ is
built from the naive flip, and is therefore \emph{not} a
$U_q(\mathfrak{su}(2))$-submodule of the tensor algebra when
$q \neq 1$.

To restore full quantum-group covariance, one would need \emph{braided
fermions} (also called $q$-fermions): operators whose exchange relations
incorporate the $R$-matrix.  Such constructions exist in the abstract
framework of braided tensor categories (Majid, 1995; Fiore, 1996;
Woronowicz, 1996), but they have not been applied to construct a specific
$q$-BLM model.

\begin{openproblem}[Braided fermion BLM model]
\label{op:braided}
Construct a version of the BLM model using braided fermions
$\hat\psi_m$ satisfying $R$-matrix exchange relations, and determine:
\begin{enumerate}[label=(\alph*)]
  \item whether the resulting supercharge $\hat{Q}_q$ is nilpotent
    ($\hat{Q}_q^2 = 0$);
  \item whether $\hat{Q}_q$ defines an intertwiner in the braided
    tensor category, yielding full $U_q(\mathfrak{su}(2))$ covariance;
  \item how the braided Fock space differs from
    $\bigwedge^*(\C^N)$, and what constraints this imposes on $j$
    and $q$.
\end{enumerate}
\end{openproblem}

\begin{remark}[Pragmatic status of the open problem]
The absence of $U_q$ covariance does \emph{not} invalidate any of the
results in this paper.  The $q$-BLM model~\eqref{eq:Q-q}--\eqref{eq:H-q}
is a perfectly well-defined quantum-mechanical system for every $q > 0$:
the SUSY algebra holds (\cref{prop:susy-q}), the spectrum is computable,
and the Feynman diagram expansion is well-defined.  The braided-fermion
question is a natural mathematical refinement whose answer would be
interesting for the connection to topological quantum field theory, but
is not required for any of the physical results that follow.
\end{remark}

%%------------------------------------------------------------
\subsection{Vertex normalization}
\label{sec:normalization}
%%------------------------------------------------------------

\afnode{1.1.2}

The coupling constant in the supercharge~\eqref{eq:Q-q} deserves
careful discussion.  We present two natural normalization conventions
and explain how they affect the large-$j$ analysis.

\subsubsection{The classical self-energy}

In the Feynman diagram expansion of the self-energy $\Sigma(\tau)$
at leading (melonic) order, two vertices are joined by two internal
propagators and one ``pillow'' contraction.  Each vertex contributes a
factor of the coupling prefactor, and the internal sum over magnetic
quantum numbers produces the bubble~\eqref{eq:bubble-classical}.

With the classical normalization $\sqrt{2JN}$, the leading self-energy is
\begin{equation}
  \Sigma_{\text{mel}}^{(q=1)} \;=\; J,
  \label{eq:sigma-classical}
\end{equation}
independent of $N$ (equivalently, of $j$).  The calculation is:
two factors of $(\sqrt{2JN})^2 / (3!)^2 = 2JN/36$ from the vertices,
combinatorial factors from Wick contraction, and a bubble
giving $1/N$ from~\eqref{eq:bubble-classical}, combine to yield a
result proportional to~$J$ with no residual $N$-dependence.  This
$N$-independence is what allows the Schwinger--Dyson (SD) equations to
close in the same form as the SYK model~\cite{SYK-Fu}.

\subsubsection{Two normalization choices for the $q$-model}

When $q \neq 1$, the bubble~\eqref{eq:q-bubble} gives $1/\qint{N}$
instead of $1/N$.  This creates a tension:

\medskip
\noindent\textbf{Choice (a): Classical normalization.}
Keep the prefactor $\sqrt{2JN}$ as in~\eqref{eq:Q-q}. Then the
melonic self-energy becomes
\begin{equation}
  \Sigma_q^{(a)}
  \;=\;
  \frac{JN}{\qint{N}}\,,
  \label{eq:sigma-choice-a}
\end{equation}
which depends on $q$ (through $\qint{N}$) and on $N$.  The SD equations
still close---the self-energy is still proportional to $\delta_{m,m'}$
and the SD equations take the same functional form as SYK---but with a
$q$-dependent effective coupling $J_{\mathrm{eff}} = JN / \qint{N}$.
This reduces to $J$ as $q \to 1$.

\medskip
\noindent\textbf{Choice (b): $q$-Adapted normalization.}
Replace the prefactor by $\sqrt{2J\qint{N}}$:
\begin{equation}
  Q_q^{(b)}
  \;=\;
  \frac{1}{3!}\,\sqrt{2J\qint{N}}\;\,
  \sum_{\substack{m_1,m_2,m_3 \\ m_1+m_2+m_3=0}}
  C^{j,q}_{m_1,m_2,m_3}\;\,
  \psi_{m_1}\,\psi_{m_2}\,\psi_{m_3}\,.
  \label{eq:Q-q-adapted}
\end{equation}
Then the melonic self-energy is
\begin{equation}
  \Sigma_q^{(b)}
  \;=\;
  J,
  \label{eq:sigma-choice-b}
\end{equation}
exactly as in the classical case.  The SD equations close with a
coupling that is manifestly $q$-independent, in precise analogy
with the SYK model.

\begin{remark}[The normalization is a model-definition choice]
\label{rem:normalization-choice}
Neither choice is ``wrong'': they define different quantum-mechanical
systems with the same qualitative features (SUSY, melonic dominance)
but quantitatively different spectra.  The key distinction is pragmatic:
\begin{itemize}
  \item Choice~(a) preserves the original BLM coupling convention and
    gives a $q$-dependent effective coupling.
  \item Choice~(b) preserves $N$-independence of the SD equations and
    gives the cleanest large-$j$ limit.
\end{itemize}
In the remainder of this paper, we will use choice~(b) as the
\emph{default convention} unless stated otherwise.  We note that all
structural results (SUSY, melonic dominance, relation to recoupling
symbols) hold in both conventions; only numerical prefactors change.
\end{remark}

%%------------------------------------------------------------
\subsection{Summary of the model}
\label{sec:model-summary}
%%------------------------------------------------------------

We collect the complete definition of the $q$-deformed BLM model for
reference.

\begin{definition}[The $q$-BLM model: complete specification]
\label{def:q-BLM-complete}
Fix parameters:
\begin{itemize}
  \item An odd integer $j \geq 1$ (spin), with $N = 2j+1$.
  \item A real number $q > 0$, $q \neq 1$ (deformation parameter).
  \item A coupling constant $J > 0$.
\end{itemize}
The model consists of:
\begin{enumerate}[label=(\roman*)]
  \item \textbf{Hilbert space:}
    $\Hilbert = \bigwedge^*(\C^N) \cong \C^{2^N}$
    with canonical fermions $\{\psi_m, \psi^\dagger_{m'}\} =
    \delta_{m,m'}$.
  \item \textbf{Supercharge} (in the $q$-adapted normalization):
    \begin{equation*}
      Q_q = \frac{1}{3!}\,\sqrt{2J\qint{N}}\;\,
      \sum_{m_1+m_2+m_3=0}
      C^{j,q}_{m_1,m_2,m_3}\;\,
      \psi_{m_1}\,\psi_{m_2}\,\psi_{m_3}\,.
    \end{equation*}
  \item \textbf{Hamiltonian:} $H_q = \{Q_q, Q_q^\dagger\} \geq 0$.
  \item \textbf{SUSY algebra:} $Q_q^2 = 0$, \;$(Q_q^\dagger)^2 = 0$,
    \;$H_q = \{Q_q, Q_q^\dagger\}$.
  \item \textbf{Bubble identity:}
    $\sum_{m_1,m_2} C^{j,q}_{m_1,m_2,m}\, C^{j,q}_{m_1,m_2,m'}
    = \delta_{m,m'} / \qint{N}$.
\end{enumerate}
The three geometric regimes of the model are determined by the
choice of $q$: Euclidean ($q = 1$), hyperbolic ($q \in \R_{>0}
\setminus\{1\}$, fixed), and root of unity ($q = e^{2\pi i/r}$, $r$
a positive integer $\geq 3$).
\end{definition}
