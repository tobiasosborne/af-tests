%% Section 5: Part III — Root-of-Unity Regime (q = e^{2\pi i/r})
%% Corresponds to AF Nodes 1.4, 1.4.1, 1.4.2, 1.4.3.
%% Epistemic status: MIXED (established mathematics + open connections).

\section{Part~III: The Root-of-Unity Regime ($q = e^{2\pi i/r}$)}
\label{sec:root-of-unity}
\afnode{1.4}
\mixedstatus

When the deformation parameter $q$ is specialized to a root of unity
$q = e^{2\pi i/r}$ with $r \geq 3$ an integer, the representation theory of
$U_q(\mathfrak{su}(2))$ undergoes a qualitative change: the tower of
spin-$l$ representations truncates to a finite set $l = 0, \half, 1, \ldots,
(r-2)/2$, and the quantum recoupling symbols become the building blocks of
the Turaev--Viro topological invariant~\cite{TV92} --- a mathematically
rigorous state sum for three-dimensional gravity with positive cosmological
constant.

The content of this section divides into three layers of decreasing
epistemic certainty.
\begin{enumerate}[label=(\Alph*),nosep]
  \item \textbf{Established mathematics} (\cref{sec:TV}):
    the Turaev--Viro state sum, its independence of triangulation, and the
    Turaev--Walker identification with Chern--Simons theory at level
    $k = r-2$.  This is textbook material.
  \item \textbf{Structural analogy} (\cref{sec:boulatov}):
    the relationship between BLM Feynman diagrams and the Boulatov group
    field theory (GFT).  The analogy is precise at the level of algebraic
    building blocks but does \emph{not} constitute a duality.
  \item \textbf{SUSY at root of unity} (\cref{sec:susy-root}):
    the survival of $\mathcal{N}=2$ supersymmetry, the admissibility bound
    $r \geq 3j+2$, and the finite truncation of the model.  These results
    are established but emerged only after correcting a significant error in
    the original formulation (see the erratum in \cref{sec:erratum}).
\end{enumerate}


%% ==================================================================
\subsection{The Turaev--Viro state sum}
\label{sec:TV}
\afnode{1.4.1}
\established

We begin by recalling the representation theory of $U_q(\mathfrak{su}(2))$
at a root of unity, and then state the Turaev--Viro invariant.

\subsubsection{Quantum integers at root of unity}

Set $A = e^{i\pi/r}$, so that $q = A^2 = e^{2\pi i/r}$ is a primitive
$r$-th root of unity.  Throughout this section we use the quantum integer
convention
\begin{equation}\label{eq:qint-A}
  \qint{n}_{\!A}
  \;=\; \frac{A^n - A^{-n}}{A - A^{-1}}
  \;=\; \frac{\sin(n\pi/r)}{\sin(\pi/r)}\,.
\end{equation}
This coincides with the convention
$\qint{n}_q = (q^{n/2} - q^{-n/2})/(q^{1/2} - q^{-1/2})$ used
elsewhere in this paper.  The key property at root of unity is that
$\qint{r}_{\!A} = 0$, so any quantum factorial $\qfact{n}$ with $n \geq r$
vanishes.  This forces the truncation of the representation category.

At this root of unity, $U_q(\mathfrak{su}(2))$ admits finitely many
irreducible representations, labelled by spin $l = 0, \half, 1, \ldots,
(r-2)/2$.  The quantum dimension of the spin-$l$ representation is
\begin{equation}\label{eq:qdim-root}
  \qdim{l}
  \;:=\; \qint{2l+1}_{\!A}
  \;=\; \frac{\sin\bigl((2l+1)\pi/r\bigr)}{\sin(\pi/r)}\,,
\end{equation}
which is \emph{strictly positive} for all admissible $l \in \{0, \half,
\ldots, (r-2)/2\}$, since $0 < (2l+1)\pi/r < \pi$ in this range.

\begin{notation}\label{not:l-vs-j}
  The TV coloring labels $l = 0, \half, 1, \ldots, (r-2)/2$ are
  \emph{distinct} from the BLM model parameter $j$ (a fixed odd integer
  $\geq 1$).  We reserve $l$ for TV colorings and $j$ for the BLM spin
  throughout this section.
\end{notation}


\subsubsection{Admissibility and the state sum}

\begin{definition}[TV admissibility]\label{def:TV-admissible}
  Let $\mathcal{T}$ be a triangulation of a closed oriented 3-manifold $M$,
  with edge set~$E$.  A coloring
  $f\colon E \to \{0, \half, 1, \ldots, (r-2)/2\}$ is \emph{admissible}
  if, for every 2-face (triangle) of $\mathcal{T}$ with edge labels
  $(a, b, c)$:
  \begin{enumerate}[label=(\roman*),nosep]
    \item the triangle inequality holds: $|a - b| \leq c \leq a + b$;
    \item $a + b + c \in \Z$ (parity condition);
    \item $a + b + c \leq r - 2$ (level truncation).
  \end{enumerate}
\end{definition}

Condition~(iii) is the hallmark of the root-of-unity regime: it has no
analogue in the $q = 1$ Ponzano--Regge theory, where the spin sum is
unrestricted (and divergent).

\begin{definition}[Turaev--Viro invariant {\cite{TV92}}]
\label{def:TV-invariant}
  Let $\mathcal{T}$ be a triangulation of a closed oriented 3-manifold $M$,
  with vertex set~$V$, edge set~$E$, and tetrahedron set~$\mathrm{Tet}$.
  Define the total quantum order
  \begin{equation}\label{eq:Dr-squared}
    D_r^2
    \;=\; \sum_{l=0}^{(r-2)/2} \qdim{l}^{\,2}
    \;=\; \sum_{l=0}^{(r-2)/2} \qint{2l+1}_{\!A}^2\,.
  \end{equation}
  The \emph{Turaev--Viro invariant} is
  \begin{equation}\label{eq:TV-invariant}
    \TV_r(M)
    \;=\; D_r^{-2|V|}
    \sum_{\substack{f\colon E \to \{0,\ldots,(r-2)/2\}\\f\;\text{admissible}}}
    \;\prod_{e \in E} \qdim{f(e)}
    \;\prod_{t \in \mathrm{Tet}}
    \sixj{j_1}{j_2}{j_3}{j_4}{j_5}{j_6}_{\!\!q}^{\!(t,f)},
  \end{equation}
  where $|V|$ is the number of vertices, the edge weight is the quantum
  dimension $\qdim{f(e)}$, and
  $\{6j\}_q^{(t,f)}$ denotes the quantum $6j$ symbol of tetrahedron~$t$
  with coloring~$f$.  The per-vertex normalization $D_r^{-2|V|}$ follows
  the convention of Turaev--Viro~\cite{TV92}, eq.~(1.1).
\end{definition}

\begin{remark}[Sign conventions]\label{rem:sign-conv}
  The original Turaev--Viro formula uses edge weights
  $w_l = (-1)^{2l}\qint{2l+1}_{\!A}$, where the sign factor
  $(-1)^{2l}$ equals $+1$ for integer~$l$ and $-1$ for half-integer~$l$.
  Some references (e.g., Kauffman--Lins) absorb this sign into the $6j$
  symbol normalization via Theta-net conventions.  For the BLM model, which
  uses only \emph{integer} $j$, this sign is always $+1$ and the
  distinction is immaterial.
\end{remark}

The fundamental result is that $\TV_r(M)$ is independent of the choice
of triangulation~$\mathcal{T}$; this was the main theorem
of~\cite{TV92}.  The invariance under Pachner moves (bistellar flips)
relies on the Biedenharn--Elliott identity for quantum $6j$ symbols and the
orthogonality relations, both of which hold at root of unity for admissible
colorings.


\subsubsection{Turaev--Walker theorem and the Chern--Simons connection}

The Turaev--Viro invariant is not merely a combinatorial curiosity: it
is the norm-squared of a much deeper invariant.

\begin{theorem}[Turaev--Walker {\cite{TuraevWalker}}]
\label{thm:TW}
  For any closed oriented 3-manifold $M$,
  \begin{equation}\label{eq:TW}
    \TV_r(M) \;=\; |\tau_r(M)|^2\,,
  \end{equation}
  where $\tau_r(M)$ is the Reshetikhin--Turaev invariant of $M$ --- the
  mathematically rigorous version of the $\mathrm{SU}(2)$ Chern--Simons
  partition function at level $k_{\CS} = r - 2$.
\end{theorem}

This theorem establishes the physical interpretation: the Turaev--Viro
state sum computes the partition function of three-dimensional Chern--Simons
gauge theory.  Since the Chern--Simons partition function can also be
interpreted as a discretized path integral for 3D gravity with positive
cosmological constant, we obtain the following semiclassical result.

\begin{proposition}[Semiclassical limit {\cite{MizoguchiTada}}]
\label{prop:semiclassical}
  In the limit $r \to \infty$, the Turaev--Viro state sum recovers a
  discretized path integral for Euclidean three-dimensional gravity with
  positive cosmological constant
  \begin{equation}\label{eq:Lambda-TV}
    \Lambda
    \;=\; \frac{4\pi^2}{r^2} + O(r^{-4})\,.
  \end{equation}
\end{proposition}

The relation~\eqref{eq:Lambda-TV} uses the convention $A^{2r} = 1$
(i.e., the Mizoguchi--Tada parameter $k_{\mathrm{MT}} = r$).  Equivalently,
in terms of the Chern--Simons level $k_{\CS} = r - 2$, this reads
$\Lambda = 4\pi^2/(k_{\CS} + 2)^2 + O((k_{\CS}+2)^{-4})$.  As $r \to
\infty$ (equivalently $\Lambda \to 0$), the TV state sum formally approaches
the Ponzano--Regge state sum of Part~I, though the latter is divergent and
requires regularization --- TV \emph{is} the regularization.


%% ==================================================================
\subsection{Boulatov GFT and the structural analogy}
\label{sec:boulatov}
\afnode{1.4.2}

The Boulatov group field theory (GFT) provides a tantalizing framework in
which Feynman diagrams of a field theory on a group manifold are dual to
triangulations of three-manifolds, and each Feynman amplitude equals the
corresponding state sum amplitude.

\begin{definition}[Boulatov model {\cite{Boulatov}}]
\label{def:boulatov}
  The Boulatov GFT is a field theory of a complex scalar field
  $\phi(g_1, g_2, g_3)$ on $\mathrm{SU}(2)^3$, subject to gauge invariance
  $\phi(g_1 h, g_2 h, g_3 h) = \phi(g_1, g_2, g_3)$ for all $h \in
  \mathrm{SU}(2)$.  The action consists of a free (quadratic) term and a
  cubic interaction:
  \begin{equation}\label{eq:boulatov-action}
    S[\phi]
    \;=\; \int \phi^2
    \;+\; \frac{\lambda}{4!}
    \int \prod_{i=1}^{4}\!\bigl[dg_i^{(1)} dg_i^{(2)} dg_i^{(3)}\bigr]\;
    \phi(g_1^{(1)},g_1^{(2)},g_1^{(3)})
    \cdots
    \phi(g_4^{(1)},g_4^{(2)},g_4^{(3)})\;
    \mathcal{V}\,,
  \end{equation}
  where $\mathcal{V}$ encodes the combinatorial pattern of argument sharing
  that implements a tetrahedral contraction.
\end{definition}

The Feynman diagrams of this model are dual to three-dimensional
triangulations, and each Feynman amplitude equals the Ponzano--Regge
($q = 1$) amplitude of the dual triangulation~\cite{Boulatov,Freidel}.
The GFT framework thus provides a ``third quantization'' of
three-dimensional gravity in which spacetime emerges from the Feynman
expansion.

\subsubsection{What the BLM model is and is not}

It is natural to ask whether the BLM model \emph{is} a GFT, or at least
a sector thereof.  The answer is \textbf{no}, for three precise reasons:

\begin{enumerate}[label=(\roman*),nosep]
  \item \textbf{Fixed spin vs.\ spin sum.}\;
    The BLM model uses a \emph{single fixed} spin $j$ (with $N = 2j+1$
    fermion modes).  The Boulatov GFT sums over \emph{all} admissible
    spins in each Feynman amplitude.
  \item \textbf{0+1d QM vs.\ field theory on a group manifold.}\;
    The BLM model is a quantum mechanical system in $0+1$ dimensions.
    The Boulatov model is a field theory on $\mathrm{SU}(2)^3$ (or its
    quantum group generalization).
  \item \textbf{Vertex structure.}\;
    The BLM cubic vertex involves $3j$ symbols at a single spin~$j$.
    The Boulatov cubic vertex involves group integration
    (or Peter--Weyl expansion over all spins).
\end{enumerate}

\begin{remark}[What IS shared]\label{rem:shared-symbols}
  Despite these differences, the two frameworks share the same
  \emph{algebraic building blocks}: individual non-melonic BLM Feynman
  diagrams (tetrahedron, cube, etc.) involve products of quantum $6j$ and
  higher recoupling symbols that are the \emph{same} objects appearing as
  TV/PR weights.  The structural relationship is:
  \[
    \text{BLM diagrams}
    \;=\; \text{products of recoupling symbols at \emph{fixed} } j\,,
  \]
  \[
    \text{TV state sums}
    \;=\; \text{products of the \emph{same} symbols,
           \emph{summed} over all admissible } l \leq (r-2)/2\,.
  \]
  The precise map from BLM Feynman amplitudes to TV-type invariants remains
  an \textbf{open problem}.
\end{remark}


%% ==================================================================
\subsection{SUSY at root of unity}
\label{sec:susy-root}
\afnode{1.4.3}
\established

We now turn to the most delicate aspect of the root-of-unity regime: the
status of $\mathcal{N}=2$ supersymmetry.  The key results are that (i) SUSY
is \emph{preserved} --- not broken --- at roots of unity, and (ii) the
BLM spin parameter $j$ must satisfy a strict admissibility bound for the
model to be well-defined.

\subsubsection{Nilpotency}

\begin{proposition}[Nilpotency at root of unity]
\label{prop:nilpotency-root}
  At $q = e^{2\pi i/r}$, the supercharge satisfies $Q_q^2 = 0$, provided
  the quantum $3j$ symbols $C^j_q(m_1, m_2, m_3)$ are well-defined
  (i.e., $j$ is admissible).
\end{proposition}

\begin{proof}
  The proof is purely algebraic and identical to the $q > 0$ case: $Q_q^2$
  is a sum of terms $C^j_q(m_1,m_2,m_3)\,C^j_q(m_4,m_5,m_6)$
  contracted with products of six fermionic operators.  The total
  antisymmetry of the $q$-$3j$ symbol under column permutations (an
  algebraic identity valid for \emph{all} $q$ where the symbols are defined,
  not depending on reality of coefficients) combined with Grassmann
  anticommutation forces each term to cancel.  No reality condition on $q$
  is needed.
\end{proof}


\subsubsection{Positive semi-definiteness}

\begin{proposition}[PSD at root of unity]
\label{prop:psd-root}
  At $q = e^{2\pi i/r}$, the Hamiltonian
  $H_q = \{Q_q, Q_q^\dagger\} \geq 0$ is positive semi-definite.
\end{proposition}

\begin{proof}
  This is \emph{tautological}.  For \textbf{any} operator $Q$ on
  \textbf{any} Hilbert space $\Hilbert$, and any state $|v\rangle \in
  \Hilbert$:
  \begin{equation}\label{eq:psd-tautology}
    \langle v | \{Q, Q^\dagger\} | v \rangle
    \;=\; \| Q^\dagger v \|^2 + \| Q v \|^2
    \;\geq\; 0\,.
  \end{equation}
  This holds regardless of whether $Q$ has real or complex matrix elements.
  The argument depends only on the fact that $Q_q^\dagger$ is the
  \emph{Fock space adjoint} of $Q_q$ (defined by the inner product on
  $\Fock$), not on any relationship between $Q_q^\dagger$ and the
  coefficient-conjugated operator $\overline{Q}_q$.
\end{proof}


\subsubsection{Erratum: the $Q^\dagger$ vs.\ $\overline{Q}$ confusion}
\label{sec:erratum}

\begin{erratum}
  The original formulation of AF node~1.4.3 claimed that
  $\{Q_q, Q_q^\dagger\}$ could have \emph{negative} eigenvalues at root of
  unity.  This was mathematically false.

  The error arose from conflating two distinct operations:
  \begin{itemize}[nosep]
    \item $\overline{Q}_q$: the operator obtained by conjugating the
      $q$-$3j$ coefficients while keeping the same fermion operator ordering.
    \item $Q_q^\dagger$: the Fock space adjoint, which reverses the fermion
      ordering and conjugates the coefficients.
  \end{itemize}
  For the cubic supercharge, reversing the order of three anticommuting
  creation/annihilation operators requires $\binom{3}{2} = 3$ transpositions,
  producing a sign $(-1)^3 = -1$.  Hence $Q_q^\dagger = -\overline{Q}_q$,
  and
  \begin{equation}\label{eq:Q-Qbar-relation}
    \{Q_q, Q_q^\dagger\}
    \;=\; -\{Q_q, \overline{Q}_q\}\,.
  \end{equation}
  The original argument incorrectly suggested that $\{Q_q, \overline{Q}_q\}$
  could have positive eigenvalues, making $-\{Q_q, \overline{Q}_q\}$
  negative.  In fact, $\{Q_q, \overline{Q}_q\}$ must have all eigenvalues
  $\leq 0$, precisely because $-\{Q_q, \overline{Q}_q\} = \{Q_q,
  Q_q^\dagger\} \geq 0$ by the tautological norm
  argument~\eqref{eq:psd-tautology}.
\end{erratum}


\subsubsection{Admissibility bounds}
\label{sec:admissibility}

At root of unity, the BLM spin parameter $j$ (an odd integer $\geq 1$)
cannot be arbitrary.  There are two constraints, the stricter of which is
operative.

\begin{definition}[Admissibility conditions for the $q$-BLM model]
\label{def:admissibility}
  Let $q = e^{2\pi i/r}$ with $r \geq 3$.
  \begin{description}[style=nextline,labelwidth=13em,leftmargin=14em]
    \item[Condition A (representation)]
      The spin-$j$ representation of $U_q(\mathfrak{su}(2))$ exists with
      nonzero quantum dimension: $\qint{2j+1}_{\!A} \neq 0$, which requires
      \begin{equation}\label{eq:cond-A}
        r \;\geq\; 2j + 2\,.
      \end{equation}
    \item[Condition B (Racah formula)]
      The quantum $3j$ symbol $(j,j,j;\,m_1,m_2,m_3)$ is well-defined via
      the Racah formula.  The triangle coefficient $\Delta(j,j,j)$ contains
      in its denominator the factorial $\qfact{3j+1} = \qint{1}_{\!A}\,
      \qint{2}_{\!A} \cdots \qint{3j+1}_{\!A}$.  For no factor to vanish,
      we need $\qint{n}_{\!A} \neq 0$ for all $n = 1, \ldots, 3j+1$, which
      requires
      \begin{equation}\label{eq:cond-B}
        r \;\geq\; 3j + 2\,.
      \end{equation}
  \end{description}
\end{definition}

Since $3j + 2 > 2j + 2$ for all $j \geq 1$, Condition~B is strictly
stronger than Condition~A.  The BLM supercharge requires the $(j,j,j)$ $3j$
symbol, so \textbf{Condition~B is the operative constraint}.

\begin{remark}[TV level truncation]\label{rem:TV-truncation}
  Condition~B is \emph{identical} to the Turaev--Viro admissibility condition
  for the triple $(j,j,j)$: the level truncation $a + b + c \leq r - 2$
  applied to $(a,b,c) = (j,j,j)$ gives $3j \leq r - 2$, i.e.,
  $r \geq 3j + 2$.  This is not a coincidence: both constraints arise from
  the requirement that all quantum factorials in the $6j$ (or $3j$) symbol
  formulae are well-defined.
\end{remark}

\begin{example}[Concrete bounds]\label{ex:concrete-bounds}
  For the first few admissible odd integers $j$:
  \begin{center}
  \begin{tabular}{ccc}
    \toprule
    $j$ & Condition A: $r \geq 2j+2$ & Condition B: $r \geq 3j+2$ \\
    \midrule
    $1$ & $r \geq 4$ & $r \geq 5$ \\
    $3$ & $r \geq 8$ & $r \geq 11$ \\
    $5$ & $r \geq 12$ & $r \geq 17$ \\
    $7$ & $r \geq 16$ & $r \geq 23$ \\
    \bottomrule
  \end{tabular}
  \end{center}
  In each case, Condition~B is the binding constraint.
\end{example}


\subsubsection{Reality of $q$-$3j$ symbols for admissible $j$}
\label{sec:reality}

An important consequence of the admissibility bound $r \geq 3j + 2$ is
that the $q$-$3j$ symbols at the BLM coupling are \emph{real-valued}.

\begin{proposition}[Reality of admissible $q$-$3j$ symbols]
\label{prop:reality}
  For $j$ satisfying $r \geq 3j + 2$, every quantum integer $\qint{n}_{\!A}$
  appearing in the Racah formula for the $(j,j,j)$ $3j$ symbol has
  $1 \leq n \leq 3j + 1 < r$, and therefore
  \begin{equation}\label{eq:qint-positive}
    \qint{n}_{\!A}
    \;=\; \frac{\sin(n\pi/r)}{\sin(\pi/r)}
    \;>\; 0
    \qquad\text{for all } 1 \leq n \leq 3j+1\,.
  \end{equation}
  Since the Racah formula for the $3j$ symbol involves only quantum integers,
  quantum factorials, and rational functions thereof --- all built from the
  strictly positive quantities~\eqref{eq:qint-positive} --- the resulting
  $q$-$3j$ symbols $C^j_q(m_1, m_2, m_3)$ are real for all admissible~$j$.
\end{proposition}

This reality has important implications: when the $3j$ symbols are real,
$\overline{Q}_q = Q_q$, and the distinction between $Q_q^\dagger$ and
$\overline{Q}_q$ that caused the erratum (see \cref{sec:erratum}) becomes
$Q_q^\dagger = -Q_q$, matching the structure of the $q = 1$ theory
up to sign conventions.


\subsubsection{Finite truncation and full SUSY}

\begin{theorem}[$\mathcal{N}=2$ SUSY for admissible $j$]
\label{thm:susy-root}
  For any odd integer $j \geq 1$ satisfying $r \geq 3j + 2$, the
  root-of-unity $q$-BLM model is a well-defined $\mathcal{N}=2$
  supersymmetric quantum mechanics:
  \begin{enumerate}[label=(\alph*),nosep]
    \item the supercharge $Q_q$ satisfies $Q_q^2 = 0$;
    \item the Hamiltonian $H_q = \{Q_q, Q_q^\dagger\} \geq 0$;
    \item BPS ground states are
      $\ker H_q = \ker Q_q \cap \ker Q_q^\dagger$;
    \item the Witten index $\Tr\bigl((-1)^F e^{-\beta H_q}\bigr)$ has its
      standard SUSY interpretation.
  \end{enumerate}
\end{theorem}

Unlike the $q > 0$ real case (Parts~I and~II), where $j$ can be any odd
positive integer, the root-of-unity model exists for only finitely many
values of~$j$.

\begin{definition}[Finite truncation]
\label{def:finite-trunc}
  The root-of-unity $q$-BLM model exists for odd integers
  $j = 1, 3, 5, \ldots, j_{\max}(r)$, where
  \begin{equation}\label{eq:jmax}
    j_{\max}(r)
    \;=\; \text{largest odd integer} \leq \frac{r-2}{3}\,.
  \end{equation}
\end{definition}

\begin{example}[Admissible models]\label{ex:trunc}
  \begin{center}
  \begin{tabular}{rrl}
    \toprule
    $r$ & $j_{\max}$ & Admissible models \\
    \midrule
    $5$ & $1$ & $j = 1$ only \\
    $11$ & $3$ & $j = 1,\, 3$ \\
    $17$ & $5$ & $j = 1,\, 3,\, 5$ \\
    $20$ & $5$ & $j = 1,\, 3,\, 5$ \quad ($j = 7$ needs $r \geq 23$) \\
    $23$ & $7$ & $j = 1,\, 3,\, 5,\, 7$ \\
    \bottomrule
  \end{tabular}
  \end{center}
  This is much sparser than the TV coloring set
  $\{0, \half, 1, \ldots, (r-2)/2\}$, both because $j$ must be a positive
  odd integer and because the Racah bound $j \leq (r-2)/3$ is stricter than
  the representation bound $l \leq (r-2)/2$.
\end{example}


%% ==================================================================
\subsection{The $r \to \infty$ limit}
\label{sec:r-to-infty}

As the root-of-unity order $r$ tends to infinity, the deformation parameter
$q = e^{2\pi i/r} \to 1$, and three things happen:
\begin{enumerate}[label=(\roman*),nosep]
  \item The admissibility bound $r \geq 3j+2$ becomes vacuous for any fixed
    $j$, recovering the unrestricted representation theory of
    $\mathrm{SU}(2)$ at $q = 1$.
  \item The Turaev--Viro state sum formally approaches the Ponzano--Regge
    state sum of Part~I (though the latter is divergent and requires
    regularization; the TV sum at finite~$r$ \emph{is} the regularization).
  \item For the BLM model at fixed~$j$, the root-of-unity model smoothly
    recovers the $q = 1$ model: all representation-theoretic constraints
    become vacuous, and the $q$-$3j$ symbols continuously approach their
    classical values.
\end{enumerate}
The cosmological constant $\Lambda = 4\pi^2/r^2 \to 0$, so the $r \to
\infty$ limit simultaneously decompactifies the geometry from positive
curvature ($\TV$) to flat ($\PR$), in agreement with the physical picture
of Part~I.


%% ==================================================================
\subsection{Summary}
\label{sec:root-summary}

\begin{center}
\begin{tabular}{lll}
  \toprule
  \textbf{Result} & \textbf{AF Node} & \textbf{Status} \\
  \midrule
  TV state sum and invariance       & 1.4.1 & \established \\
  Turaev--Walker: $\TV_r = |\tau_r|^2$ & 1.4.1 & \established \\
  Semiclassical: $\Lambda = 4\pi^2/r^2$ & 1.4.1 & \established \\
  Boulatov GFT analogy              & 1.4.2 & Structural (not a duality) \\
  BLM $\neq$ GFT                    & 1.4.2 & \established \\
  $Q_q^2 = 0$ at root of unity      & 1.4.3 & \established \\
  $\{Q_q,Q_q^\dagger\} \geq 0$      & 1.4.3 & \established{} (tautological) \\
  Admissibility: $r \geq 3j+2$      & 1.4.3 & \established \\
  Finite truncation                  & 1.4.3 & \established \\
  Reality of $q$-$3j$ symbols        & 1.4.3 & \established \\
  BLM $\leftrightarrow$ TV precise map & 1.4 & \textbf{Open} \\
  \bottomrule
\end{tabular}
\end{center}

The root-of-unity regime presents a rich interplay between established
topological quantum field theory (the TV/RT framework) and the specific
structure of the BLM model.  The SUSY structure survives intact for
admissible spins, but the precise relationship between individual BLM
Feynman amplitudes and TV invariants of specific triangulations remains the
central open problem of this regime.
