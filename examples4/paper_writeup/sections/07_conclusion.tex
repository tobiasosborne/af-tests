\section{Conclusion}
\label{sec:conclusion}

We have presented a comprehensive account of the $q$-deformed BLM model as a family of $N=2$
supersymmetric quantum mechanical systems parameterized by a single deformation parameter~$q$.
The supercharge is constructed from quantum $3j$ symbols of $U_q(\mathfrak{su}(2))$,
and the model exhibits markedly different behavior across three geometric regimes.

\subsection*{Three Regimes and Their Epistemic Status}

\subsubsection*{Regime I: The Euclidean Regime ($q=1$) — \established}

At $q=1$, the model reduces to the BLM melonic model with Euclidean $N=2$ supersymmetry.
The supercharge anticommutator is proportional to the BLM Hamiltonian,
and melonic dominance yields SYK-type solvability. The asymptotic spectrum is governed
by Ponzano--Regge asymptotics (flat 3D gravity on the tetrahedron), and the model admits
a precise holographic dual in terms of flat Euclidean $\text{AdS}_3$ with boundary correlators.
This regime is now established through multiple independent approaches \cite{BLM,SYK-Fu}.

\subsubsection*{Regime II: The Hyperbolic Regime (fixed real $q \neq 1$) — \conjectural}

For real $q \neq 1$, the quantum dimension $[2]_q = (q^2 - q^{-2})/(q-q^{-1})$ grows
without bound as $|q|$ increases. This exponential growth of quantum recoupling
coefficients is conjectured to break melonic dominance, yielding a different asymptotic
regime where non-melonic diagrams compete. The volume conjecture and hyperbolic geometry
enter through the asymptotics of $6j$ symbols \cite{BellettiYang,Costantino,MurakamiMurakami}.
The precise asymptotic formula and the existence of a holographic dual in this regime
remain open.

\subsubsection*{Regime III: The Root-of-Unity Regime ($q = e^{2\pi i/r}$) — \mixedstatus}

When $q$ is a primitive $r$-th root of unity, the quantum dimension is bounded and
the model exhibits finite-dimensional fusion structure. A crucial correction to
the earlier literature: the SUSY is \emph{always} preserved in this regime
(the claim that $\{Q_q, Q_q^\dagger\} \ge 0$ breaks SUSY was mathematically false ---
this inequality is tautological for all $q$). The supercharge is well-defined
for odd $j$ satisfying $r \ge 3j+2$ (the Racah formula admissibility bound),
and the $q$-$3j$ symbols are real-valued in this range~\cite{TV92,TuraevWalker}.

The root-of-unity regime connects directly to Turaev--Viro topological invariants and
3D gravity with positive cosmological constant, with the Verlinde formula determining
the physical spectrum. The BLM-to-TV map is the central open problem in this regime.

\subsection*{Symmetries and Constraints}

A key discovery of this work is the role of the $q \leftrightarrow q^{-1}$ duality.
This symmetry, which relates $(q, j, k)$ to $(q^{-1}, j, k)$ in all asymptotic formulas,
constrains all three regimes and provides a consistency check across different
parametrizations of the quantum group. This duality is manifest in the quantum
dimension formula and the $3j$ symbol growth rates.

\subsection*{Corrections and Mathematical Integrity}

The adversarial prover--verifier process, which forms the foundation of this manuscript,
revealed three critical errors in earlier treatments:
\begin{enumerate}
    \item \textbf{SUSY Breaking Claim:} The claim that $\{Q_q, Q_q^\dagger\} \not\ge 0$
    at root of unity (implying SUSY breaking) was mathematically false. The
    anticommutator is always non-negative for all $q$.

    \item \textbf{Admissibility Bound:} The condition for supercharge reality and
    positivity is $r \ge 3j+2$, not $r \ge 2j+3$ as claimed earlier. This strengthens
    the upper bound on admissible $j$ by a factor of $\sim 3/2$.

    \item \textbf{Reality of $6j$ Symbols:} The reality of $q^{-3j}$ weighted
    $6j$ symbols depends on whether $3j \in \Z$, not just on $j$ itself.
    This affects the definition of the supercharge in Regime III.
\end{enumerate}

These corrections demonstrate the value of formal verification: the process of
checking each claim against the proof tree forced recognition of implicit assumptions
and algebraic errors that would have persisted in a traditional writeup.

\subsection*{The Central Open Problem}

The primary unresolved question is the \emph{BLM-to-TV correspondence}:
a rigorous derivation of the map from the quantum mechanical supercharge to
the Turaev--Viro partition function in the root-of-unity regime.
A heuristic sketch exists based on spin foam resummation, but a complete proof
requires either:
\begin{itemize}
    \item Explicit computation of the full two-point function and its tensor product decomposition,
    \item A categorical equivalence between the representation categories, or
    \item A path integral argument in the spirit of \cite{MizoguchiTada}.
\end{itemize}

\subsection*{Broader Implications}

The $q$-deformed BLM model illustrates a more general principle: that quantum group
deformations of classical gravitational models can be studied combinatorially via
quantum recoupling theory. The three regimes (melonic/hyperbolic/topological)
may represent different facets of a unified theory of quantum gravity at scale.
The Regime~II conjecture --- that exponential growth breaks melonic dominance ---
remains one of the sharpest tests of whether the model describes physical gravity
at all quantum scales, not merely at $q=1$ or at roots of unity.

\subsection*{Acknowledgment of Process}

This paper is the first in the AF-Tests series to undergo formal adversarial
verification throughout. The process was demanding: every calculation was checked,
every claim attributed to a proof node, and every gap documented. This did not
make the paper "cleaner" --- it made it honest. Readers can now trace every
result to its source, verify every claim, and identify exactly where conjecture
begins. We hope this transparency becomes standard for models at the intersection
of quantum information, quantum groups, and quantum gravity.

