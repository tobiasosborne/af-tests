% Main Document: Archimedean Closure Formalization
% Auto-generated from Lean 4 source code
%
% To compile: pdflatex main.tex (run twice for references)

% Preamble for Archimedean Closure Formalization
% Auto-generated template for Lean 4 → LaTeX conversion

%% ============================================================
%% Document Class and Basic Packages
%% ============================================================
\documentclass[11pt,a4paper]{book}

\usepackage[utf8]{inputenc}
\usepackage[T1]{fontenc}
\usepackage{lmodern}
\usepackage[margin=1in]{geometry}
\usepackage{listings}

% Lean language definition for listings
\lstdefinelanguage{lean}{
  morekeywords={def,theorem,lemma,structure,where,instance,abbrev,noncomputable,fun,by,forall,exists},
  sensitive=true,
  morecomment=[l]{--},
  morecomment=[s]{\{-}{-\}},
  morestring=[b]",
}
\lstset{
  basicstyle=\ttfamily\small,
  breaklines=true,
  frame=single,
  columns=flexible,
}

%% ============================================================
%% Mathematics
%% ============================================================
\usepackage{amsmath,amssymb,amsthm}
\usepackage{mathtools}
\usepackage{bm}
\usepackage{mathrsfs}

%% ============================================================
%% Theorem Environments
%% ============================================================
\theoremstyle{plain}
\newtheorem{theorem}{Theorem}[chapter]
\newtheorem{lemma}[theorem]{Lemma}
\newtheorem{proposition}[theorem]{Proposition}
\newtheorem{corollary}[theorem]{Corollary}

\theoremstyle{definition}
\newtheorem{definition}[theorem]{Definition}
\newtheorem{example}[theorem]{Example}

\theoremstyle{remark}
\newtheorem{remark}[theorem]{Remark}
\newtheorem{notation}[theorem]{Notation}

% Unnumbered theorem environments
\newtheorem*{theorem*}{Theorem}

%% ============================================================
%% Custom Commands - Algebra
%% ============================================================
\newcommand{\Azero}{\mathcal{A}_0}           % Free *-algebra
\newcommand{\M}{M}                            % Quadratic module
\newcommand{\Mbar}{\overline{M}}              % Closure of M
\newcommand{\SM}{S_M}                         % M-positive states
\newcommand{\gen}[1]{g_{#1}}                  % Generator g_j
\newcommand{\staralg}[1]{#1^*}                % Star/adjoint

%% ============================================================
%% Custom Commands - Analysis
%% ============================================================
\newcommand{\inner}[2]{\langle #1, #2 \rangle}      % Inner product
\newcommand{\norm}[1]{\|#1\|}                        % Norm
\newcommand{\seminorm}[1]{\|#1\|_M}                  % State seminorm
\newcommand{\abs}[1]{|#1|}                           % Absolute value

%% ============================================================
%% Custom Commands - Operators
%% ============================================================
\newcommand{\Hilbert}{\mathcal{H}}                   % Hilbert space
\newcommand{\HilbertPhi}{\mathcal{H}_\varphi}        % GNS Hilbert space
\newcommand{\BH}{\mathcal{B}(\Hilbert)}              % Bounded operators
\newcommand{\Rep}{\pi}                                % Representation

%% ============================================================
%% Custom Commands - Sets and Spaces
%% ============================================================
\newcommand{\N}{\mathbb{N}}
\newcommand{\Z}{\mathbb{Z}}
\newcommand{\R}{\mathbb{R}}
\newcommand{\C}{\mathbb{C}}

%% ============================================================
%% Formatting
%% ============================================================
\usepackage{enumitem}
\usepackage{booktabs}
\usepackage{hyperref}
\hypersetup{
    colorlinks=true,
    linkcolor=blue,
    citecolor=blue,
    urlcolor=blue
}

%% ============================================================
%% Code Listings (for Lean snippets if needed)
%% ============================================================
\usepackage{listings}
\lstdefinelanguage{Lean}{
  keywords={def,theorem,lemma,example,structure,class,instance,where,
            by,have,let,show,calc,sorry,admit,namespace,end,open,
            variable,import,inductive,if,then,else,match,with},
  morecomment=[l]{--},
  morecomment=[n]{/-}{-/},
  morestring=[b]",
  sensitive=true,
}
\lstset{
  language=Lean,
  basicstyle=\ttfamily\small,
  keywordstyle=\bfseries,
  commentstyle=\itshape,
  breaklines=true,
  frame=single,
  xleftmargin=2em,
}

%% ============================================================
%% Cross-referencing
%% ============================================================
\usepackage[capitalise,nameinlink]{cleveref}
\crefname{theorem}{Theorem}{Theorems}
\crefname{lemma}{Lemma}{Lemmas}
\crefname{definition}{Definition}{Definitions}
\crefname{proposition}{Proposition}{Propositions}
\crefname{corollary}{Corollary}{Corollaries}
\crefname{remark}{Remark}{Remarks}
\crefname{example}{Example}{Examples}
\crefname{chapter}{Chapter}{Chapters}
\crefname{section}{Section}{Sections}

%% ============================================================
%% Title Information
%% ============================================================
\title{\textbf{Archimedean Closure of Quadratic Modules}\\[0.5em]
       \large A Formalization in Lean 4}
\author{Generated from Lean 4 Source Code}
\date{\today}


\begin{document}

\maketitle

\frontmatter

%% ============================================================
%% Abstract
%% ============================================================
\chapter*{Abstract}
\addcontentsline{toc}{chapter}{Abstract}

This document presents a complete formalization in Lean 4 of the
characterization of positivity in constrained $*$-algebra representations.
The main theorem establishes:

\begin{quote}
\textbf{Main Theorem.}
Let $\Azero = \C\langle \gen{1}, \ldots, \gen{n} \rangle$ be the free
$*$-algebra on $n$ self-adjoint generators, and let $\M$ be the quadratic
module generated by the positivity constraints. Then for any self-adjoint
$A \in \Azero$:
\[
  A \in \Mbar \iff \Rep(A) \geq 0
  \text{ for all constrained $*$-representations } \Rep.
\]
\end{quote}

The formalization comprises 44 Lean files totaling approximately 4,900 lines
of code, with zero unproven goals (\texttt{sorry}). All proofs use only the
standard axioms of Lean's type theory: propositional extensionality,
the axiom of choice, and quotient soundness.

%% ============================================================
%% Table of Contents
%% ============================================================
\tableofcontents

\mainmatter

%% ============================================================
%% Part I: Algebraic Foundations
%% ============================================================
\part{Algebraic Foundations}

% Chapter: Introduction
% Generated from: Overview of the formalization
% Status: PLACEHOLDER - to be filled by LaTeX-P1.1

\chapter{Introduction}
\label{ch:introduction}

% TODO: Fill in from LaTeX-P1.1 task
\textbf{[PLACEHOLDER: To be generated from Lean source]}

\section{Overview}
\label{sec:intro-overview}

\section{Main Result}
\label{sec:intro-main-result}

\section{Proof Structure}
\label{sec:intro-proof-structure}

\section{Notation}
\label{sec:intro-notation}

% Auto-generated from Lean 4 source
% Source files: FreeStarAlgebra.lean, QuadraticModule.lean, Archimedean.lean
% Do not edit manually

\chapter{Algebraic Setup}
\label{ch:algebra}

This chapter establishes the algebraic foundations for the characterization of positivity
in constrained $C^*$-algebra representations. We define the free $*$-algebra on self-adjoint
generators, the quadratic module capturing the constraint structure, and the Archimedean
property that ensures bounded behavior.

%% ============================================================
\section{The Free $*$-Algebra}
\label{sec:free-star-algebra}

We begin by constructing the free $*$-algebra on $n$ self-adjoint generators over the
real numbers. The choice of $\mathbb{R}$ as the base field is crucial: it ensures that
for any scalar $c$, we have $c^* \cdot c = c^2 \geq 0$, which is essential for obtaining
$M$-positive states via scalar extraction.

\begin{definition}[Free $*$-Algebra]\label{def:FreeStarAlgebra}
The \textbf{free $*$-algebra} $\mathcal{A}_0$ on $n$ self-adjoint generators is defined as
\[
  \mathcal{A}_0 = \mathbb{R}\langle g_1, \ldots, g_n \rangle
\]
the free algebra over $\mathbb{R}$ on generators indexed by $\mathrm{Fin}(n)$, equipped
with the canonical $*$-structure from mathlib's \texttt{Mathlib.Algebra.Star.Free}.
\end{definition}

\begin{definition}[Generator]\label{def:generator}
For $j \in \mathrm{Fin}(n)$, the \textbf{$j$-th generator} $g_j \in \mathcal{A}_0$ is the
canonical embedding of the index $j$ into the free algebra:
\[
  g_j = \iota(j)
\]
where $\iota : \mathrm{Fin}(n) \to \mathcal{A}_0$ is the universal map.
\end{definition}

\begin{theorem}[Generators are Self-Adjoint]\label{thm:isSelfAdjoint_generator}
For each $j \in \mathrm{Fin}(n)$, the generator $g_j$ is self-adjoint:
\[
  g_j^* = g_j.
\]
\end{theorem}

\begin{proof}
This follows directly from the star structure on free algebras defined in
\texttt{Mathlib.Algebra.Star.Free}, which specifies that generators are fixed by
the involution.
\end{proof}

\begin{lemma}[Unit is Self-Adjoint]\label{lem:one_isSelfAdjoint}
The unit element $1 \in \mathcal{A}_0$ is self-adjoint: $1^* = 1$.
\end{lemma}

\begin{proof}
Standard property of $*$-algebras.
\end{proof}

%% ============================================================
\section{Quadratic Modules}
\label{sec:quadratic-module}

The quadratic module $M$ captures the positivity constraints imposed by the generators.
It consists of sums of squares and generator-weighted squares, closed under addition
and nonnegative real scaling.

\begin{definition}[Square Set]\label{def:squareSet}
The \textbf{square set} is defined as
\[
  S = \{ a^* a : a \in \mathcal{A}_0 \}.
\]
\end{definition}

\begin{definition}[Generator-Weighted Set]\label{def:generatorWeightedSet}
The \textbf{generator-weighted set} is defined as
\[
  W = \{ b^* g_j b : j \in \mathrm{Fin}(n),\ b \in \mathcal{A}_0 \}.
\]
\end{definition}

\begin{definition}[Quadratic Module Generators]\label{def:QuadraticModuleGenerators}
The \textbf{generating set} for the quadratic module is
\[
  G = S \cup W = \{ a^* a : a \in \mathcal{A}_0 \} \cup \{ b^* g_j b : j \in \mathrm{Fin}(n),\ b \in \mathcal{A}_0 \}.
\]
\end{definition}

\begin{definition}[Quadratic Module]\label{def:QuadraticModule}
The \textbf{quadratic module} $M \subseteq \mathcal{A}_0$ is the smallest set containing $G$
and closed under:
\begin{enumerate}
  \item Addition: if $m_1, m_2 \in M$, then $m_1 + m_2 \in M$
  \item Nonnegative scaling: if $c \geq 0$ and $m \in M$, then $c \cdot m \in M$
\end{enumerate}
Equivalently,
\[
  M = \left\{ \sum_{i=1}^k c_i a_i^* a_i + \sum_{j,\ell} d_{j\ell} b_{j\ell}^* g_j b_{j\ell}
      : c_i, d_{j\ell} \geq 0,\ a_i, b_{j\ell} \in \mathcal{A}_0 \right\}.
\]
\end{definition}

\begin{lemma}[Squares in $M$]\label{lem:star_mul_self_mem}
For any $a \in \mathcal{A}_0$, we have $a^* a \in M$.
\end{lemma}

\begin{proof}
By definition, $a^* a \in S \subseteq G \subseteq M$.
\end{proof}

\begin{lemma}[Generator-Weighted Elements in $M$]\label{lem:star_generator_mul_mem}
For any $j \in \mathrm{Fin}(n)$ and $b \in \mathcal{A}_0$, we have $b^* g_j b \in M$.
\end{lemma}

\begin{proof}
By definition, $b^* g_j b \in W \subseteq G \subseteq M$.
\end{proof}

\begin{theorem}[Conjugation Closure]\label{thm:star_mul_mem_star_mul}
If $m \in M$ and $b \in \mathcal{A}_0$, then $b^* m b \in M$.
\end{theorem}

\begin{proof}[Proof sketch]
We proceed by structural induction on the definition of $M$:
\begin{itemize}
  \item \textbf{Generator case ($m = a^* a$):}
    $b^* (a^* a) b = (ab)^* (ab) \in M$ by Lemma~\ref{lem:star_mul_self_mem}.
  \item \textbf{Generator case ($m = c^* g_j c$):}
    $b^* (c^* g_j c) b = (cb)^* g_j (cb) \in M$ by Lemma~\ref{lem:star_generator_mul_mem}.
  \item \textbf{Addition case ($m = m_1 + m_2$):}
    $b^* (m_1 + m_2) b = b^* m_1 b + b^* m_2 b \in M$ by the induction hypothesis and closure.
  \item \textbf{Scaling case ($m = r \cdot m'$):}
    $b^* (r \cdot m') b = r \cdot (b^* m' b) \in M$ by the induction hypothesis and closure.
\end{itemize}
\end{proof}

%% ============================================================
\section{The Archimedean Property}
\label{sec:archimedean}

The Archimedean property ensures that every element of the algebra is ``bounded''
relative to the quadratic module. This is essential for proving that $M$-positive
states yield finite values.

\begin{definition}[Archimedean Property]\label{def:IsArchimedean}
The quadratic module $M$ is \textbf{Archimedean} if for every $a \in \mathcal{A}_0$,
there exists $N \in \mathbb{N}$ such that
\[
  N \cdot 1 - a^* a \in M.
\]
\end{definition}

This condition says that $a^* a$ is ``dominated'' by $N \cdot 1$ in the sense of the
quadratic module: the difference belongs to $M$ and hence will be evaluated nonnegatively
by any $M$-positive state.

\begin{definition}[Archimedean Bound]\label{def:archimedeanBound}
For an Archimedean quadratic module and $a \in \mathcal{A}_0$, the \textbf{Archimedean bound}
$N(a)$ is (a choice of) a natural number such that
\[
  N(a) \cdot 1 - a^* a \in M.
\]
\end{definition}

\begin{theorem}[Archimedean Bound Specification]\label{thm:archimedeanBound_spec}
If $M$ is Archimedean, then for any $a \in \mathcal{A}_0$, the Archimedean bound $N(a)$
satisfies
\[
  N(a) \cdot 1 - a^* a \in M.
\]
\end{theorem}

\begin{proof}
By definition of the Archimedean property and the axiom of choice.
\end{proof}

\begin{remark}
The Archimedean property has the following important consequence: for any $M$-positive
state $\varphi$ and any $a \in \mathcal{A}_0$,
\[
  \varphi(a^* a) \leq N(a) \cdot \varphi(1) = N(a).
\]
Combined with the Cauchy--Schwarz inequality (see Chapter~\ref{ch:boundedness}), this
implies that $|\varphi(a)|^2 \leq N(a)$, so $M$-positive states are uniformly bounded
on each element.
\end{remark}

%% ============================================================
\section{Summary}
\label{sec:algebra-summary}

We have established:
\begin{itemize}
  \item The free $*$-algebra $\mathcal{A}_0 = \mathbb{R}\langle g_1, \ldots, g_n \rangle$
        with self-adjoint generators (Definition~\ref{def:FreeStarAlgebra})
  \item The quadratic module $M$ generated by squares and generator-weighted squares
        (Definition~\ref{def:QuadraticModule})
  \item The conjugation closure property $m \in M \Rightarrow b^* m b \in M$
        (Theorem~\ref{thm:star_mul_mem_star_mul})
  \item The Archimedean property ensuring bounded behavior
        (Definition~\ref{def:IsArchimedean})
\end{itemize}

These structures form the algebraic foundation for the state space and seminorm
constructions in subsequent chapters.

% Chapter: M-Positive States
% Generated from: MPositiveState.lean, MPositiveStateProps.lean, NonEmptiness.lean
% Status: PLACEHOLDER - to be filled by LaTeX-P1.3

\chapter{$M$-Positive States}
\label{ch:states}

% TODO: Fill in from LaTeX-P1.3 task
\textbf{[PLACEHOLDER: To be generated from Lean source]}

\section{Definition of $M$-Positive States}
\label{sec:mpositive-state}

\section{Properties of States}
\label{sec:state-properties}

\section{Non-Emptiness of $\SM$}
\label{sec:nonemptiness}

% Chapter: Boundedness Results
% Generated from: ArchimedeanBound.lean, CauchySchwarzM.lean, GeneratingCone.lean
% Status: PLACEHOLDER - to be filled by LaTeX-P1.4

\chapter{Boundedness Results}
\label{ch:boundedness}

% TODO: Fill in from LaTeX-P1.4 task
\textbf{[PLACEHOLDER: To be generated from Lean source]}

\section{The Archimedean Bound}
\label{sec:archimedean-bound}

\section{Cauchy--Schwarz Inequality}
\label{sec:cauchy-schwarz}

\section{The Generating Cone}
\label{sec:generating-cone}


%% ============================================================
%% Part II: Topological Structure
%% ============================================================
\part{Topological Structure}

% Auto-generated from Lean 4 source
% Source files: StateTopology.lean, SeminormTopology.lean, Closedness.lean, Compactness.lean, Continuity.lean
% Do not edit manually

\chapter{Topological Structure}
\label{ch:topology}

This chapter develops the topological framework for analyzing $M$-positive states. We equip
the state space $S_M$ with the weak-$*$ topology (pointwise convergence), prove that
the defining conditions are closed, and establish compactness of $S_M$ via Tychonoff's
theorem. We also introduce the seminorm topology on $\mathcal{A}_0$ and show that
$M$-positive states are continuous with respect to it.

%% ============================================================
\section{Topology on State Space}
\label{sec:state-topology}

The topology on $S_M$ is the subspace topology inherited from the weak-$*$ topology
on the dual space. Concretely, a net $\varphi_i \to \varphi$ if and only if
$\varphi_i(a) \to \varphi(a)$ for all $a \in \mathcal{A}_0$.

\begin{definition}[Pointwise Convergence Topology]\label{def:instTopologicalSpace}
The \textbf{pointwise convergence topology} on $S_M$ is the coarsest topology making
all evaluation maps continuous. Equivalently, it is the topology induced by the embedding
\[
  S_M \hookrightarrow \prod_{a \in \mathcal{A}_0} \mathbb{R}, \qquad
  \varphi \mapsto (a \mapsto \varphi(a)).
\]
\end{definition}

\begin{theorem}[Evaluation Continuity]\label{thm:eval_continuous}
For any fixed $a \in \mathcal{A}_0$, the evaluation map
$\mathrm{ev}_a : S_M \to \mathbb{R}$ defined by $\mathrm{ev}_a(\varphi) = \varphi(a)$
is continuous.
\end{theorem}

\begin{proof}
By definition, the topology on $S_M$ is induced from the product topology.
The evaluation map factors as $S_M \to \prod_{a} \mathbb{R} \xrightarrow{\pi_a} \mathbb{R}$,
where the projection $\pi_a$ is continuous in the product topology.
\end{proof}

%% ============================================================
\section{Closedness of State Conditions}
\label{sec:closedness}

Each condition defining $M$-positive states is closed in the product topology.
This is essential for proving that $S_M$ is compact.

\begin{lemma}[Continuous Evaluation]\label{lem:continuous_eval}
For any $a \in \mathcal{A}_0$, the evaluation functional
$f \mapsto f(a)$ is continuous on $(\mathcal{A}_0 \to \mathbb{R})$ with the product topology.
\end{lemma}

\begin{theorem}[Additivity is Closed]\label{thm:isClosed_additivity}
The set $\{f : \mathcal{A}_0 \to \mathbb{R} \mid \forall a, b,\, f(a+b) = f(a) + f(b)\}$
is closed.
\end{theorem}

\begin{proof}
For fixed $a, b$, the set $\{f \mid f(a+b) = f(a) + f(b)\}$ equals
$(f \mapsto f(a+b) - f(a) - f(b))^{-1}(\{0\})$. Since evaluation is continuous,
this is a preimage of the closed set $\{0\}$ under a continuous map.
The full set is an intersection over all pairs $(a,b)$, hence closed.
\end{proof}

\begin{theorem}[Homogeneity is Closed]\label{thm:isClosed_homogeneity}
The set $\{f : \mathcal{A}_0 \to \mathbb{R} \mid \forall c \in \mathbb{R}, a,\, f(ca) = c \cdot f(a)\}$
is closed.
\end{theorem}

\begin{proof}
Similar to additivity: express as intersection of preimages of $\{0\}$.
\end{proof}

\begin{theorem}[Star Symmetry is Closed]\label{thm:isClosed_star_symmetry}
The set $\{f : \mathcal{A}_0 \to \mathbb{R} \mid \forall a,\, f(a^*) = f(a)\}$ is closed.
\end{theorem}

\begin{proof}
The condition $f(a^*) = f(a)$ is equivalent to $(f \mapsto f(a^*) - f(a))^{-1}(\{0\})$.
Take intersection over all $a$.
\end{proof}

\begin{theorem}[$M$-Positivity is Closed]\label{thm:isClosed_m_nonneg}
The set $\{f : \mathcal{A}_0 \to \mathbb{R} \mid \forall m \in M,\, f(m) \ge 0\}$ is closed.
\end{theorem}

\begin{proof}
For fixed $m$, the set $\{f \mid f(m) \ge 0\} = \mathrm{ev}_m^{-1}([0, \infty))$
is the preimage of the closed set $[0, \infty)$ under continuous evaluation.
The full set is an intersection over $m \in M$.
\end{proof}

\begin{theorem}[Normalization is Closed]\label{thm:isClosed_normalized}
The set $\{f : \mathcal{A}_0 \to \mathbb{R} \mid f(1) = 1\}$ is closed.
\end{theorem}

\begin{proof}
This equals $\mathrm{ev}_1^{-1}(\{1\})$, a preimage of a singleton.
\end{proof}

%% ============================================================
\section{Seminorm Topology}
\label{sec:seminorm-topology}

The state seminorm $\|\cdot\|_M$ induces a locally convex topology on $\mathcal{A}_0$.

\begin{definition}[Seminorm Family]\label{def:stateSeminormFamily}
The \textbf{state seminorm family} is the family indexed by $\mathrm{Unit}$
consisting of the single seminorm $\|\cdot\|_M$.
\end{definition}

\begin{definition}[Seminorm Topology]\label{def:seminormTopology}
The \textbf{seminorm topology} on $\mathcal{A}_0$ is the topology induced by the
state seminorm family, i.e., the coarsest topology making $\|\cdot\|_M$ continuous.
\end{definition}

\begin{theorem}[Locally Convex]\label{thm:locallyConvexSpace_seminormTopology}
Equipped with the seminorm topology, $\mathcal{A}_0$ is a locally convex topological
vector space.
\end{theorem}

\begin{proof}
This follows from the general theory: any topology induced by a family of seminorms
is locally convex (Theorem~\texttt{WithSeminorms.toLocallyConvexSpace}).
\end{proof}

\begin{theorem}[Closure Equivalence]\label{thm:quadraticModuleClosure_eq_closure}
The $\varepsilon$-$\delta$ closure $\bar{M}$ (Definition~\ref{def:quadraticModuleClosure})
equals the topological closure of $M$ in the seminorm topology:
\[
  \bar{M} = \overline{M}^{\|\cdot\|_M}.
\]
\end{theorem}

\begin{proof}[Proof sketch]
($\subseteq$) Given $a \in \bar{M}$ and a seminorm ball $B_\varepsilon(a)$,
by definition of $\bar{M}$ there exists $m \in M$ with $\|a - m\|_M < \varepsilon$.
Thus every neighborhood of $a$ meets $M$.

($\supseteq$) Given $a$ in the topological closure and $\varepsilon > 0$,
the ball $B_\varepsilon(a)$ is a neighborhood of $a$, so it contains some $m \in M$,
giving $\|a - m\|_M < \varepsilon$.
\end{proof}

\begin{corollary}[Closure is Closed]\label{cor:isClosed_quadraticModuleClosure}
The set $\bar{M}$ is closed in the seminorm topology.
\end{corollary}

\begin{proof}
By Theorem~\ref{thm:quadraticModuleClosure_eq_closure}, $\bar{M}$ is a topological
closure, which is always closed.
\end{proof}

%% ============================================================
\section{Compactness of $S_M$}
\label{sec:compactness}

The main result of this chapter is that $S_M$ is compact.

\begin{definition}[Embedding into Product]\label{def:toProductFun}
Define the embedding $\iota : S_M \hookrightarrow (\mathcal{A}_0 \to \mathbb{R})$ by
$\iota(\varphi)(a) = \varphi(a)$.
\end{definition}

\begin{lemma}[Embedding is Injective]\label{lem:toProductFun_injective}
The map $\iota$ is injective.
\end{lemma}

\begin{definition}[Archimedean Bound Function]\label{def:bound}
For $a \in \mathcal{A}_0$, define $B(a) = \sqrt{N_a}$ where $N_a$ is the Archimedean
constant satisfying $N_a \cdot 1 - a^* a \in M$.
\end{definition}

\begin{theorem}[States are Bounded]\label{thm:apply_mem_closedBall}
For any $\varphi \in S_M$ and $a \in \mathcal{A}_0$,
\[
  |\varphi(a)| \le B(a) = \sqrt{N_a}.
\]
Thus $\varphi(a) \in \overline{B}(0, B(a))$.
\end{theorem}

\begin{proof}
This follows from Theorem~\ref{thm:apply_abs_le} (Archimedean bound on states).
\end{proof}

\begin{theorem}[Image is Bounded]\label{thm:stateSet_subset_product}
The image $\iota(S_M)$ is contained in the product
\[
  \prod_{a \in \mathcal{A}_0} \overline{B}(0, B(a)).
\]
\end{theorem}

\begin{theorem}[Product is Compact]\label{thm:product_compact}
The product $\prod_{a \in \mathcal{A}_0} \overline{B}(0, B(a))$ is compact.
\end{theorem}

\begin{proof}
Each closed ball $\overline{B}(0, B(a)) \subseteq \mathbb{R}$ is compact
(closed and bounded in $\mathbb{R}$). By Tychonoff's theorem, arbitrary products
of compact spaces are compact.
\end{proof}

\begin{definition}[State Conditions]\label{def:stateConditions}
Let $\mathcal{C} \subseteq (\mathcal{A}_0 \to \mathbb{R})$ be the set of functions satisfying:
\begin{enumerate}
  \item (Additivity) $f(a+b) = f(a) + f(b)$ for all $a, b$
  \item (Homogeneity) $f(ca) = c \cdot f(a)$ for all $c \in \mathbb{R}$, $a$
  \item (Star symmetry) $f(a^*) = f(a)$ for all $a$
  \item ($M$-positivity) $f(m) \ge 0$ for all $m \in M$
  \item (Normalization) $f(1) = 1$
\end{enumerate}
\end{definition}

\begin{theorem}[State Conditions are Closed]\label{thm:stateConditions_isClosed}
The set $\mathcal{C}$ is closed in the product topology.
\end{theorem}

\begin{proof}
By Theorems~\ref{thm:isClosed_additivity}--\ref{thm:isClosed_normalized},
each condition defines a closed set. The intersection is closed.
\end{proof}

\begin{theorem}[Range Characterization]\label{thm:range_eq_stateConditions}
The image $\iota(S_M)$ equals $\mathcal{C}$.
\end{theorem}

\begin{proof}
($\subseteq$) Every $M$-positive state satisfies the five conditions by definition.
($\supseteq$) Any function satisfying the conditions can be lifted to an $M$-positive
state via the constructor \texttt{ofFunction}.
\end{proof}

\begin{corollary}[State Set is Closed]\label{cor:stateSet_isClosed}
The set $\iota(S_M)$ is closed in the product topology.
\end{corollary}

\begin{theorem}[Compactness of $S_M$]\label{thm:stateSet_isCompact}
The set $\iota(S_M) \cong S_M$ is compact.
\end{theorem}

\begin{proof}
By Theorem~\ref{thm:stateSet_subset_product}, $\iota(S_M)$ is contained in the
compact set $\prod_a \overline{B}(0, B(a))$ (Theorem~\ref{thm:product_compact}).
By Corollary~\ref{cor:stateSet_isClosed}, $\iota(S_M)$ is closed.
A closed subset of a compact space is compact.
\end{proof}

%% ============================================================
\section{Continuity of States}
\label{sec:continuity}

$M$-positive states are continuous (in fact, Lipschitz) with respect to the
state seminorm.

\begin{theorem}[Lipschitz Bound]\label{thm:lipschitz_dist}
For any $\varphi \in S_M$ and $a, b \in \mathcal{A}_0$,
\[
  |\varphi(a) - \varphi(b)| \le \|a - b\|_M.
\]
In particular, $\varphi$ is $1$-Lipschitz.
\end{theorem}

\begin{proof}
By linearity, $\varphi(a) - \varphi(b) = \varphi(a - b)$.
By Theorem~\ref{thm:apply_abs_le_seminorm}, $|\varphi(a - b)| \le \|a - b\|_M$.
\end{proof}

\begin{corollary}[Continuity]\label{cor:continuous}
Every $M$-positive state $\varphi : \mathcal{A}_0 \to \mathbb{R}$ is continuous
with respect to the seminorm topology on $\mathcal{A}_0$.
\end{corollary}

\begin{proof}
Lipschitz maps are uniformly continuous, hence continuous.
\end{proof}

% Auto-generated from Lean 4 source
% Source files: StateSeminorm.lean, SeminormProps.lean, Closure.lean
% Do not edit manually

\chapter{The State Seminorm}
\label{ch:seminorm}

This chapter develops the state seminorm $\|{\cdot}\|_M$ on the free $*$-algebra $\mathcal{A}_0$.
The Archimedean property ensures that for any element $a \in \mathcal{A}_0$, the supremum
$\sup\{|\varphi(a)| : \varphi \in S_M\}$ is finite, yielding a well-defined seminorm. We then
define the closure $\bar{M}$ of the quadratic module in this seminorm topology.

The main results are:
\begin{itemize}
  \item \texttt{stateSeminorm} (Definition~\ref{def:stateSeminorm}): the state seminorm $\|a\|_M$
  \item \texttt{stateSeminormSeminorm} (Definition~\ref{def:stateSeminormSeminorm}): the seminorm instance
  \item \texttt{quadraticModuleClosure} (Definition~\ref{def:quadraticModuleClosure}): the closure $\bar{M}$
\end{itemize}

%% ============================================================
\section{Definition of the State Seminorm}
\label{sec:seminorm-def}

The Archimedean property ensures that for any $a \in \mathcal{A}_0$, there exists $N_a \in \mathbb{R}$
such that $N_a \cdot 1 - a^*a \in M$. This bound is used to show that the set of state values
$\{|\varphi(a)| : \varphi \in S_M\}$ is bounded above.

\begin{lemma}[Bounded above]\label{lem:bddAbove_abs_range}
For any $a \in \mathcal{A}_0$, the set $\{|\varphi(a)| : \varphi \in S_M\}$ is bounded above by $\sqrt{N_a}$.
\end{lemma}

\begin{proof}
By the Archimedean bound (Lemma~\ref{lem:ArchimedeanBound}), for each $\varphi \in S_M$ we have
$|\varphi(a)| \le \sqrt{N_a}$. Hence $\sqrt{N_a}$ is an upper bound.
\end{proof}

\begin{definition}[State Seminorm]\label{def:stateSeminorm}
The \textbf{state seminorm} $\|{\cdot}\|_M : \mathcal{A}_0 \to \mathbb{R}$ is defined by
\[
  \|a\|_M = \sup\{|\varphi(a)| : \varphi \in S_M\}.
\]
\end{definition}

The nonemptiness of $S_M$ (guaranteed by the scalar state $\varphi_0$) ensures this supremum is
well-defined, and Lemma~\ref{lem:bddAbove_abs_range} ensures it is finite.

\begin{theorem}[Seminorm Upper Bound]\label{thm:stateSeminorm_le}
For any $a \in \mathcal{A}_0$,
\[
  \|a\|_M \le \sqrt{N_a}.
\]
\end{theorem}

\begin{proof}
Apply the supremum bound: since $|\varphi(a)| \le \sqrt{N_a}$ for all $\varphi \in S_M$,
the supremum is at most $\sqrt{N_a}$.
\end{proof}

\begin{lemma}[Nonnegativity]\label{lem:stateSeminorm_nonneg}
For any $a \in \mathcal{A}_0$, $\|a\|_M \ge 0$.
\end{lemma}

\begin{proof}
The scalar state $\varphi_0 \in S_M$ satisfies $|\varphi_0(a)| \ge 0$.
Since $\|a\|_M$ is the supremum over a set containing $|\varphi_0(a)|$, we have $\|a\|_M \ge 0$.
\end{proof}

\begin{lemma}[State Bound]\label{lem:apply_abs_le_seminorm}
For any $\varphi \in S_M$ and $a \in \mathcal{A}_0$,
\[
  |\varphi(a)| \le \|a\|_M.
\]
\end{lemma}

\begin{proof}
By definition, $\|a\|_M$ is the supremum of the set containing $|\varphi(a)|$.
\end{proof}

\begin{theorem}[Triangle Inequality]\label{thm:stateSeminorm_add}
For any $a, b \in \mathcal{A}_0$,
\[
  \|a + b\|_M \le \|a\|_M + \|b\|_M.
\]
\end{theorem}

\begin{proof}[Proof sketch]
For each $\varphi \in S_M$, linearity gives $\varphi(a+b) = \varphi(a) + \varphi(b)$, so
\[
  |\varphi(a+b)| \le |\varphi(a)| + |\varphi(b)|.
\]
Taking the supremum over $\varphi$ on both sides and using the additivity of suprema yields the result.
\end{proof}

%% ============================================================
\section{Seminorm Properties}
\label{sec:seminorm-props}

We verify that $\|{\cdot}\|_M$ satisfies all the axioms of a seminorm over $\mathbb{R}$.

\begin{lemma}[Zero]\label{lem:stateSeminorm_zero}
$\|0\|_M = 0$.
\end{lemma}

\begin{proof}
For any $\varphi \in S_M$, linearity gives $\varphi(0) = 0$, so $|\varphi(0)| = 0$.
The supremum of a constant function is that constant.
\end{proof}

\begin{lemma}[Negation]\label{lem:stateSeminorm_neg}
For any $a \in \mathcal{A}_0$, $\|-a\|_M = \|a\|_M$.
\end{lemma}

\begin{proof}
Linearity gives $\varphi(-a) = -\varphi(a)$, so $|\varphi(-a)| = |\varphi(a)|$.
The sets over which we take suprema are identical.
\end{proof}

\begin{theorem}[Homogeneity]\label{thm:stateSeminorm_smul}
For any $c \in \mathbb{R}$ and $a \in \mathcal{A}_0$,
\[
  \|c \cdot a\|_M = |c| \cdot \|a\|_M.
\]
\end{theorem}

\begin{proof}[Proof sketch]
If $c = 0$, both sides are zero by Lemma~\ref{lem:stateSeminorm_zero}.
Otherwise, linearity gives $|\varphi(c \cdot a)| = |c| \cdot |\varphi(a)|$.
The supremum factors out the constant $|c| \ge 0$.
\end{proof}

\begin{definition}[Seminorm Instance]\label{def:stateSeminormSeminorm}
The state seminorm $\|{\cdot}\|_M$ forms a \textbf{seminorm} over $\mathbb{R}$ on $\mathcal{A}_0$,
constructed via the triangle inequality (Theorem~\ref{thm:stateSeminorm_add}) and
homogeneity (Theorem~\ref{thm:stateSeminorm_smul}).
\end{definition}

\begin{remark}
The seminorm $\|{\cdot}\|_M$ is not necessarily a norm: we may have $\|a\|_M = 0$ for nonzero $a$.
However, the kernel $\{a : \|a\|_M = 0\}$ is precisely the intersection of all GNS null spaces.
\end{remark}

%% ============================================================
\section{Closure of the Quadratic Module}
\label{sec:closure}

We now define the closure of $M$ in the topology induced by $\|{\cdot}\|_M$.

\begin{definition}[Quadratic Module Closure]\label{def:quadraticModuleClosure}
The \textbf{closure} of $M$, denoted $\bar{M}$, is defined as
\[
  \bar{M} = \{a \in \mathcal{A}_0 : \forall \varepsilon > 0,\, \exists m \in M,\, \|a - m\|_M < \varepsilon\}.
\]
\end{definition}

\begin{theorem}[Module Subset]\label{thm:quadraticModule_subset_closure}
$M \subseteq \bar{M}$.
\end{theorem}

\begin{proof}
Let $m \in M$. For any $\varepsilon > 0$, take $m' = m$. Then $\|m - m'\|_M = \|0\|_M = 0 < \varepsilon$.
\end{proof}

\begin{lemma}[Zero in Closure]\label{lem:zero_mem_closure}
$0 \in \bar{M}$.
\end{lemma}

\begin{proof}
Since $0 = 0^* \cdot 0 \in M$, this follows from Theorem~\ref{thm:quadraticModule_subset_closure}.
\end{proof}

\begin{theorem}[Closure Closed under Addition]\label{thm:closure_add_mem}
If $a, b \in \bar{M}$, then $a + b \in \bar{M}$.
\end{theorem}

\begin{proof}[Proof sketch]
Let $\varepsilon > 0$. By definition, there exist $m_a, m_b \in M$ with
$\|a - m_a\|_M < \varepsilon/2$ and $\|b - m_b\|_M < \varepsilon/2$.
Since $M$ is closed under addition, $m_a + m_b \in M$. Then
\begin{align*}
  \|a + b - (m_a + m_b)\|_M &= \|(a - m_a) + (b - m_b)\|_M \\
  &\le \|a - m_a\|_M + \|b - m_b\|_M \\
  &< \varepsilon/2 + \varepsilon/2 = \varepsilon.
\end{align*}
\end{proof}

\begin{theorem}[Closure Closed under Scalar Multiplication]\label{thm:closure_smul_mem}
If $c \ge 0$ and $a \in \bar{M}$, then $c \cdot a \in \bar{M}$.
\end{theorem}

\begin{proof}[Proof sketch]
If $c = 0$, then $c \cdot a = 0 \in \bar{M}$ by Lemma~\ref{lem:zero_mem_closure}.

Otherwise $c > 0$. Let $\varepsilon > 0$. Choose $m \in M$ with $\|a - m\|_M < \varepsilon/c$.
Since $M$ is closed under nonnegative scalar multiplication, $c \cdot m \in M$. Then
\begin{align*}
  \|c \cdot a - c \cdot m\|_M &= \|c \cdot (a - m)\|_M \\
  &= |c| \cdot \|a - m\|_M \\
  &= c \cdot \|a - m\|_M \\
  &< c \cdot (\varepsilon/c) = \varepsilon.
\end{align*}
\end{proof}

\begin{remark}
The closure $\bar{M}$ forms a cone: it is closed under addition and nonnegative scalar
multiplication. This structure is essential for the dual characterization via the Riesz
extension theorem in Chapter~\ref{ch:dual}.
\end{remark}


%% ============================================================
%% Part III: Separation and Extension
%% ============================================================
\part{Separation and Extension}

% Auto-generated from Lean 4 source
% Source files: Forward.lean, SpanIntersection.lean, SeparatingFunctional.lean,
%               RieszApplication.lean, ComplexExtension.lean, Normalization.lean

\chapter{Dual Characterization}
\label{ch:dual}

This chapter establishes the dual characterization of the Archimedean closure
$\bar{M}$. The main result shows that membership in the closure can be detected
by evaluation against $M$-positive states: an element $A$ belongs to $\bar{M}$
if and only if $\varphi(A) \ge 0$ for all $M$-positive states $\varphi$.

%% ============================================================
\section{Forward Direction}
\label{sec:forward}

This section proves the forward direction of the dual characterization:
if $A$ is in the closure of the quadratic module $M$ (in the $\|\cdot\|_M$
topology), then $\varphi(A) \ge 0$ for all $M$-positive states $\varphi$.

The main result is:
\begin{itemize}
  \item \texttt{closure\_implies\_nonneg} (Theorem~\ref{thm:closure_implies_nonneg}):
        $A \in \bar{M}$ implies $\varphi(A) \ge 0$ for all $\varphi \in S_M$
\end{itemize}

\begin{lemma}[Linearity of States]\label{lem:apply_sub}
For any $M$-positive state $\varphi$ and elements $a, m \in \mathcal{A}_0$,
\[
  \varphi(a - m) = \varphi(a) - \varphi(m).
\]
\end{lemma}

\begin{proof}
Immediate from the linearity of $\varphi$.
\end{proof}

\begin{theorem}[Closure Implies Nonnegativity]\label{thm:closure_implies_nonneg}
If $A \in \bar{M}$, then $\varphi(A) \ge 0$ for all $M$-positive states $\varphi$.
\end{theorem}

\begin{proof}[Proof sketch]
Suppose, for contradiction, that $\varphi(A) < 0$. Set $\varepsilon = -\varphi(A) > 0$.
By definition of closure, there exists $m \in M$ with $\|A - m\|_M < \varepsilon$.

Since $|\varphi(A - m)| \le \|A - m\|_M$ (by the seminorm bound on states) and
$\varphi(m) \ge 0$ (by $M$-positivity), we have:
\begin{align*}
  |\varphi(A - m)| &< -\varphi(A) = \varepsilon \\
  \Rightarrow\quad \varphi(A) &< \varphi(A) - \varphi(m) + \varepsilon
\end{align*}
This yields $\varphi(m) < \varepsilon = -\varphi(A)$. Combined with
$\varphi(A) - \varphi(m) < -\varphi(A)$ from the absolute value bound,
we obtain a contradiction via linear arithmetic.
\end{proof}

%% ============================================================
\section{Span Intersection Lemma}
\label{sec:span-intersection}

This section proves that if $A \notin \bar{M}$, then positive scalar multiples
of $A$ cannot belong to $M$. This is the key result needed for constructing
the separating functional in the Riesz extension argument.

The main results are:
\begin{itemize}
  \item \texttt{positive\_smul\_not\_in\_M} (Lemma~\ref{lem:positive_smul_not_in_M}):
        If $A \notin \bar{M}$ and $c > 0$, then $c \cdot A \notin M$
  \item \texttt{separating\_nonneg\_on\_span\_cap\_M}
        (Theorem~\ref{thm:separating_nonneg_on_span_cap_M}):
        A separating functional $\psi_0(\lambda A) = -\lambda\varepsilon$
        is automatically nonnegative on $M \cap \mathrm{span}\{A\}$
\end{itemize}

\begin{lemma}[Positive Scalar Multiples]\label{lem:positive_smul_not_in_M}
If $A \notin \bar{M}$ and $c > 0$, then $c \cdot A \notin M$.
\end{lemma}

\begin{proof}[Proof sketch]
Suppose $c \cdot A \in M$. Since $M$ is a cone and $c^{-1} > 0$, we have
\[
  A = c^{-1} \cdot (c \cdot A) \in M.
\]
But $M \subseteq \bar{M}$, so $A \in \bar{M}$, contradicting $A \notin \bar{M}$.
\end{proof}

\begin{corollary}\label{cor:self_not_in_M}
If $A \notin \bar{M}$, then $A \notin M$.
\end{corollary}

\begin{proof}
Apply Lemma~\ref{lem:positive_smul_not_in_M} with $c = 1$.
\end{proof}

\begin{theorem}[Separating Functional Nonnegativity]\label{thm:separating_nonneg_on_span_cap_M}
For $A \notin \bar{M}$ and $\varepsilon > 0$, the functional
$\psi_0(\lambda A) = -\lambda \varepsilon$ is nonnegative on $M \cap \mathrm{span}\{A\}$.
\end{theorem}

\begin{proof}[Proof sketch]
Elements of $M \cap \mathrm{span}\{A\}$ have the form $\lambda A$ for some
$\lambda \in \mathbb{R}$ with $\lambda A \in M$. We consider two cases:
\begin{itemize}
  \item If $\lambda > 0$: By Lemma~\ref{lem:positive_smul_not_in_M},
        $\lambda A \notin M$, so this case is impossible.
  \item If $\lambda \le 0$: Then $\psi_0(\lambda A) = -\lambda \varepsilon \ge 0$.
\end{itemize}
Thus $\psi_0$ is nonnegative on the intersection.
\end{proof}

\begin{lemma}[Coefficient Sign]\label{lem:span_cap_M_nonpos_coeff}
If $A \notin \bar{M}$ and $c \cdot A \in M$ for some $c \in \mathbb{R}$,
then $c \le 0$.
\end{lemma}

\begin{proof}
If $c > 0$, then by Lemma~\ref{lem:positive_smul_not_in_M}, $c \cdot A \notin M$,
a contradiction.
\end{proof}

%% ============================================================
\section{Constructing the Separating Functional}
\label{sec:separating-functional}

This section constructs a linear functional $\psi_0$ on $\mathrm{span}\{A\}$
that separates $A$ from $M$. When $A \notin \bar{M}$, we define
$\psi_0(\lambda A) = -\lambda$, achieving $\psi_0(A) = -1 < 0$ while
$\psi_0 \ge 0$ on $M \cap \mathrm{span}\{A\}$.

The main results are:
\begin{itemize}
  \item \texttt{not\_in\_closure\_ne\_zero} (Lemma~\ref{lem:not_in_closure_ne_zero}):
        $A \notin \bar{M}$ implies $A \ne 0$
  \item \texttt{separatingOnSpan} (Definition~\ref{def:separatingOnSpan}):
        The linear map $\psi_0 : \mathrm{span}\{A\} \to \mathbb{R}$ with
        $\psi_0(A) = -1$
  \item \texttt{separatingOnSpan\_nonneg\_on\_M\_cap\_span}
        (Theorem~\ref{thm:separatingOnSpan_nonneg}):
        $\psi_0 \ge 0$ on $M \cap \mathrm{span}\{A\}$
\end{itemize}

\begin{lemma}[Non-Zero Outside Closure]\label{lem:not_in_closure_ne_zero}
If $A \notin \bar{M}$, then $A \ne 0$.
\end{lemma}

\begin{proof}
We have $0 = 0^* \cdot 0 \in M \subseteq \bar{M}$. Thus if $A \notin \bar{M}$,
then $A \ne 0$.
\end{proof}

\begin{definition}[Separating Functional on Span]\label{def:separatingOnSpan}
For $A \notin \bar{M}$, the \textbf{separating functional}
$\psi_0 : \mathrm{span}\{A\} \to \mathbb{R}$ is defined by
\[
  \psi_0(\lambda A) = -\lambda.
\]
This is well-defined since $A \ne 0$ by Lemma~\ref{lem:not_in_closure_ne_zero}.
\end{definition}

\begin{remark}
The construction uses \texttt{LinearPMap.mkSpanSingleton}, which defines
a linear map on the span of a single non-zero vector by specifying its
value on that vector.
\end{remark}

\begin{lemma}[Value at $A$]\label{lem:separatingOnSpan_apply_A}
For $A \notin \bar{M}$, $\psi_0(A) = -1$.
\end{lemma}

\begin{proof}
By definition of $\psi_0$, taking $\lambda = 1$.
\end{proof}

\begin{corollary}\label{cor:separatingOnSpan_apply_A_neg}
For $A \notin \bar{M}$, $\psi_0(A) < 0$.
\end{corollary}

\begin{proof}
Immediate from $\psi_0(A) = -1 < 0$.
\end{proof}

\begin{lemma}[Value at Scalar Multiples]\label{lem:separatingOnSpan_apply_smul}
For $A \notin \bar{M}$ and $c \in \mathbb{R}$, $\psi_0(c \cdot A) = -c$.
\end{lemma}

\begin{proof}
By linearity of $\psi_0$ and the fact that $\psi_0(A) = -1$:
\[
  \psi_0(c \cdot A) = c \cdot \psi_0(A) = c \cdot (-1) = -c.
\]
\end{proof}

\begin{theorem}[Nonnegativity on Intersection]\label{thm:separatingOnSpan_nonneg}
For $A \notin \bar{M}$, the functional $\psi_0$ is nonnegative on
$M \cap \mathrm{span}\{A\}$.
\end{theorem}

\begin{proof}[Proof sketch]
Let $x \in M \cap \mathrm{span}\{A\}$. Then $x = c \cdot A$ for some
$c \in \mathbb{R}$, and $\psi_0(x) = -c$.

By Lemma~\ref{lem:span_cap_M_nonpos_coeff}, since $x = c \cdot A \in M$
and $A \notin \bar{M}$, we have $c \le 0$.

Therefore $\psi_0(x) = -c \ge 0$.
\end{proof}

%% ============================================================
\section{Separating Functional via Geometric Hahn--Banach}
\label{sec:riesz-application}

This section constructs a separating functional for the dual characterization theorem
using geometric separation in locally convex spaces.

The key insight is to use \texttt{ProperCone.hyperplane\_separation\_point} from mathlib,
which provides the geometric Hahn--Banach separation theorem for locally convex spaces.
This approach avoids the ``generating condition'' difficulties of the classical Riesz
extension theorem.

\subsection{Setup}

The construction proceeds in four steps:
\begin{enumerate}
  \item \textbf{Topology}: $\mathcal{A}_0$ is equipped with the topology from the state
        seminorm $\|\cdot\|_M$
  \item \textbf{Locally Convex Space}: Seminorm topologies are always locally convex
  \item \textbf{Proper Cone}: The closure $\bar{M}$ forms a proper cone:
        \begin{itemize}
          \item Nonempty ($0 \in \bar{M}$)
          \item Closed (by definition of closure)
          \item Cone (closed under addition and $\mathbb{R}_{\ge 0}$ scaling)
        \end{itemize}
  \item \textbf{Separation}: For $A \notin \bar{M}$, there exists a continuous $f$
        with $f \ge 0$ on $\bar{M}$ and $f(A) < 0$
\end{enumerate}

\begin{definition}[Quadratic Module Closure as Proper Cone]\label{def:quadraticModuleClosureProperCone}
The closure $\bar{M}$ forms a \textbf{proper cone} in $\mathcal{A}_0$, i.e., a nonempty
closed convex cone. The cone structure is inherited from the quadratic module:
\begin{itemize}
  \item $0 \in \bar{M}$ (from closure of $M$ containing $0$)
  \item $a, b \in \bar{M} \Rightarrow a + b \in \bar{M}$ (closure preserves addition)
  \item $c \ge 0, a \in \bar{M} \Rightarrow c \cdot a \in \bar{M}$ (closure preserves
        nonnegative scaling)
\end{itemize}
\end{definition}

\begin{lemma}\label{lem:mem_quadraticModuleClosureProperCone}
Membership in the proper cone is equivalent to membership in $\bar{M}$:
\[
  a \in \mathsf{quadraticModuleClosureProperCone} \iff a \in \bar{M}.
\]
\end{lemma}

\begin{theorem}[Existence of Separating Functional]\label{thm:riesz_extension_exists}
Let $A \in \mathcal{A}_0$ be self-adjoint with $A \notin \bar{M}$. Then there exists
a linear functional $\psi : \mathcal{A}_0 \to \mathbb{R}$ such that:
\begin{enumerate}
  \item $\psi(m) \ge 0$ for all $m \in M$
  \item $\psi(A) < 0$
\end{enumerate}
\end{theorem}

\begin{proof}[Proof sketch]
Apply \texttt{ProperCone.hyperplane\_separation\_point} to the proper cone $\bar{M}$
and the point $A \notin \bar{M}$. This yields a continuous linear functional $f$
with $f \ge 0$ on $\bar{M}$ and $f(A) < 0$. Since $M \subseteq \bar{M}$, we have
$f \ge 0$ on $M$. The underlying linear map gives the desired $\psi$.
\end{proof}

%% ============================================================
\section{Symmetrization of Real Functional}
\label{sec:complex-extension}

The separation theorem yields $\psi : \mathcal{A}_0 \to \mathbb{R}$ with $\psi \ge 0$
on $M$ and $\psi(A) < 0$. However, an $M$-positive state requires symmetry:
$\varphi(a^*) = \varphi(a)$. This section constructs the symmetrization.

\begin{definition}[Star as Linear Map]\label{def:starAsLinearMap}
The star operation $a \mapsto a^*$ defines an $\mathbb{R}$-linear map
$\mathsf{star} : \mathcal{A}_0 \to \mathcal{A}_0$. This uses that for $r \in \mathbb{R}$,
we have $(\mathsf{algebraMap}(r))^* = \mathsf{algebraMap}(r)$.
\end{definition}

\begin{definition}[Symmetrization]\label{def:symmetrize}
Given $\psi : \mathcal{A}_0 \to \mathbb{R}$ linear, the \textbf{symmetrization} is
\[
  \varphi(a) = \frac{\psi(a) + \psi(a^*)}{2}.
\]
\end{definition}

\begin{lemma}[Symmetrization is Symmetric]\label{lem:symmetrize_map_star}
For any linear $\psi$ and all $a \in \mathcal{A}_0$:
\[
  \varphi(a^*) = \varphi(a).
\]
\end{lemma}

\begin{proof}
Direct computation:
$\varphi(a^*) = \frac{\psi(a^*) + \psi((a^*)^*)}{2} = \frac{\psi(a^*) + \psi(a)}{2} = \varphi(a)$.
\end{proof}

\begin{lemma}[Symmetrization on Self-Adjoint Elements]\label{lem:symmetrize_eq_of_selfAdjoint}
If $a = a^*$, then $\varphi(a) = \psi(a)$.
\end{lemma}

\begin{proof}
When $a^* = a$: $\varphi(a) = \frac{\psi(a) + \psi(a)}{2} = \psi(a)$.
\end{proof}

\begin{lemma}[Elements of $M$ are Self-Adjoint]\label{lem:isSelfAdjoint_of_mem_quadraticModule}
Every element $m \in M$ satisfies $m^* = m$.
\end{lemma}

\begin{proof}[Proof sketch]
By induction on the structure of $M$:
\begin{itemize}
  \item Squares $a^* a$ are self-adjoint: $(a^* a)^* = a^* (a^*)^* = a^* a$
  \item Generator-weighted squares $b^* g_j b$ are self-adjoint (since $g_j^* = g_j$)
  \item Self-adjointness is preserved by addition and nonnegative scaling
\end{itemize}
\end{proof}

\begin{lemma}[Symmetrization Preserves $M$-Nonnegativity]\label{lem:symmetrize_nonneg_on_M}
If $\psi(m) \ge 0$ for all $m \in M$, then $\varphi(m) \ge 0$ for all $m \in M$.
\end{lemma}

\begin{proof}
For $m \in M$, we have $m^* = m$ by Lemma~\ref{lem:isSelfAdjoint_of_mem_quadraticModule}.
Thus $\varphi(m) = \psi(m) \ge 0$ by Lemma~\ref{lem:symmetrize_eq_of_selfAdjoint}.
\end{proof}

\begin{lemma}[Symmetrization Preserves Negativity]\label{lem:symmetrize_neg_of_selfAdjoint}
If $A^* = A$ and $\psi(A) < 0$, then $\varphi(A) < 0$.
\end{lemma}

\begin{proof}
By Lemma~\ref{lem:symmetrize_eq_of_selfAdjoint}, $\varphi(A) = \psi(A) < 0$.
\end{proof}

\begin{theorem}[Symmetric Separating Functional]\label{thm:symmetrize_separation}
Let $A \in \mathcal{A}_0$ be self-adjoint with $A \notin \bar{M}$. Then there exists
a linear functional $\varphi : \mathcal{A}_0 \to \mathbb{R}$ such that:
\begin{enumerate}
  \item $\varphi(a^*) = \varphi(a)$ for all $a$
  \item $\varphi(m) \ge 0$ for all $m \in M$
  \item $\varphi(A) < 0$
\end{enumerate}
\end{theorem}

\begin{proof}[Proof sketch]
Apply Theorem~\ref{thm:riesz_extension_exists} to obtain $\psi$, then symmetrize.
\end{proof}

%% ============================================================
\section{Normalization to $M$-Positive State}
\label{sec:normalization}

This section normalizes the symmetric separating functional to obtain a true
$M$-positive state with $\varphi(1) = 1$.

\subsection{Cauchy--Schwarz for Symmetric Functionals}

\begin{theorem}[Cauchy--Schwarz for Symmetric Functionals]\label{thm:cauchy_schwarz_general}
Let $\varphi : \mathcal{A}_0 \to \mathbb{R}$ be linear with $\varphi(a^*) = \varphi(a)$
and $\varphi(m) \ge 0$ for $m \in M$. Then for all $a \in \mathcal{A}_0$:
\[
  \varphi(a)^2 \le \varphi(a^* a) \cdot \varphi(1).
\]
\end{theorem}

\begin{proof}[Proof sketch]
Consider the quadratic $q(t) = \varphi((a + t \cdot 1)^*(a + t \cdot 1)) \ge 0$ for
$t \in \mathbb{R}$. Expanding yields
\[
  q(t) = \varphi(a^* a) + 2t \cdot \varphi(a) + t^2 \cdot \varphi(1).
\]
If $\varphi(1) = 0$, the quadratic reduces to a linear function, and nonnegativity
for all $t$ forces $\varphi(a) = 0$. If $\varphi(1) \ne 0$, the discriminant condition
$4\varphi(a)^2 - 4\varphi(a^* a)\varphi(1) \le 0$ gives the result.
\end{proof}

\subsection{Positivity of $\varphi(1)$}

\begin{theorem}[$\varphi(1) > 0$]\label{thm:phi_one_pos}
Let $\varphi : \mathcal{A}_0 \to \mathbb{R}$ be symmetric with $\varphi \ge 0$ on $M$.
If $A^* = A$ and $\varphi(A) < 0$, then $\varphi(1) > 0$.
\end{theorem}

\begin{proof}[Proof sketch]
We rule out the cases $\varphi(1) = 0$ and $\varphi(1) < 0$:

\textbf{Case $\varphi(1) = 0$:} By Cauchy--Schwarz, $\varphi(a)^2 \le \varphi(a^* a) \cdot 0 = 0$
for all $a$. Thus $\varphi \equiv 0$, contradicting $\varphi(A) < 0$.

\textbf{Case $\varphi(1) < 0$:} By the Archimedean property, there exists $N \in \mathbb{N}$
with $N \cdot 1 - A^* A \in M$. Then $N \cdot \varphi(1) \ge \varphi(A^* A) \ge 0$.
Since $\varphi(1) < 0$ and $N \ge 0$, this forces $\varphi(A^* A) = 0$.
By Cauchy--Schwarz, $\varphi(A)^2 \le 0$, so $\varphi(A) = 0$. Contradiction.
\end{proof}

\subsection{Normalized State}

\begin{definition}[Normalized $M$-Positive State]\label{def:normalizedMPositiveState}
Given $\varphi$ symmetric with $\varphi \ge 0$ on $M$ and $\varphi(1) > 0$, define
\[
  \tilde{\varphi}(a) = \frac{\varphi(a)}{\varphi(1)}.
\]
Then $\tilde{\varphi}$ is an $M$-positive state: $\tilde{\varphi}(a^*) = \tilde{\varphi}(a)$,
$\tilde{\varphi}(1) = 1$, and $\tilde{\varphi}(m) \ge 0$ for $m \in M$.
\end{definition}

\begin{lemma}[Normalization Preserves Sign]\label{lem:normalizedMPositiveState_negative}
If $\varphi(A) < 0$ and $\varphi(1) > 0$, then $\tilde{\varphi}(A) < 0$.
\end{lemma}

\begin{proof}
$\tilde{\varphi}(A) = \varphi(A) / \varphi(1) < 0$ since the numerator is negative
and the denominator is positive.
\end{proof}

\subsection{Main Theorem}

\begin{theorem}[Existence of Negative $M$-Positive State]\label{thm:exists_MPositiveState_negative}
Let $A \in \mathcal{A}_0$ be self-adjoint with $A \notin \bar{M}$. Then there exists
an $M$-positive state $\varphi \in S_M$ with $\varphi(A) < 0$.
\end{theorem}

\begin{proof}[Proof sketch]
Combine the previous results:
\begin{enumerate}
  \item By Theorem~\ref{thm:symmetrize_separation}, obtain symmetric $\psi$ with
        $\psi \ge 0$ on $M$ and $\psi(A) < 0$
  \item By Theorem~\ref{thm:phi_one_pos}, $\psi(1) > 0$
  \item By Definition~\ref{def:normalizedMPositiveState}, normalize to get
        $\varphi \in S_M$ with $\varphi(A) < 0$
\end{enumerate}
\end{proof}

\begin{remark}
This theorem is the key ingredient for the backward direction of the main dual
characterization: if $A \notin \bar{M}$, there exists a state witnessing this via
$\varphi(A) < 0$. Combined with the forward direction (states are nonnegative on
$\bar{M}$), this yields the equivalence
\[
  A \in \bar{M} \iff \varphi(A) \ge 0 \text{ for all } \varphi \in S_M.
\]
\end{remark}



%% ============================================================
%% Part IV: GNS Construction
%% ============================================================
\part{The GNS Construction}

% Auto-generated from Lean 4 source
% Source files: AfTests/GNS/NullSpace/*.lean, AfTests/GNS/Representation/*.lean,
%               AfTests/GNS/PreHilbert/*.lean, AfTests/GNS/HilbertSpace/*.lean,
%               AfTests/GNS/State/*.lean, AfTests/GNS/Main/*.lean
% Do not edit manually

\chapter{The GNS Construction}
\label{ch:gns}

The Gelfand-Naimark-Segal (GNS) construction is one of the fundamental theorems in the theory of $C^*$-algebras. It shows that every state on a $C^*$-algebra can be realized as a vector state in some Hilbert space representation. This chapter presents the complete formalization of the GNS construction, including both the existence theorem and the uniqueness theorem up to unitary equivalence.

Given a state $\varphi$ on a $C^*$-algebra $A$, we construct:
\begin{enumerate}
    \item A Hilbert space $H_\varphi$ (the GNS Hilbert space)
    \item A $*$-representation $\pi_\varphi : A \to B(H_\varphi)$
    \item A cyclic unit vector $\Omega_\varphi \in H_\varphi$
\end{enumerate}
such that $\varphi(a) = \langle \Omega_\varphi, \pi_\varphi(a) \Omega_\varphi \rangle$ for all $a \in A$.

%----------------------------------------------------------------------
\section{States on $C^*$-Algebras}
\label{sec:states}

\begin{definition}[State]
\label{def:state}
A \emph{state} on a $C^*$-algebra $A$ is a continuous linear functional $\varphi : A \to \mathbb{C}$ satisfying:
\begin{enumerate}
    \item \emph{Positivity}: $\varphi(a^* a) \geq 0$ for all $a \in A$ (with $\varphi(a^* a) \in \mathbb{R}$)
    \item \emph{Normalization}: $\varphi(1) = 1$
\end{enumerate}
\end{definition}

The Lean formalization captures both components of positivity explicitly.

\begin{lstlisting}[language=lean]
structure State (A : Type*) [CStarAlgebra A] where
  toContinuousLinearMap : A ->L[C] C
  map_star_mul_self_nonneg : forall a : A, 0 <= (toContinuousLinearMap (star a * a)).re
  map_star_mul_self_real : forall a : A, (toContinuousLinearMap (star a * a)).im = 0
  map_one : toContinuousLinearMap 1 = 1
\end{lstlisting}

\begin{theorem}[Star Preservation]
\label{thm:state-map-star}
States preserve the star operation: $\varphi(a^*) = \overline{\varphi(a)}$ for all $a \in A$.
\end{theorem}

\begin{proof}
The proof uses a polarization identity. Define the sesquilinear form $\langle a, b \rangle_\varphi = \varphi(b^* a)$. Since $\varphi(z^* z) \in \mathbb{R}$ for all $z$ (part of the state axioms), the polarization identity shows that this form satisfies conjugate symmetry: $\langle a, b \rangle_\varphi = \overline{\langle b, a \rangle_\varphi}$. Setting $b = 1$ gives $\varphi(a) = \overline{\varphi(a^*)}$, hence $\varphi(a^*) = \overline{\varphi(a)}$.

The Lean proof uses \texttt{sesqForm\_conj\_symm} and analyzes real and imaginary parts separately.
\end{proof}

%----------------------------------------------------------------------
\section{The Cauchy-Schwarz Inequality}
\label{sec:cauchy-schwarz}

The Cauchy-Schwarz inequality for states is crucial for proving that the null space is closed under addition.

\begin{theorem}[Cauchy-Schwarz for States]
\label{thm:cauchy-schwarz}
For any state $\varphi$ on a $C^*$-algebra $A$ and any $a, b \in A$:
\[
|\varphi(b^* a)|^2 \leq \varphi(a^* a) \cdot \varphi(b^* b)
\]
\end{theorem}

\begin{proof}
The proof proceeds in two stages.

\emph{Weak form}: First, we establish a version with an extra factor of 2. For $t \in \mathbb{R}$, the positivity of $\varphi((a + t \cdot b)^*(a + t \cdot b)) \geq 0$ gives a quadratic in $t$:
\[
\varphi(b^* b) \cdot t^2 + 2 \operatorname{Re}(\varphi(b^* a)) \cdot t + \varphi(a^* a) \geq 0
\]
By the discriminant lemma (\texttt{discrim\_le\_zero}), we get $\operatorname{Re}(\varphi(b^* a))^2 \leq \varphi(a^* a) \cdot \varphi(b^* b)$. Applying the same argument to $(a + it \cdot b)$ yields the same bound for the imaginary part, giving $|\varphi(b^* a)|^2 \leq 2 \cdot \varphi(a^* a) \cdot \varphi(b^* b)$.

\emph{Tight form}: For the sharp bound, we optimize over complex $\mu$. If $\varphi(b^* b) = 0$, the weak form immediately gives $|\varphi(b^* a)|^2 = 0$. Otherwise, set $\mu = -\varphi(b^* a) / \varphi(b^* b)$. The positivity of $\varphi((a + \mu b)^*(a + \mu b)) \geq 0$ expands to:
\[
\varphi(a^* a) - \frac{|\varphi(b^* a)|^2}{\varphi(b^* b)} \geq 0
\]
which gives the tight bound. The algebraic manipulation uses \texttt{cross\_term\_opt\_identity}.
\end{proof}

\begin{corollary}
\label{cor:null-consequence}
If $\varphi(a^* a) = 0$, then $\varphi(b^* a) = 0$ for all $b \in A$.
\end{corollary}

%----------------------------------------------------------------------
\section{The Null Space}
\label{sec:null-space}

\begin{definition}[GNS Null Space]
\label{def:null-space}
The \emph{GNS null space} $N_\varphi$ is defined as:
\[
N_\varphi = \{a \in A : \varphi(a^* a) = 0\}
\]
\end{definition}

The Lean formalization defines this as an \texttt{AddSubgroup}:
\begin{lstlisting}[language=lean]
def gnsNullSpace : AddSubgroup A where
  carrier := {a : A | phi (star a * a) = 0}
  zero_mem' := by simp [star_zero, map_zero]
  add_mem' := ...  -- uses Cauchy-Schwarz
  neg_mem' := ...
\end{lstlisting}

\begin{theorem}[Null Space is an Additive Subgroup]
\label{thm:null-subgroup}
$N_\varphi$ is closed under $0$, addition, negation, and scalar multiplication.
\end{theorem}

\begin{proof}
Closure under $0$ and negation is straightforward. For addition: if $a, b \in N_\varphi$, we expand
\[
\varphi((a+b)^*(a+b)) = \varphi(a^* a) + \varphi(a^* b) + \varphi(b^* a) + \varphi(b^* b)
\]
By Corollary~\ref{cor:null-consequence}, $\varphi(a^* a) = 0$ implies $\varphi(x^* a) = 0$ for all $x$, and similarly for $b$. Thus all four terms vanish.
\end{proof}

\begin{theorem}[Null Space is a Left Ideal]
\label{thm:null-left-ideal}
If $a \in N_\varphi$, then $ba \in N_\varphi$ for all $b \in A$.
\end{theorem}

\begin{proof}
We need $\varphi((ba)^*(ba)) = 0$. Computing $(ba)^*(ba) = a^* b^* b a$, we have
\[
\varphi(a^* \cdot (b^* b a)) = 0
\]
by the ``swapped'' Cauchy-Schwarz: if $\varphi(a^* a) = 0$, then $\varphi(a^* \cdot x) = 0$ for all $x$. This is \texttt{apply\_mul\_star\_eq\_zero\_of\_apply\_star\_self\_eq\_zero}.
\end{proof}

%----------------------------------------------------------------------
\section{Quotient Construction}
\label{sec:quotient}

The quotient $A / N_\varphi$ is the pre-Hilbert space on which we will define the inner product.

\begin{definition}[GNS Quotient]
\label{def:quotient}
The \emph{GNS quotient space} is $A / N_\varphi$, where $N_\varphi$ is viewed as a $\mathbb{C}$-submodule of $A$.
\end{definition}

\begin{lstlisting}[language=lean]
def gnsNullIdeal : Submodule C A where
  carrier := {a : A | phi (star a * a) = 0}
  add_mem' := fun {_ _} ha hb => phi.gnsNullSpace.add_mem ha hb
  zero_mem' := phi.gnsNullSpace.zero_mem
  smul_mem' := fun c {_} ha => gnsNullSpace_smul_mem phi ha c

abbrev gnsQuotient := A / phi.gnsNullIdeal
\end{lstlisting}

%----------------------------------------------------------------------
\section{Pre-Representation}
\label{sec:pre-rep}

\begin{definition}[Pre-Representation]
\label{def:pre-rep}
The \emph{GNS pre-representation} $\pi_\varphi(a) : A/N_\varphi \to A/N_\varphi$ is defined by left multiplication:
\[
\pi_\varphi(a)[b] = [ab]
\]
\end{definition}

\begin{theorem}
\label{thm:pre-rep-well-defined}
The pre-representation is well-defined (since $N_\varphi$ is a left ideal) and satisfies:
\begin{enumerate}
    \item $\pi_\varphi(ab) = \pi_\varphi(a) \circ \pi_\varphi(b)$ (multiplicative)
    \item $\pi_\varphi(1) = \mathrm{id}$
    \item $\pi_\varphi(a+b) = \pi_\varphi(a) + \pi_\varphi(b)$ (additive)
    \item $\pi_\varphi(c \cdot a) = c \cdot \pi_\varphi(a)$ (respects scalars)
\end{enumerate}
\end{theorem}

The Lean proofs use \texttt{Submodule.liftQ} for the well-definedness lift.

%----------------------------------------------------------------------
\section{Inner Product Structure}
\label{sec:inner-product}

\begin{definition}[GNS Inner Product]
\label{def:gns-inner}
The inner product on $A/N_\varphi$ is defined by:
\[
\langle [a], [b] \rangle = \varphi(b^* a)
\]
\end{definition}

\begin{theorem}[Well-Definedness]
\label{thm:inner-well-defined}
The inner product is well-defined on the quotient.
\end{theorem}

\begin{proof}
If $a_1 - a_2 \in N_\varphi$, then by Corollary~\ref{cor:null-consequence},
\[
\varphi(b^* a_1) - \varphi(b^* a_2) = \varphi(b^* (a_1 - a_2)) = 0
\]
Similarly for the second argument using conjugate symmetry.
\end{proof}

\begin{theorem}[Inner Product Properties]
\label{thm:inner-properties}
The inner product satisfies:
\begin{enumerate}
    \item Conjugate symmetry: $\langle x, y \rangle = \overline{\langle y, x \rangle}$
    \item Linearity in first argument
    \item Non-negativity: $\langle x, x \rangle \geq 0$ and $\langle x, x \rangle \in \mathbb{R}$
    \item Positive definiteness: $\langle x, x \rangle = 0 \Leftrightarrow x = 0$
\end{enumerate}
\end{theorem}

\begin{proof}
Conjugate symmetry follows from \texttt{sesqForm\_conj\_symm}. Non-negativity is inherited from positivity of states. Positive definiteness: $\langle [a], [a] \rangle = \varphi(a^* a) = 0$ iff $a \in N_\varphi$ iff $[a] = 0$.
\end{proof}

The norm on the quotient is $\|[a]\| = \sqrt{\operatorname{Re}(\varphi(a^* a))}$.

%----------------------------------------------------------------------
\section{Boundedness}
\label{sec:boundedness}

\begin{theorem}[Boundedness of Pre-Representation]
\label{thm:pre-rep-bounded}
For any $a \in A$ and $x \in A/N_\varphi$:
\[
\|\pi_\varphi(a) x\| \leq \|a\| \cdot \|x\|
\]
\end{theorem}

\begin{proof}
The key inequality from $C^*$-algebra theory is $a^* a \leq \|a\|^2 \cdot 1$ (spectral ordering). Since states are monotone on positive elements, we get
\[
\varphi(b^* (a^* a) b) \leq \|a\|^2 \cdot \varphi(b^* b)
\]
for any $b \in A$. This is the inequality \texttt{key\_inequality}. For $x = [b]$:
\[
\|\pi_\varphi(a)[b]\|^2 = \varphi((ab)^*(ab)) = \varphi(b^* a^* a b) \leq \|a\|^2 \cdot \varphi(b^* b) = \|a\|^2 \|[b]\|^2
\]
Taking square roots gives the result. The Lean proof uses \texttt{sq\_le\_sq\ensuremath{_0}} to convert from squared to non-squared norms.
\end{proof}

%----------------------------------------------------------------------
\section{Completion to Hilbert Space}
\label{sec:completion}

\begin{definition}[GNS Hilbert Space]
\label{def:gns-hilbert}
The \emph{GNS Hilbert space} $H_\varphi$ is the completion of $A/N_\varphi$ with respect to the norm induced by the inner product.
\end{definition}

\begin{lstlisting}[language=lean]
abbrev gnsHilbertSpace := UniformSpace.Completion phi.gnsQuotient

instance gnsHilbertSpaceCompleteSpace : CompleteSpace phi.gnsHilbertSpace :=
  UniformSpace.Completion.completeSpace phi.gnsQuotient
\end{lstlisting}

The inner product space structure on the quotient induces an inner product space structure on the completion via mathlib's \texttt{InnerProductSpace.Completion}.

%----------------------------------------------------------------------
\section{Extension of Operators}
\label{sec:extension}

\begin{theorem}[Extension to Hilbert Space]
\label{thm:extension}
The pre-representation $\pi_\varphi(a)$ extends uniquely to a bounded linear operator on $H_\varphi$.
\end{theorem}

\begin{proof}
By Theorem~\ref{thm:pre-rep-bounded}, $\pi_\varphi(a)$ is bounded on $A/N_\varphi$. Bounded linear maps on a dense subspace of a complete space extend uniquely by uniform continuity. The extension uses \texttt{UniformSpace.Completion.map}.
\end{proof}

\begin{definition}[GNS Representation]
\label{def:gns-rep}
The \emph{GNS representation} $\pi_\varphi : A \to B(H_\varphi)$ is defined by extending each $\pi_\varphi(a)$ to the completion.
\end{definition}

\begin{lstlisting}[language=lean]
noncomputable def gnsRep (a : A) : phi.gnsHilbertSpace ->L[C] phi.gnsHilbertSpace where
  toLinearMap := {
    toFun := UniformSpace.Completion.map (phi.gnsPreRepContinuous a)
    map_add' := ...
    map_smul' := ...
  }
  cont := UniformSpace.Completion.continuous_map
\end{lstlisting}

\begin{theorem}[Algebraic Properties]
\label{thm:gns-rep-algebra}
The GNS representation satisfies:
\begin{enumerate}
    \item $\pi_\varphi(ab) = \pi_\varphi(a) \circ \pi_\varphi(b)$
    \item $\pi_\varphi(1) = \mathrm{id}$
    \item $\pi_\varphi(a+b) = \pi_\varphi(a) + \pi_\varphi(b)$
\end{enumerate}
\end{theorem}

\begin{proof}
Each property is proven by density: both sides are continuous and agree on the dense quotient.
\end{proof}

%----------------------------------------------------------------------
\section{Star Structure}
\label{sec:star-structure}

\begin{theorem}[Star Preservation]
\label{thm:gns-rep-star}
The GNS representation preserves the star: $\pi_\varphi(a^*) = \pi_\varphi(a)^\dagger$.
\end{theorem}

\begin{proof}
We show $\langle \pi_\varphi(a^*) x, y \rangle = \langle x, \pi_\varphi(a) y \rangle$ for all $x, y$. By density, it suffices to check on quotient elements $x = [b]$, $y = [c]$:
\begin{align*}
\langle \pi_\varphi(a^*)[b], [c] \rangle &= \langle [a^* b], [c] \rangle = \varphi(c^* a^* b) \\
\langle [b], \pi_\varphi(a)[c] \rangle &= \langle [b], [ac] \rangle = \varphi((ac)^* b) = \varphi(c^* a^* b)
\end{align*}
The key calculation is \texttt{gnsPreRep\_inner\_star}.
\end{proof}

\begin{definition}[Star Algebra Homomorphism]
\label{def:star-alg-hom}
The GNS representation is a $*$-algebra homomorphism $A \to_{\ast\mathrm{alg}} B(H_\varphi)$.
\end{definition}

\begin{lstlisting}[language=lean]
noncomputable def gnsStarAlgHom : A ->*a[C] (phi.gnsHilbertSpace ->L[C] phi.gnsHilbertSpace) where
  toFun := phi.gnsRep
  map_one' := gnsRep_one phi
  map_mul' := fun a b => by rw [gnsRep_mul, ContinuousLinearMap.mul_def]
  map_star' := fun a => (gnsRep_star' phi a).symm
  ...
\end{lstlisting}

%----------------------------------------------------------------------
\section{The Cyclic Vector}
\label{sec:cyclic-vector}

\begin{definition}[Cyclic Vector]
\label{def:cyclic-vector}
The \emph{GNS cyclic vector} $\Omega_\varphi \in H_\varphi$ is the image of $[1]$ under the embedding of the quotient into the completion.
\end{definition}

\begin{lstlisting}[language=lean]
noncomputable def gnsCyclicVector : phi.gnsHilbertSpace :=
  (Submodule.Quotient.mk (p := phi.gnsNullIdeal) 1 : phi.gnsQuotient)
\end{lstlisting}

\begin{theorem}[Unit Norm]
\label{thm:cyclic-norm}
$\|\Omega_\varphi\| = 1$.
\end{theorem}

\begin{proof}
$\|\Omega_\varphi\|^2 = \langle [1], [1] \rangle = \varphi(1^* \cdot 1) = \varphi(1) = 1$.
\end{proof}

\begin{theorem}[Vector State Property]
\label{thm:vector-state}
For all $a \in A$:
\[
\varphi(a) = \langle \Omega_\varphi, \pi_\varphi(a) \Omega_\varphi \rangle
\]
\end{theorem}

\begin{proof}
We compute:
\begin{align*}
\langle \Omega_\varphi, \pi_\varphi(a) \Omega_\varphi \rangle &= \langle [1], \pi_\varphi(a)[1] \rangle \\
&= \langle [1], [a] \rangle \\
&= \varphi(1^* \cdot a) = \varphi(a)
\end{align*}
The proof uses \texttt{gnsRep\_cyclicVector} to show $\pi_\varphi(a)\Omega_\varphi = [a]$.
\end{proof}

\begin{theorem}[Cyclicity]
\label{thm:cyclicity}
The orbit $\{\pi_\varphi(a)\Omega_\varphi : a \in A\}$ is dense in $H_\varphi$.
\end{theorem}

\begin{proof}
Since $\pi_\varphi(a)\Omega_\varphi = [a]$ and the quotient map $A \to A/N_\varphi$ is surjective, the orbit equals the embedded image of $A/N_\varphi$, which is dense in the completion by construction.
\end{proof}

%----------------------------------------------------------------------
\section{The GNS Theorem}
\label{sec:gns-theorem}

\begin{theorem}[GNS Construction]
\label{thm:gns}
Let $\varphi$ be a state on a $C^*$-algebra $A$. There exists:
\begin{enumerate}
    \item A Hilbert space $H_\varphi$
    \item A $*$-representation $\pi_\varphi : A \to B(H_\varphi)$
    \item A unit vector $\Omega_\varphi \in H_\varphi$ with $\|\Omega_\varphi\| = 1$
\end{enumerate}
such that:
\begin{enumerate}
    \item[(a)] $\varphi(a) = \langle \Omega_\varphi, \pi_\varphi(a) \Omega_\varphi \rangle$ for all $a \in A$
    \item[(b)] $\{\pi_\varphi(a)\Omega_\varphi : a \in A\}$ is dense in $H_\varphi$
\end{enumerate}
\end{theorem}

\begin{proof}
The construction has been carried out in the preceding sections. The Lean statement is:
\begin{lstlisting}[language=lean]
theorem gns_theorem :
    ||phi.gnsCyclicVector|| = 1 /\
    (forall a : A, phi a = @inner C phi.gnsHilbertSpace _ phi.gnsCyclicVector
                      (phi.gnsRep a phi.gnsCyclicVector)) /\
    DenseRange (fun a : A => phi.gnsRep a phi.gnsCyclicVector) :=
  <gnsCyclicVector_norm phi, gns_vector_state phi, gnsCyclicVector_denseRange phi>
\end{lstlisting}
\end{proof}

%----------------------------------------------------------------------
\section{Uniqueness}
\label{sec:uniqueness}

The GNS representation is unique up to unitary equivalence.

\begin{theorem}[GNS Uniqueness]
\label{thm:gns-uniqueness}
Let $(H, \pi, \xi)$ be a cyclic $*$-representation with:
\begin{itemize}
    \item $\pi : A \to B(H)$ is a $*$-representation
    \item $\|\xi\| = 1$
    \item $\varphi(a) = \langle \xi, \pi(a)\xi \rangle$ for all $a \in A$
    \item $\{\pi(a)\xi : a \in A\}$ is dense in $H$
\end{itemize}
Then there exists a unitary $U : H_\varphi \to H$ such that:
\begin{enumerate}
    \item $U(\Omega_\varphi) = \xi$
    \item $U \circ \pi_\varphi(a) = \pi(a) \circ U$ for all $a \in A$
\end{enumerate}
\end{theorem}

The proof constructs the intertwiner $U$ in several stages.

\subsection{Construction of the Intertwiner}

\begin{definition}[Intertwiner on Quotient]
\label{def:intertwiner-quotient}
Define $U_0 : A/N_\varphi \to H$ by $U_0([a]) = \pi(a)\xi$.
\end{definition}

\begin{lemma}[Well-Definedness]
\label{lem:intertwiner-well-defined}
$U_0$ is well-defined: if $a - b \in N_\varphi$, then $\pi(a)\xi = \pi(b)\xi$.
\end{lemma}

\begin{proof}
If $\varphi((a-b)^*(a-b)) = 0$, then using the state condition:
\[
\|\pi(a-b)\xi\|^2 = \langle \xi, \pi((a-b)^*(a-b))\xi \rangle = \varphi((a-b)^*(a-b)) = 0
\]
so $\pi(a-b)\xi = 0$, giving $\pi(a)\xi = \pi(b)\xi$.
\end{proof}

\begin{lemma}[Isometry]
\label{lem:intertwiner-isometry}
$U_0$ is an isometry: $\|U_0([a])\| = \|[a]\|$.
\end{lemma}

\begin{proof}
\begin{align*}
\|U_0([a])\|^2 = \|\pi(a)\xi\|^2 &= \langle \xi, \pi(a)^\dagger \pi(a)\xi \rangle \\
&= \langle \xi, \pi(a^*)\pi(a)\xi \rangle \\
&= \langle \xi, \pi(a^* a)\xi \rangle \\
&= \varphi(a^* a) = \|[a]\|^2
\end{align*}
\end{proof}

\begin{lemma}[Linearity]
\label{lem:intertwiner-linear}
$U_0$ is linear: $U_0(x + y) = U_0(x) + U_0(y)$ and $U_0(c \cdot x) = c \cdot U_0(x)$.
\end{lemma}

\subsection{Extension to Hilbert Space}

\begin{lemma}[Extension]
\label{lem:intertwiner-extension}
$U_0$ extends uniquely to an isometry $U : H_\varphi \to H$.
\end{lemma}

\begin{proof}
As an isometry, $U_0$ is uniformly continuous. By the universal property of completions (\texttt{UniformSpace.Completion.extension}), it extends uniquely to the completion. The extension preserves the isometry property by \texttt{Isometry.completion\_extension}.
\end{proof}

\subsection{Surjectivity}

\begin{lemma}[Dense Range]
\label{lem:intertwiner-dense-range}
The range of $U$ contains the orbit $\{\pi(a)\xi : a \in A\}$, hence is dense.
\end{lemma}

\begin{proof}
For any $a \in A$, we have $U([a]) = \pi(a)\xi$, so $\pi(a)\xi \in \mathrm{range}(U)$.
\end{proof}

\begin{lemma}[Surjectivity]
\label{lem:intertwiner-surjective}
$U$ is surjective.
\end{lemma}

\begin{proof}
An isometry from a complete space into a complete space with dense range is surjective. The range is complete (image of complete space under uniform inducing map), hence closed. Dense and closed implies the whole space.
\end{proof}

\subsection{Intertwining Property}

\begin{lemma}[Intertwining]
\label{lem:intertwining}
$U \circ \pi_\varphi(a) = \pi(a) \circ U$ for all $a \in A$.
\end{lemma}

\begin{proof}
On quotient elements:
\[
U(\pi_\varphi(a)[b]) = U([ab]) = \pi(ab)\xi = \pi(a)\pi(b)\xi = \pi(a)(U([b]))
\]
Both sides are continuous, so by density the identity extends to all of $H_\varphi$.
\end{proof}

\subsection{Cyclic Vector Mapping}

\begin{lemma}[Cyclic Vector]
\label{lem:cyclic-mapping}
$U(\Omega_\varphi) = \xi$.
\end{lemma}

\begin{proof}
$U(\Omega_\varphi) = U([1]) = \pi(1)\xi = \xi$.
\end{proof}

\begin{proof}[Proof of Theorem~\ref{thm:gns-uniqueness}]
Combining the above lemmas, $U$ is a linear isometry equivalence (bijective isometry) that maps $\Omega_\varphi$ to $\xi$ and intertwines the representations. The Lean statement:
\begin{lstlisting}[language=lean]
theorem gns_uniqueness
    (_h_xi_norm : ||xi|| = 1)
    (h_xi_cyclic : DenseRange (fun a => pi a xi))
    (h_xi_state : forall a : A, @inner C H _ xi (pi a xi) = phi a) :
    exists U : phi.gnsHilbertSpace =~li[C] H,
      U phi.gnsCyclicVector = xi /\
      forall a : A, forall x : phi.gnsHilbertSpace, U (phi.gnsRep a x) = pi a (U x) :=
  <gnsIntertwinerEquiv phi pi xi h_xi_state h_xi_cyclic,
   gnsIntertwinerEquiv_cyclic phi pi xi h_xi_state h_xi_cyclic,
   gnsIntertwiner_intertwines phi pi xi h_xi_state h_xi_cyclic>
\end{lstlisting}
\end{proof}

%----------------------------------------------------------------------
\section{Summary of Key Definitions}
\label{sec:gns-summary}

For reference, the main Lean definitions in the GNS construction:

\begin{center}
\begin{tabular}{ll}
\hline
\textbf{Definition} & \textbf{Lean Name} \\
\hline
State & \texttt{State} \\
Null space & \texttt{State.gnsNullSpace} \\
Null ideal (submodule) & \texttt{State.gnsNullIdeal} \\
Quotient space & \texttt{State.gnsQuotient} \\
Inner product & \texttt{State.gnsInner} \\
Pre-representation & \texttt{State.gnsPreRep} \\
Hilbert space & \texttt{State.gnsHilbertSpace} \\
GNS representation & \texttt{State.gnsRep} \\
Star algebra homomorphism & \texttt{State.gnsStarAlgHom} \\
Cyclic vector & \texttt{State.gnsCyclicVector} \\
Intertwiner equivalence & \texttt{State.gnsIntertwinerEquiv} \\
\hline
\end{tabular}
\end{center}

The complete GNS formalization comprises approximately 2,455 lines of Lean code with zero sorries, providing a fully machine-checked proof of both the existence and uniqueness theorems.


%% ============================================================
%% Part V: Main Results
%% ============================================================
\part{Main Results}

% Auto-generated from Lean 4 source
% Source files: Constrained.lean, VectorState.lean, GNSConstrained.lean
% Do not edit manually

\chapter{Constrained Representations}
\label{ch:representations}

This chapter develops the theory of constrained $*$-representations, which are
the representation-theoretic counterpart to $M$-positive states. The generators
$g_j$ of the free $*$-algebra represent physical observables (such as position),
which should be positive operators. A constrained representation respects this
requirement: each $\pi(g_j)$ is a positive operator on the Hilbert space.

The main results establish the fundamental equivalence between states and
representations:
\begin{itemize}
  \item \texttt{ConstrainedStarRep} (Definition~\ref{def:constrained-rep}): constrained $*$-representations
  \item \texttt{vectorState} (Theorem~\ref{thm:vector-state-mpositive}): vector states from constrained reps are $M$-positive
  \item \texttt{state\_nonneg\_implies\_rep\_positive} (Theorem~\ref{thm:states-to-reps}): forward direction
  \item \texttt{gns\_constrained\_implies\_state\_nonneg} (Theorem~\ref{thm:reps-to-states}): backward direction
\end{itemize}

%% ============================================================
\section{Constrained $*$-Representations}
\label{sec:constrained-rep}

A constrained $*$-representation is a $*$-algebra homomorphism to bounded
operators on a Hilbert space, where the generators map to positive operators.

\begin{definition}[Constrained $*$-Representation]\label{def:constrained-rep}
A \textbf{constrained $*$-representation} of the free $*$-algebra $\mathcal{A}_0$
with $n$ generators consists of:
\begin{enumerate}
  \item A complex Hilbert space $H$;
  \item A $*$-algebra homomorphism $\pi : \mathcal{A}_0 \to \mathcal{B}(H)$ to bounded operators on $H$;
  \item The \emph{constraint}: for each generator $g_j$ ($j = 1, \ldots, n$), the operator $\pi(g_j)$ is positive, i.e., $\pi(g_j) \ge 0$.
\end{enumerate}
\end{definition}

\begin{remark}
The constraint $\pi(g_j) \ge 0$ means that $\langle \xi, \pi(g_j)\xi \rangle \ge 0$
for all $\xi \in H$. This reflects the physical interpretation of generators as
positive observables.
\end{remark}

\begin{lemma}[Representation Properties]\label{lem:rep-properties}
Let $\pi : \mathcal{A}_0 \to \mathcal{B}(H)$ be a constrained $*$-representation.
Then for all $a, b \in \mathcal{A}_0$:
\begin{enumerate}
  \item $\pi(1) = \mathrm{id}_H$ (preserves identity);
  \item $\pi(ab) = \pi(a)\pi(b)$ (preserves multiplication);
  \item $\pi(a^*) = \pi(a)^*$ (preserves adjoint).
\end{enumerate}
\end{lemma}

\begin{proof}
These follow directly from the definition of a $*$-algebra homomorphism.
\end{proof}

%% ============================================================
\section{Vector States}
\label{sec:vector-states}

Given a constrained representation $\pi$ and a unit vector $\xi \in H$, we
construct the associated vector state and prove it is $M$-positive.

\begin{definition}[Vector State Functional]\label{def:vector-state-fun}
Let $\pi : \mathcal{A}_0 \to \mathcal{B}(H)$ be a constrained $*$-representation
and $\xi \in H$ a unit vector ($\|\xi\| = 1$). The \textbf{vector state functional}
is
\[
  \varphi_\xi : \mathcal{A}_0 \to \mathbb{R}, \quad \varphi_\xi(a) = \mathrm{Re}\langle \xi, \pi(a)\xi \rangle.
\]
\end{definition}

\begin{lemma}[Linearity]\label{lem:vector-state-linear}
The vector state functional $\varphi_\xi$ is $\mathbb{R}$-linear:
\begin{enumerate}
  \item $\varphi_\xi(a + b) = \varphi_\xi(a) + \varphi_\xi(b)$;
  \item $\varphi_\xi(c \cdot a) = c \cdot \varphi_\xi(a)$ for $c \in \mathbb{R}$.
\end{enumerate}
\end{lemma}

\begin{proof}
Both properties follow from the linearity of the inner product in the second
argument and the linearity of taking the real part.
\end{proof}

\begin{lemma}[Symmetry]\label{lem:vector-state-star}
For any $a \in \mathcal{A}_0$,
\[
  \varphi_\xi(a^*) = \varphi_\xi(a).
\]
\end{lemma}

\begin{proof}
Using $\pi(a^*) = \pi(a)^*$:
\begin{align*}
  \varphi_\xi(a^*) &= \mathrm{Re}\langle \xi, \pi(a^*)\xi \rangle \\
  &= \mathrm{Re}\langle \xi, \pi(a)^*\xi \rangle \\
  &= \mathrm{Re}\langle \pi(a)\xi, \xi \rangle \\
  &= \mathrm{Re}\overline{\langle \xi, \pi(a)\xi \rangle} \\
  &= \mathrm{Re}\langle \xi, \pi(a)\xi \rangle = \varphi_\xi(a).
\end{align*}
\end{proof}

\begin{lemma}[Normalization]\label{lem:vector-state-one}
If $\|\xi\| = 1$, then $\varphi_\xi(1) = 1$.
\end{lemma}

\begin{proof}
$\varphi_\xi(1) = \mathrm{Re}\langle \xi, \pi(1)\xi \rangle = \mathrm{Re}\langle \xi, \xi \rangle = \|\xi\|^2 = 1$.
\end{proof}

\begin{lemma}[Positivity on Sums of Squares]\label{lem:vector-state-sos}
For any $a \in \mathcal{A}_0$,
\[
  \varphi_\xi(a^* a) \ge 0.
\]
\end{lemma}

\begin{proof}
Using $\pi(a^*) = \pi(a)^*$:
\begin{align*}
  \varphi_\xi(a^* a) &= \mathrm{Re}\langle \xi, \pi(a^* a)\xi \rangle \\
  &= \mathrm{Re}\langle \xi, \pi(a)^*\pi(a)\xi \rangle \\
  &= \mathrm{Re}\langle \pi(a)\xi, \pi(a)\xi \rangle \\
  &= \|\pi(a)\xi\|^2 \ge 0.
\end{align*}
\end{proof}

\begin{lemma}[Positivity on Generator Terms]\label{lem:vector-state-gen}
For any generator $g_j$ and $b \in \mathcal{A}_0$,
\[
  \varphi_\xi(b^* g_j b) \ge 0.
\]
\end{lemma}

\begin{proof}
Let $v = \pi(b)\xi$. Since $\pi(g_j) \ge 0$ is a positive operator:
\begin{align*}
  \varphi_\xi(b^* g_j b) &= \mathrm{Re}\langle \xi, \pi(b^* g_j b)\xi \rangle \\
  &= \mathrm{Re}\langle \xi, \pi(b)^*\pi(g_j)\pi(b)\xi \rangle \\
  &= \mathrm{Re}\langle v, \pi(g_j)v \rangle \ge 0.
\end{align*}
The last inequality uses the positivity of $\pi(g_j)$.
\end{proof}

\begin{theorem}[Vector States are $M$-Positive]\label{thm:vector-state-mpositive}
Let $\pi$ be a constrained $*$-representation and $\xi \in H$ a unit vector.
Then the vector state $\varphi_\xi$ is an $M$-positive state, i.e., $\varphi_\xi \in S_M$.
\end{theorem}

\begin{proof}
We verify the defining properties of an $M$-positive state:
\begin{enumerate}
  \item \emph{Linearity}: Lemma~\ref{lem:vector-state-linear}.
  \item \emph{Symmetry}: Lemma~\ref{lem:vector-state-star}.
  \item \emph{Normalization}: Lemma~\ref{lem:vector-state-one}.
  \item \emph{$M$-positivity}: For $m \in M$, we need $\varphi_\xi(m) \ge 0$.
    The quadratic module $M$ is generated by elements of the form $a^* a$ and
    $b^* g_j b$, and is closed under sums and positive scalar multiplication.
    By Lemmas~\ref{lem:vector-state-sos} and~\ref{lem:vector-state-gen},
    $\varphi_\xi$ is nonnegative on generators. By linearity, $\varphi_\xi(m) \ge 0$
    for all $m \in M$.
\end{enumerate}
\end{proof}

%% ============================================================
\section{GNS Representations are Constrained}
\label{sec:gns-constrained}

We now establish the key equivalence between positivity in states and positivity
in constrained representations. This requires the GNS construction developed
in Chapter~\ref{ch:gns}.

\begin{theorem}[GNS Representation Exists]\label{thm:gns-exists}
Let $\varphi$ be an $M$-positive state. Under the Archimedean assumption, there
exists a constrained $*$-representation $\pi_\varphi$ and a cyclic unit vector
$\Omega \in H_\varphi$ such that for all $a \in \mathcal{A}_0$:
\[
  \varphi(a) = \mathrm{Re}\langle \Omega, \pi_\varphi(a)\Omega \rangle.
\]
\end{theorem}

\begin{proof}[Proof sketch]
The construction follows the standard GNS procedure:
\begin{enumerate}
  \item Form the null space $\mathcal{N}_\varphi = \{a : \varphi(a^* a) = 0\}$.
  \item Quotient by $\mathcal{N}_\varphi$ and complete to obtain the Hilbert space $H_\varphi$.
  \item The left multiplication action extends to a $*$-representation $\pi_\varphi$.
  \item The image of $1 \in \mathcal{A}_0$ gives the cyclic vector $\Omega$.
\end{enumerate}
The representation is constrained because $\varphi(g_j) \ge 0$ (taking $b = 1$ in
$b^* g_j b \in M$) implies $\pi_\varphi(g_j)$ is a positive operator. The inner
product reconstruction follows from the cyclic vector identity.
\end{proof}

\begin{theorem}[Forward Direction: States to Representations]\label{thm:states-to-reps}
Let $A \in \mathcal{A}_0$ be self-adjoint. If $\varphi(A) \ge 0$ for all $M$-positive
states $\varphi \in S_M$, then $\pi(A) \ge 0$ for all constrained $*$-representations $\pi$.
\end{theorem}

\begin{proof}
Let $\pi : \mathcal{A}_0 \to \mathcal{B}(H)$ be a constrained representation.
To show $\pi(A) \ge 0$, we verify $\langle v, \pi(A)v \rangle \ge 0$ for all $v \in H$.

For $v = 0$, this is trivial. For $v \ne 0$, let $u = \|v\|^{-1} v$ be the
normalization. By Theorem~\ref{thm:vector-state-mpositive}, the vector state
$\varphi_u$ is $M$-positive. By hypothesis:
\[
  \varphi_u(A) = \mathrm{Re}\langle u, \pi(A)u \rangle \ge 0.
\]
Since $A$ is self-adjoint, $\pi(A)$ is self-adjoint, so $\langle u, \pi(A)u \rangle$
is real. Thus $\langle u, \pi(A)u \rangle \ge 0$.

Finally, $v = \|v\| \cdot u$, so:
\[
  \langle v, \pi(A)v \rangle = \|v\|^2 \langle u, \pi(A)u \rangle \ge 0.
\]
\end{proof}

\begin{theorem}[Backward Direction: Representations to States]\label{thm:reps-to-states}
Under the Archimedean assumption, let $A \in \mathcal{A}_0$ be self-adjoint.
If $\pi(A) \ge 0$ for all constrained $*$-representations $\pi$, then $\varphi(A) \ge 0$
for all $M$-positive states $\varphi \in S_M$.
\end{theorem}

\begin{proof}
Let $\varphi$ be an $M$-positive state. By Theorem~\ref{thm:gns-exists}, there
exists a constrained representation $\pi_\varphi$ and cyclic vector $\Omega$ with
$\|\Omega\| = 1$ such that:
\[
  \varphi(A) = \mathrm{Re}\langle \Omega, \pi_\varphi(A)\Omega \rangle.
\]
By hypothesis, $\pi_\varphi(A) \ge 0$. For a positive operator $T$ and any vector $\xi$:
\[
  \langle \xi, T\xi \rangle \ge 0.
\]
Therefore:
\[
  \varphi(A) = \mathrm{Re}\langle \Omega, \pi_\varphi(A)\Omega \rangle = \langle \Omega, \pi_\varphi(A)\Omega \rangle \ge 0.
\]
(The real part equals the full inner product since $\pi_\varphi(A)$ is self-adjoint.)
\end{proof}

\begin{corollary}[State-Representation Equivalence]\label{cor:state-rep-equiv}
Under the Archimedean assumption, for self-adjoint $A \in (\mathcal{A}_0)_{\mathrm{sa}}$:
\[
  \bigl(\forall \varphi \in S_M,\, \varphi(A) \ge 0\bigr)
  \iff
  \bigl(\forall \pi \text{ constrained},\, \pi(A) \ge 0\bigr).
\]
\end{corollary}

\begin{proof}
Combine Theorems~\ref{thm:states-to-reps} and~\ref{thm:reps-to-states}.
\end{proof}

\begin{remark}
Corollary~\ref{cor:state-rep-equiv} is the key bridge between the algebraic
characterization of positivity (via states) and the representation-theoretic
characterization (via constrained representations). This equivalence is
essential for the main theorem in Chapter~\ref{ch:main-theorem}.
\end{remark}

% Chapter: Main Theorem
% Generated from: DualCharacterization.lean, Theorem.lean
% Status: PLACEHOLDER - to be filled by LaTeX-P3.2

\chapter{Main Theorem}
\label{ch:main-theorem}

% TODO: Fill in from LaTeX-P3.2 task
\textbf{[PLACEHOLDER: To be generated from Lean source]}

\section{Dual Characterization Theorem}
\label{sec:dual-characterization}

\section{The Main Theorem}
\label{sec:main-theorem}

\section{Conclusion}
\label{sec:conclusion}


%% ============================================================
%% Appendices
%% ============================================================
\appendix
\part*{Appendices}
\addcontentsline{toc}{part}{Appendices}

% Appendix: Non-emptiness of S_M
% Proves that the set of M-positive states is non-empty

\chapter{Non-emptiness of \texorpdfstring{$S_M$}{S\_M}}
\label{app:nonemptiness}

This appendix provides a direct proof that the set $S_M$ of $M$-positive
states is non-empty. This is a crucial ingredient for the main theorem,
as the seminorm $\|\cdot\|_M$ is defined as a supremum over $S_M$.

\section{The Scalar Extraction Functional}

\begin{lemma}
\label{lem:SM-nonempty}
$S_M \neq \emptyset$.
\end{lemma}

\begin{proof}
Define $\varphi_0 : A_0 \to \C$ as the \emph{scalar extraction} functional:
for any element $a = \sum_w c_w w$ where $w$ ranges over words in the
generators $g_1, \ldots, g_n$, we set
\[
  \varphi_0(a) = c_\emptyset
\]
where $c_\emptyset$ is the coefficient of the empty word (i.e., the
scalar part of $a$).

We verify that $\varphi_0 \in S_M$:

\textbf{Step 1:} $\varphi_0(1) = 1$. This is immediate since the unit
element $1 \in A_0$ is exactly the empty word with coefficient 1.

\textbf{Step 2:} We show $\varphi_0(m) \geq 0$ for all $m \in M$.
Recall that elements of $M$ have the form
\[
  m = \sum_i a_i^* a_i + \sum_{j,k} b_{jk}^* g_j b_{jk}
\]
where $a_i, b_{jk} \in A_0$.

\textbf{Step 2a:} Consider the scalar part of $a_i^* a_i$ where
$a_i = \sum_\alpha c_\alpha w_\alpha$. The product expands as
\[
  a_i^* a_i = \sum_{\alpha,\beta} \overline{c_\alpha} c_\beta \,
              \widetilde{w_\alpha} w_\beta
\]
where $\widetilde{w_\alpha}$ denotes the reversed word with involution
applied. This contributes to the scalar part only when
$\widetilde{w_\alpha} w_\beta = \emptyset$, which happens exactly when
$w_\beta = w_\alpha$. Hence the scalar part of $a_i^* a_i$ is
\[
  \sum_\alpha |c_\alpha|^2 \geq 0.
\]

\textbf{Step 2b:} Consider the scalar part of $b_{jk}^* g_j b_{jk}$.
The product $b_{jk}^* g_j b_{jk}$ always contains at least one
generator $g_j$, so it never reduces to the empty word. Hence the
scalar part is $0$.

\textbf{Conclusion:} Combining Steps 2a and 2b, we have
\[
  \varphi_0(m) = \sum_i \sum_\alpha |c_\alpha^{(i)}|^2 \geq 0.
\]

Therefore $\varphi_0 \in S_M$, proving that $S_M \neq \emptyset$.
\end{proof}

\begin{remark}
The functional $\varphi_0$ is sometimes called the \emph{vacuum state}
or \emph{tracial state} in operator algebra contexts. It plays a
fundamental role in the structure theory of free algebras.
\end{remark}

\section{Lean Formalization}

The non-emptiness of $S_M$ is formalized in
\texttt{AfTests/ArchimedeanClosure/State/NonEmpty.lean}:

\begin{lstlisting}[language=Lean]
/-- The scalar extraction functional -/
def scalarExtraction : FreeStarAlgebra n ->* C :=
  FreeStarAlgebra.lift n (fun _ => 0)

/-- The scalar extraction is M-positive -/
theorem scalarExtraction_mPositive :
    MPositiveState (scalarExtraction (n := n)) := by
  constructor
  . simp [scalarExtraction]
  . intro m hm
    induction hm with
    | squares_sum => exact sum_nonneg (fun i _ => sq_nonneg _)
    | generator_terms => simp [scalarExtraction]
\end{lstlisting}

% Appendix: Lean Declaration Index
% Generated from: All .lean files in AfTests/ArchimedeanClosure/
% Status: PLACEHOLDER - to be filled by LaTeX-P4.1

\chapter{Lean Declaration Index}
\label{ch:lean-index}

% TODO: Fill in from LaTeX-P4.1 task
\textbf{[PLACEHOLDER: To be generated from Lean source]}

This appendix provides a complete index of all Lean declarations
in the formalization, organized by module.

\section{Definitions}
\label{sec:index-definitions}

\section{Theorems and Lemmas}
\label{sec:index-theorems}

\section{Instances}
\label{sec:index-instances}


%% ============================================================
%% Back Matter
%% ============================================================
\backmatter

\chapter*{Acknowledgments}
\addcontentsline{toc}{chapter}{Acknowledgments}

This formalization was developed using:
\begin{itemize}
  \item Lean 4 (version 4.26.0)
  \item Mathlib4 (the Lean 4 mathematical library)
  \item The GNS construction infrastructure from \texttt{AfTests/GNS/}
\end{itemize}

The proof relies on classical logic and the axiom of choice, as provided
by Mathlib's \texttt{Classical} namespace.

\end{document}
