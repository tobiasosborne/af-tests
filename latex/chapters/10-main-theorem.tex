% Auto-generated from Lean 4 source
% Source files: DualCharacterization.lean, Theorem.lean
% Do not edit manually

\chapter{Main Theorem}
\label{ch:main-theorem}

This chapter presents the culmination of the formalization: the main theorem
characterizing positivity in constrained $*$-representations. We first establish
the dual characterization relating membership in $\Mbar$ to nonnegativity under
all $\M$-positive states, then lift this to the representation-theoretic
statement that is the central result of this work.

The proof synthesizes results from all preceding chapters:
\begin{itemize}
  \item The algebraic framework of the free $*$-algebra and quadratic module
        (\Cref{ch:algebra})
  \item The theory of $\M$-positive states (\Cref{ch:states})
  \item Boundedness via Cauchy--Schwarz and the Archimedean property
        (\Cref{ch:boundedness})
  \item Compactness of the state space (\Cref{ch:topology})
  \item The state seminorm and closure characterization (\Cref{ch:seminorm})
  \item The dual characterization via Riesz extension (\Cref{ch:dual})
  \item The GNS construction and constrained representations
        (\Cref{ch:gns,ch:representations})
\end{itemize}

%% ============================================================
\section{Dual Characterization}
\label{sec:dual-characterization}

The dual characterization theorem provides the first key equivalence:
membership in the closure $\Mbar$ is equivalent to nonnegativity under
all $\M$-positive states.

\begin{theorem}[Dual Characterization]\label{thm:dual_characterization_final}
Let $A \in \Azero$ be self-adjoint. Then
\[
  A \in \Mbar \quad\Longleftrightarrow\quad
  \varphi(A) \ge 0 \text{ for all } \varphi \in \SM.
\]
\end{theorem}

\begin{proof}
We prove both directions separately.

\textbf{Forward direction ($\Rightarrow$):}
Suppose $A \in \Mbar$. By \Cref{thm:closure_implies_nonneg}, if $A$ belongs
to the closure of $\M$ in the state seminorm topology, then $\varphi(A) \ge 0$
for all $\M$-positive states $\varphi$. The proof proceeds by contradiction:
if $\varphi(A) < 0$ for some state, the seminorm bound $|\varphi(A - m)| \le \|A - m\|_{\M}$
together with $\M$-positivity of $\varphi$ leads to a contradiction with the
closure hypothesis.

\textbf{Backward direction ($\Leftarrow$):}
We prove the contrapositive: if $A \notin \Mbar$, then there exists
$\varphi \in \SM$ with $\varphi(A) < 0$.

Suppose $A \notin \Mbar$. By \Cref{thm:exists_MPositiveState_negative},
there exists an $\M$-positive state $\varphi$ with $\varphi(A) < 0$.
The construction proceeds through several steps developed in \Cref{ch:dual}:
\begin{enumerate}
  \item Apply geometric Hahn--Banach separation (\Cref{thm:riesz_extension_exists})
        to obtain a linear functional $\psi : \Azero \to \R$ with $\psi \ge 0$ on $\M$
        and $\psi(A) < 0$
  \item Symmetrize $\psi$ to obtain $\tilde{\psi}$ with $\tilde{\psi}(a^*) = \tilde{\psi}(a)$
        (\Cref{def:symmetrize})
  \item Show $\tilde{\psi}(1) > 0$ using Cauchy--Schwarz (\Cref{thm:phi_one_pos})
  \item Normalize to obtain $\varphi = \tilde{\psi}/\tilde{\psi}(1) \in \SM$
        (\Cref{def:normalizedMPositiveState})
\end{enumerate}
The resulting $\varphi$ satisfies $\varphi(A) < 0$.
\end{proof}

\begin{remark}[Lean Implementation]
In the Lean formalization, this theorem is \texttt{dual\_characterization}
in the namespace \texttt{FreeStarAlgebra}. The proof combines
\texttt{closure\_implies\_nonneg} (forward) with the contrapositive of
\texttt{exists\_MPositiveState\_negative} (backward).
\end{remark}

%% ============================================================
\section{The Main Theorem}
\label{sec:main-theorem}

We now state and prove the main theorem, which characterizes membership in
$\Mbar$ in terms of positivity in all constrained $*$-representations.

\begin{definition}[Positivity of Operators]\label{def:operator_positive}
A bounded operator $T \in \BH$ on a Hilbert space $\Hilbert$ is
\textbf{positive}, written $T \ge 0$ or $T.\mathsf{IsPositive}$, if
\[
  \langle T\xi, \xi \rangle \ge 0 \quad \text{for all } \xi \in \Hilbert.
\]
\end{definition}

\begin{theorem}[Main Theorem]\label{thm:main_theorem_final}
Let $A \in \Azero$ be self-adjoint. Then
\[
  A \in \Mbar \quad\Longleftrightarrow\quad
  \Rep(A) \ge 0 \text{ for all constrained $*$-representations } \Rep.
\]
\end{theorem}

The proof chains together three equivalences, which we develop in turn.

\subsection{Forward Direction: Closure Implies Representation Positivity}

\begin{proposition}\label{prop:closure_to_rep}
If $A \in \Mbar$, then $\Rep(A) \ge 0$ for all constrained $*$-representations $\Rep$.
\end{proposition}

\begin{proof}
Assume $A \in \Mbar$. We must show $\Rep(A) \ge 0$ for an arbitrary
constrained representation $\Rep : \Azero \to \BH$.

By the dual characterization (\Cref{thm:dual_characterization_final}),
$A \in \Mbar$ implies $\varphi(A) \ge 0$ for all $\varphi \in \SM$.

Now let $\Rep$ be any constrained representation and $\xi \in \Hilbert$
a unit vector. The vector state $\varphi_\xi(a) = \langle \Rep(a)\xi, \xi \rangle$
is an $\M$-positive state (\Cref{sec:vector-state}), since:
\begin{itemize}
  \item For squares: $\varphi_\xi(a^* a) = \|\Rep(a)\xi\|^2 \ge 0$
  \item For generator-weighted squares: $\varphi_\xi(b^* \gen{j} b) =
        \langle \Rep(\gen{j})\Rep(b)\xi, \Rep(b)\xi \rangle \ge 0$
        because $\Rep(\gen{j}) \ge 0$ (the constraint condition)
\end{itemize}

Since $\varphi_\xi \in \SM$, we have $\varphi_\xi(A) \ge 0$, i.e.,
$\langle \Rep(A)\xi, \xi \rangle \ge 0$. Since $\xi$ was arbitrary,
$\Rep(A) \ge 0$.
\end{proof}

\subsection{Backward Direction: Representation Positivity Implies Closure}

\begin{proposition}\label{prop:rep_to_closure}
If $\Rep(A) \ge 0$ for all constrained $*$-representations $\Rep$,
then $A \in \Mbar$.
\end{proposition}

\begin{proof}
We prove the contrapositive. Assume $A \notin \Mbar$.

By the dual characterization (\Cref{thm:dual_characterization_final}),
there exists $\varphi \in \SM$ with $\varphi(A) < 0$.

Apply the GNS construction (\Cref{ch:gns}) to $\varphi$ to obtain a
$*$-representation $\Rep_\varphi : \Azero \to \mathcal{B}(\HilbertPhi)$
with cyclic vector $\Omega_\varphi$ such that
\[
  \varphi(a) = \langle \Rep_\varphi(a)\Omega_\varphi, \Omega_\varphi \rangle.
\]

By \Cref{sec:gns-constrained}, the GNS representation $\Rep_\varphi$
is constrained: for each generator $\gen{j}$, we have $\Rep_\varphi(\gen{j}) \ge 0$.
This follows because for any $\xi$ in the dense subspace,
\[
  \langle \Rep_\varphi(\gen{j})\xi, \xi \rangle =
  \varphi(b^* \gen{j} b) \ge 0
\]
where $\xi = \Rep_\varphi(b)\Omega_\varphi$, using that $\varphi$ is
$\M$-positive and $b^* \gen{j} b \in \M$.

Now we have a constrained representation $\Rep_\varphi$ with
\[
  \langle \Rep_\varphi(A)\Omega_\varphi, \Omega_\varphi \rangle =
  \varphi(A) < 0,
\]
so $\Rep_\varphi(A)$ is not positive.
\end{proof}

\subsection{Proof of the Main Theorem}

\begin{proof}[Proof of \Cref{thm:main_theorem_final}]
Combine \Cref{prop:closure_to_rep} and \Cref{prop:rep_to_closure}:
\begin{align*}
  A \in \Mbar
  &\Longleftrightarrow \varphi(A) \ge 0 \text{ for all } \varphi \in \SM
    &&\text{(\Cref{thm:dual_characterization_final})} \\
  &\Longrightarrow \Rep(A) \ge 0 \text{ for all constrained } \Rep
    &&\text{(\Cref{prop:closure_to_rep})} \\
  &\Longrightarrow \varphi(A) \ge 0 \text{ for all } \varphi \in \SM
    &&\text{(GNS gives constrained rep)} \\
  &\Longleftrightarrow A \in \Mbar
    &&\text{(\Cref{thm:dual_characterization_final})}
\end{align*}
The middle implications establish the equivalence with representation positivity.
\end{proof}

\begin{remark}[Lean Implementation]
The main theorem is \texttt{main\_theorem} in the namespace
\texttt{ArchimedeanClosure}. The forward direction uses
\texttt{state\_nonneg\_implies\_rep\_positive}, and the backward
direction uses \texttt{gns\_constrained\_implies\_state\_nonneg}.
The proof is approximately 15 lines, reflecting how the substantial
work was done in establishing the component results.
\end{remark}

\begin{remark}[Universe Level]
In the Lean formalization, the main theorem is stated with universe
level 0 for the Hilbert space type. This matches the universe level
used in the GNS construction and ensures compatibility between the
state-derived representations and abstract constrained representations.
\end{remark}

%% ============================================================
\section{Conclusion}
\label{sec:conclusion}

\subsection{Summary of Results}

We have formally verified the following main result:

\begin{theorem*}[Main Theorem, Restated]
For a self-adjoint element $A$ in the free $*$-algebra $\Azero$ with
Archimedean quadratic module $\M$:
\[
  A \in \Mbar \quad\Longleftrightarrow\quad
  \Rep(A) \ge 0 \text{ for all constrained $*$-representations } \Rep.
\]
\end{theorem*}

This theorem provides a dual characterization of the closure of a
quadratic module: membership can be detected either topologically
(as a limit of elements in $\M$) or representation-theoretically
(as positivity in all constrained representations).

\subsection{Key Components}

The proof assembled results from eight phases of development:

\begin{enumerate}
  \item \textbf{Algebraic Setup} (\Cref{ch:algebra}):
        Construction of the free $*$-algebra with self-adjoint generators,
        definition of the quadratic module $\M$, and the Archimedean property.

  \item \textbf{States} (\Cref{ch:states}):
        Definition of $\M$-positive states and proof of non-emptiness via
        the scalar extraction functional.

  \item \textbf{Boundedness} (\Cref{ch:boundedness}):
        Cauchy--Schwarz inequality for states and uniform boundedness from
        the Archimedean property.

  \item \textbf{Topology} (\Cref{ch:topology}):
        Weak-$*$ topology on the dual space and compactness of $\SM$ via
        the Tychonoff theorem.

  \item \textbf{Seminorm} (\Cref{ch:seminorm}):
        State seminorm $\|\cdot\|_{\M}$ and characterization of the closure
        $\Mbar$ in this topology.

  \item \textbf{Dual Characterization} (\Cref{ch:dual}):
        The equivalence $A \in \Mbar \Leftrightarrow \varphi(A) \ge 0$
        for all $\varphi \in \SM$, using geometric Hahn--Banach separation.

  \item \textbf{GNS Construction} (\Cref{ch:gns}):
        The Gelfand--Naimark--Segal construction producing a Hilbert space
        representation from any state.

  \item \textbf{Representations} (\Cref{ch:representations}):
        Constrained $*$-representations, vector states, and the proof that
        GNS representations are constrained.
\end{enumerate}

\subsection{Significance}

The main theorem has several important consequences:

\begin{itemize}
  \item \textbf{Certification of Positivity}: To verify that $A \in \Mbar$,
        it suffices to check positivity in constrained representations.
        This is often more tractable than directly verifying topological closure.

  \item \textbf{Duality}: The theorem establishes a precise duality between
        the algebraic-topological notion of closure and the
        representation-theoretic notion of positivity.

  \item \textbf{Separation}: Conversely, if $A \notin \Mbar$, the theorem
        guarantees the existence of a ``witness'' representation where
        $\Rep(A)$ fails to be positive.
\end{itemize}

\subsection{Formalization Statistics}

The complete Lean formalization comprises:
\begin{itemize}
  \item Approximately 965 lines of Lean code
  \item 26 source files organized by proof phase
  \item Zero remaining \texttt{sorry} statements
  \item Full type-checked verification in Lean~4
\end{itemize}

The main theorem itself is a concise 15-line proof, demonstrating how
the careful development of infrastructure enables clean statement of
the final result. The bulk of the work lies in the supporting theory:
the GNS construction, the dual characterization via Riesz extension,
and the compactness arguments for the state space.

