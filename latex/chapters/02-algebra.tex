% Auto-generated from Lean 4 source
% Source files: FreeStarAlgebra.lean, QuadraticModule.lean, Archimedean.lean
% Do not edit manually

\chapter{Algebraic Setup}
\label{ch:algebra}

This chapter establishes the algebraic foundations for the characterization of positivity
in constrained $C^*$-algebra representations. We define the free $*$-algebra on self-adjoint
generators, the quadratic module capturing the constraint structure, and the Archimedean
property that ensures bounded behavior.

%% ============================================================
\section{The Free $*$-Algebra}
\label{sec:free-star-algebra}

We begin by constructing the free $*$-algebra on $n$ self-adjoint generators over the
real numbers. The choice of $\mathbb{R}$ as the base field is crucial: it ensures that
for any scalar $c$, we have $c^* \cdot c = c^2 \geq 0$, which is essential for obtaining
$M$-positive states via scalar extraction.

\begin{definition}[Free $*$-Algebra]\label{def:FreeStarAlgebra}
The \textbf{free $*$-algebra} $\mathcal{A}_0$ on $n$ self-adjoint generators is defined as
\[
  \mathcal{A}_0 = \mathbb{R}\langle g_1, \ldots, g_n \rangle
\]
the free algebra over $\mathbb{R}$ on generators indexed by $\mathrm{Fin}(n)$, equipped
with the canonical $*$-structure from mathlib's \texttt{Mathlib.Algebra.Star.Free}.
\end{definition}

\begin{definition}[Generator]\label{def:generator}
For $j \in \mathrm{Fin}(n)$, the \textbf{$j$-th generator} $g_j \in \mathcal{A}_0$ is the
canonical embedding of the index $j$ into the free algebra:
\[
  g_j = \iota(j)
\]
where $\iota : \mathrm{Fin}(n) \to \mathcal{A}_0$ is the universal map.
\end{definition}

\begin{theorem}[Generators are Self-Adjoint]\label{thm:isSelfAdjoint_generator}
For each $j \in \mathrm{Fin}(n)$, the generator $g_j$ is self-adjoint:
\[
  g_j^* = g_j.
\]
\end{theorem}

\begin{proof}
This follows directly from the star structure on free algebras defined in
\texttt{Mathlib.Algebra.Star.Free}, which specifies that generators are fixed by
the involution.
\end{proof}

\begin{lemma}[Unit is Self-Adjoint]\label{lem:one_isSelfAdjoint}
The unit element $1 \in \mathcal{A}_0$ is self-adjoint: $1^* = 1$.
\end{lemma}

\begin{proof}
Standard property of $*$-algebras.
\end{proof}

%% ============================================================
\section{Quadratic Modules}
\label{sec:quadratic-module}

The quadratic module $M$ captures the positivity constraints imposed by the generators.
It consists of sums of squares and generator-weighted squares, closed under addition
and nonnegative real scaling.

\begin{definition}[Square Set]\label{def:squareSet}
The \textbf{square set} is defined as
\[
  S = \{ a^* a : a \in \mathcal{A}_0 \}.
\]
\end{definition}

\begin{definition}[Generator-Weighted Set]\label{def:generatorWeightedSet}
The \textbf{generator-weighted set} is defined as
\[
  W = \{ b^* g_j b : j \in \mathrm{Fin}(n),\ b \in \mathcal{A}_0 \}.
\]
\end{definition}

\begin{definition}[Quadratic Module Generators]\label{def:QuadraticModuleGenerators}
The \textbf{generating set} for the quadratic module is
\[
  G = S \cup W = \{ a^* a : a \in \mathcal{A}_0 \} \cup \{ b^* g_j b : j \in \mathrm{Fin}(n),\ b \in \mathcal{A}_0 \}.
\]
\end{definition}

\begin{definition}[Quadratic Module]\label{def:QuadraticModule}
The \textbf{quadratic module} $M \subseteq \mathcal{A}_0$ is the smallest set containing $G$
and closed under:
\begin{enumerate}
  \item Addition: if $m_1, m_2 \in M$, then $m_1 + m_2 \in M$
  \item Nonnegative scaling: if $c \geq 0$ and $m \in M$, then $c \cdot m \in M$
\end{enumerate}
Equivalently,
\[
  M = \left\{ \sum_{i=1}^k c_i a_i^* a_i + \sum_{j,\ell} d_{j\ell} b_{j\ell}^* g_j b_{j\ell}
      : c_i, d_{j\ell} \geq 0,\ a_i, b_{j\ell} \in \mathcal{A}_0 \right\}.
\]
\end{definition}

\begin{lemma}[Squares in $M$]\label{lem:star_mul_self_mem}
For any $a \in \mathcal{A}_0$, we have $a^* a \in M$.
\end{lemma}

\begin{proof}
By definition, $a^* a \in S \subseteq G \subseteq M$.
\end{proof}

\begin{lemma}[Generator-Weighted Elements in $M$]\label{lem:star_generator_mul_mem}
For any $j \in \mathrm{Fin}(n)$ and $b \in \mathcal{A}_0$, we have $b^* g_j b \in M$.
\end{lemma}

\begin{proof}
By definition, $b^* g_j b \in W \subseteq G \subseteq M$.
\end{proof}

\begin{theorem}[Conjugation Closure]\label{thm:star_mul_mem_star_mul}
If $m \in M$ and $b \in \mathcal{A}_0$, then $b^* m b \in M$.
\end{theorem}

\begin{proof}[Proof sketch]
We proceed by structural induction on the definition of $M$:
\begin{itemize}
  \item \textbf{Generator case ($m = a^* a$):}
    $b^* (a^* a) b = (ab)^* (ab) \in M$ by Lemma~\ref{lem:star_mul_self_mem}.
  \item \textbf{Generator case ($m = c^* g_j c$):}
    $b^* (c^* g_j c) b = (cb)^* g_j (cb) \in M$ by Lemma~\ref{lem:star_generator_mul_mem}.
  \item \textbf{Addition case ($m = m_1 + m_2$):}
    $b^* (m_1 + m_2) b = b^* m_1 b + b^* m_2 b \in M$ by the induction hypothesis and closure.
  \item \textbf{Scaling case ($m = r \cdot m'$):}
    $b^* (r \cdot m') b = r \cdot (b^* m' b) \in M$ by the induction hypothesis and closure.
\end{itemize}
\end{proof}

%% ============================================================
\section{The Archimedean Property}
\label{sec:archimedean}

The Archimedean property ensures that every element of the algebra is ``bounded''
relative to the quadratic module. This is essential for proving that $M$-positive
states yield finite values.

\begin{definition}[Archimedean Property]\label{def:IsArchimedean}
The quadratic module $M$ is \textbf{Archimedean} if for every $a \in \mathcal{A}_0$,
there exists $N \in \mathbb{N}$ such that
\[
  N \cdot 1 - a^* a \in M.
\]
\end{definition}

This condition says that $a^* a$ is ``dominated'' by $N \cdot 1$ in the sense of the
quadratic module: the difference belongs to $M$ and hence will be evaluated nonnegatively
by any $M$-positive state.

\begin{definition}[Archimedean Bound]\label{def:archimedeanBound}
For an Archimedean quadratic module and $a \in \mathcal{A}_0$, the \textbf{Archimedean bound}
$N(a)$ is (a choice of) a natural number such that
\[
  N(a) \cdot 1 - a^* a \in M.
\]
\end{definition}

\begin{theorem}[Archimedean Bound Specification]\label{thm:archimedeanBound_spec}
If $M$ is Archimedean, then for any $a \in \mathcal{A}_0$, the Archimedean bound $N(a)$
satisfies
\[
  N(a) \cdot 1 - a^* a \in M.
\]
\end{theorem}

\begin{proof}
By definition of the Archimedean property and the axiom of choice.
\end{proof}

\begin{remark}
The Archimedean property has the following important consequence: for any $M$-positive
state $\varphi$ and any $a \in \mathcal{A}_0$,
\[
  \varphi(a^* a) \leq N(a) \cdot \varphi(1) = N(a).
\]
Combined with the Cauchy--Schwarz inequality (see Chapter~\ref{ch:boundedness}), this
implies that $|\varphi(a)|^2 \leq N(a)$, so $M$-positive states are uniformly bounded
on each element.
\end{remark}

%% ============================================================
\section{Summary}
\label{sec:algebra-summary}

We have established:
\begin{itemize}
  \item The free $*$-algebra $\mathcal{A}_0 = \mathbb{R}\langle g_1, \ldots, g_n \rangle$
        with self-adjoint generators (Definition~\ref{def:FreeStarAlgebra})
  \item The quadratic module $M$ generated by squares and generator-weighted squares
        (Definition~\ref{def:QuadraticModule})
  \item The conjugation closure property $m \in M \Rightarrow b^* m b \in M$
        (Theorem~\ref{thm:star_mul_mem_star_mul})
  \item The Archimedean property ensuring bounded behavior
        (Definition~\ref{def:IsArchimedean})
\end{itemize}

These structures form the algebraic foundation for the state space and seminorm
constructions in subsequent chapters.
