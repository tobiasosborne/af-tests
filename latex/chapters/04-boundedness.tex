% Auto-generated from Lean 4 source
% Source files: ArchimedeanBound.lean, CauchySchwarzM.lean, GeneratingCone.lean
% Do not edit manually

% Chapter: Boundedness of M-Positive States
% Generated from: AfTests/ArchimedeanClosure/Boundedness/

\chapter{Boundedness of M-Positive States}
\label{ch:boundedness}

This chapter establishes fundamental boundedness properties of $M$-positive states.
The key results are the Cauchy--Schwarz inequality for the sesquilinear form
induced by states, and the uniform bound on state values derived from the
Archimedean property. We also prove that $M \cap (\mathcal{A}_0)_{\mathrm{sa}}$
generates the self-adjoint part as differences.

The main results are:
\begin{itemize}
  \item \texttt{cauchy\_schwarz} (Theorem~\ref{thm:cauchy-schwarz}): the Cauchy--Schwarz inequality
  \item \texttt{apply\_abs\_le} (Theorem~\ref{thm:apply-abs-le}): uniform bound $|\varphi(a)| \le \sqrt{N_a}$
  \item \texttt{quadraticModule\_selfAdjoint\_generating} (Theorem~\ref{thm:generating}): generating property
\end{itemize}

%% ============================================================
\section{Cauchy--Schwarz Inequality}
\label{sec:cauchy-schwarz}

We prove the Cauchy--Schwarz inequality for $M$-positive states using the
classical discriminant argument for real quadratics.

\begin{lemma}[Symmetry of Sesquilinear Form]\label{lem:sesqForm-symm}
For any $M$-positive state $\varphi$ and $a, b \in \mathcal{A}_0$,
\[
  \varphi(a^* b) = \varphi(b^* a).
\]
\end{lemma}

\begin{proof}
Apply the symmetry property $\varphi(x^*) = \varphi(x)$ to $x = a^* b$.
Since $(a^* b)^* = b^* a$, we have $\varphi(a^* b) = \varphi((a^* b)^*) = \varphi(b^* a)$.
\end{proof}

\begin{theorem}[Cauchy--Schwarz]\label{thm:cauchy-schwarz}
For any $M$-positive state $\varphi$ and $a, b \in \mathcal{A}_0$,
\[
  \varphi(b^* a)^2 \le \varphi(a^* a) \cdot \varphi(b^* b).
\]
\end{theorem}

\begin{proof}[Proof sketch]
For $t \in \mathbb{R}$, consider the element $(a + tb)^*(a + tb) \in M$.
By $M$-positivity,
\[
  q(t) := \varphi((a + tb)^*(a + tb)) \ge 0.
\]
Expanding using linearity and Lemma~\ref{lem:sesqForm-symm}:
\[
  q(t) = \varphi(a^* a) + 2t\,\varphi(b^* a) + t^2\,\varphi(b^* b).
\]
This is a nonnegative quadratic in $t$, so its discriminant satisfies
\[
  (2\varphi(b^* a))^2 - 4\,\varphi(a^* a)\,\varphi(b^* b) \le 0,
\]
which yields the result.
\end{proof}

\begin{corollary}\label{cor:apply-sq-le}
For any $M$-positive state $\varphi$ and $a \in \mathcal{A}_0$,
\[
  \varphi(a)^2 \le \varphi(a^* a).
\]
\end{corollary}

\begin{proof}
Apply Theorem~\ref{thm:cauchy-schwarz} with $b = 1$ and use $\varphi(1) = 1$.
\end{proof}

\begin{lemma}[Null Space Property]\label{lem:null-space}
If $\varphi(a^* a) = 0$, then $\varphi(b^* a) = 0$ for all $b \in \mathcal{A}_0$.
\end{lemma}

\begin{proof}
By Theorem~\ref{thm:cauchy-schwarz}, $\varphi(b^* a)^2 \le \varphi(a^* a) \cdot \varphi(b^* b) = 0$.
\end{proof}

%% ============================================================
\section{Archimedean Bound}
\label{sec:archimedean-bound}

Using the Archimedean property of the quadratic module $M$, we derive uniform
bounds on state values. Recall that the Archimedean property states: for each
$a \in \mathcal{A}_0$, there exists $N_a \in \mathbb{N}$ such that
$N_a \cdot 1 - a^* a \in M$.

\begin{theorem}[Bound on $\varphi(a^* a)$]\label{thm:star-mul-self-bound}
Let $\varphi$ be an $M$-positive state and $a \in \mathcal{A}_0$. Then
\[
  \varphi(a^* a) \le N_a,
\]
where $N_a$ is the Archimedean bound for $a$.
\end{theorem}

\begin{proof}[Proof sketch]
By the Archimedean property, $N_a \cdot 1 - a^* a \in M$.
By $M$-positivity:
\[
  \varphi(N_a \cdot 1 - a^* a) \ge 0.
\]
Using linearity and $\varphi(1) = 1$:
\[
  N_a - \varphi(a^* a) \ge 0,
\]
which gives $\varphi(a^* a) \le N_a$.
\end{proof}

\begin{theorem}[Combined Bound]\label{thm:apply-bound}
For any $M$-positive state $\varphi$ and $a \in \mathcal{A}_0$,
\[
  \varphi(a)^2 \le N_a.
\]
\end{theorem}

\begin{proof}
By Corollary~\ref{cor:apply-sq-le}, $\varphi(a)^2 \le \varphi(a^* a)$.
By Theorem~\ref{thm:star-mul-self-bound}, $\varphi(a^* a) \le N_a$.
\end{proof}

\begin{theorem}[Uniform Absolute Bound]\label{thm:apply-abs-le}
For any $M$-positive state $\varphi$ and $a \in \mathcal{A}_0$,
\[
  |\varphi(a)| \le \sqrt{N_a}.
\]
\end{theorem}

\begin{proof}
From Theorem~\ref{thm:apply-bound}, $\varphi(a)^2 \le N_a$.
Taking square roots (noting $N_a \ge 0$) gives the result.
\end{proof}

\begin{remark}
Theorem~\ref{thm:apply-abs-le} is the key estimate needed for compactness
of the state space $S_M$ (Chapter~\ref{ch:compactness}). It shows that for
each $a$, the set $\{\varphi(a) : \varphi \in S_M\}$ is bounded.
\end{remark}

%% ============================================================
\section{Generating Cone Property}
\label{sec:generating-cone}

We prove that $M \cap (\mathcal{A}_0)_{\mathrm{sa}}$ generates the self-adjoint
part as differences. This uses the algebraic identity expressing any self-adjoint
element as a difference of two squares.

\begin{definition}[Self-Adjoint Part]\label{def:selfAdjointPart}
The \textbf{self-adjoint part} of $\mathcal{A}_0$ is
\[
  (\mathcal{A}_0)_{\mathrm{sa}} = \{a \in \mathcal{A}_0 : a^* = a\}.
\]
\end{definition}

\begin{lemma}\label{lem:one-pm-sa}
If $x \in (\mathcal{A}_0)_{\mathrm{sa}}$, then $1 + x$ and $1 - x$ are self-adjoint.
\end{lemma}

\begin{proof}
$(1 + x)^* = 1^* + x^* = 1 + x$, and similarly for $1 - x$.
\end{proof}

\begin{lemma}\label{lem:sq-sa}
If $a \in (\mathcal{A}_0)_{\mathrm{sa}}$, then $a^2 \in (\mathcal{A}_0)_{\mathrm{sa}}$.
\end{lemma}

\begin{proof}
$(a^2)^* = (a \cdot a)^* = a^* \cdot a^* = a \cdot a = a^2$.
\end{proof}

\begin{theorem}[Algebraic Decomposition]\label{thm:selfAdjoint-decomp}
For any self-adjoint $x \in (\mathcal{A}_0)_{\mathrm{sa}}$,
\[
  x = \frac{1}{4}(1+x)^*(1+x) - \frac{1}{4}(1-x)^*(1-x).
\]
\end{theorem}

\begin{proof}[Proof sketch]
By Lemma~\ref{lem:one-pm-sa}, $(1+x)^* = 1+x$ and $(1-x)^* = 1-x$. Thus:
\begin{align*}
  (1+x)^2 &= 1 + 2x + x^2, \\
  (1-x)^2 &= 1 - 2x + x^2.
\end{align*}
Subtracting: $(1+x)^2 - (1-x)^2 = 4x$, so $x = \tfrac{1}{4}(1+x)^2 - \tfrac{1}{4}(1-x)^2$.
\end{proof}

\begin{lemma}[Decomposition Terms in $M$]\label{lem:decomp-in-M}
For any $x \in \mathcal{A}_0$, both $(1+x)^*(1+x)$ and $(1-x)^*(1-x)$ lie in $M$.
\end{lemma}

\begin{proof}
These are sums of squares $a^* a$ with $a = 1 + x$ and $a = 1 - x$ respectively,
which belong to $M$ by definition of the quadratic module.
\end{proof}

\begin{theorem}[Generating Property]\label{thm:generating}
The cone $M \cap (\mathcal{A}_0)_{\mathrm{sa}}$ generates $(\mathcal{A}_0)_{\mathrm{sa}}$
as differences. That is, for every $x \in (\mathcal{A}_0)_{\mathrm{sa}}$, there exist
$m_1, m_2 \in M \cap (\mathcal{A}_0)_{\mathrm{sa}}$ such that
\[
  x = \tfrac{1}{4} m_1 - \tfrac{1}{4} m_2.
\]
\end{theorem}

\begin{proof}
Let $x \in (\mathcal{A}_0)_{\mathrm{sa}}$. Define:
\begin{align*}
  m_1 &= (1+x)^*(1+x), \\
  m_2 &= (1-x)^*(1-x).
\end{align*}
By Lemma~\ref{lem:decomp-in-M}, $m_1, m_2 \in M$.
By Lemmas~\ref{lem:one-pm-sa} and~\ref{lem:sq-sa}, $m_1, m_2 \in (\mathcal{A}_0)_{\mathrm{sa}}$.
By Theorem~\ref{thm:selfAdjoint-decomp}, $x = \tfrac{1}{4}m_1 - \tfrac{1}{4}m_2$.
\end{proof}

\begin{remark}
Theorem~\ref{thm:generating} shows that $M \cap (\mathcal{A}_0)_{\mathrm{sa}}$ is a
\emph{generating cone} for the real vector space $(\mathcal{A}_0)_{\mathrm{sa}}$.
This property is essential for applying the Riesz extension theorem in
Chapter~\ref{ch:dual}.
\end{remark}
