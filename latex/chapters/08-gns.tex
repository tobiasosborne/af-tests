% Auto-generated from Lean 4 source
% Source files: AfTests/GNS/NullSpace/*.lean, AfTests/GNS/Representation/*.lean,
%               AfTests/GNS/PreHilbert/*.lean, AfTests/GNS/HilbertSpace/*.lean,
%               AfTests/GNS/State/*.lean, AfTests/GNS/Main/*.lean
% Do not edit manually

\chapter{The GNS Construction}
\label{ch:gns}

The Gelfand-Naimark-Segal (GNS) construction is one of the fundamental theorems in the theory of $C^*$-algebras. It shows that every state on a $C^*$-algebra can be realized as a vector state in some Hilbert space representation. This chapter presents the complete formalization of the GNS construction, including both the existence theorem and the uniqueness theorem up to unitary equivalence.

Given a state $\varphi$ on a $C^*$-algebra $A$, we construct:
\begin{enumerate}
    \item A Hilbert space $H_\varphi$ (the GNS Hilbert space)
    \item A $*$-representation $\pi_\varphi : A \to B(H_\varphi)$
    \item A cyclic unit vector $\Omega_\varphi \in H_\varphi$
\end{enumerate}
such that $\varphi(a) = \langle \Omega_\varphi, \pi_\varphi(a) \Omega_\varphi \rangle$ for all $a \in A$.

%----------------------------------------------------------------------
\section{States on $C^*$-Algebras}
\label{sec:states}

\begin{definition}[State]
\label{def:state}
A \emph{state} on a $C^*$-algebra $A$ is a continuous linear functional $\varphi : A \to \mathbb{C}$ satisfying:
\begin{enumerate}
    \item \emph{Positivity}: $\varphi(a^* a) \geq 0$ for all $a \in A$ (with $\varphi(a^* a) \in \mathbb{R}$)
    \item \emph{Normalization}: $\varphi(1) = 1$
\end{enumerate}
\end{definition}

The Lean formalization captures both components of positivity explicitly.

\begin{lstlisting}[language=lean]
structure State (A : Type*) [CStarAlgebra A] where
  toContinuousLinearMap : A ->L[C] C
  map_star_mul_self_nonneg : forall a : A, 0 <= (toContinuousLinearMap (star a * a)).re
  map_star_mul_self_real : forall a : A, (toContinuousLinearMap (star a * a)).im = 0
  map_one : toContinuousLinearMap 1 = 1
\end{lstlisting}

\begin{theorem}[Star Preservation]
\label{thm:state-map-star}
States preserve the star operation: $\varphi(a^*) = \overline{\varphi(a)}$ for all $a \in A$.
\end{theorem}

\begin{proof}
The proof uses a polarization identity. Define the sesquilinear form $\langle a, b \rangle_\varphi = \varphi(b^* a)$. Since $\varphi(z^* z) \in \mathbb{R}$ for all $z$ (part of the state axioms), the polarization identity shows that this form satisfies conjugate symmetry: $\langle a, b \rangle_\varphi = \overline{\langle b, a \rangle_\varphi}$. Setting $b = 1$ gives $\varphi(a) = \overline{\varphi(a^*)}$, hence $\varphi(a^*) = \overline{\varphi(a)}$.

The Lean proof uses \texttt{sesqForm\_conj\_symm} and analyzes real and imaginary parts separately.
\end{proof}

%----------------------------------------------------------------------
\section{The Cauchy-Schwarz Inequality}
\label{sec:cauchy-schwarz}

The Cauchy-Schwarz inequality for states is crucial for proving that the null space is closed under addition.

\begin{theorem}[Cauchy-Schwarz for States]
\label{thm:cauchy-schwarz}
For any state $\varphi$ on a $C^*$-algebra $A$ and any $a, b \in A$:
\[
|\varphi(b^* a)|^2 \leq \varphi(a^* a) \cdot \varphi(b^* b)
\]
\end{theorem}

\begin{proof}
The proof proceeds in two stages.

\emph{Weak form}: First, we establish a version with an extra factor of 2. For $t \in \mathbb{R}$, the positivity of $\varphi((a + t \cdot b)^*(a + t \cdot b)) \geq 0$ gives a quadratic in $t$:
\[
\varphi(b^* b) \cdot t^2 + 2 \operatorname{Re}(\varphi(b^* a)) \cdot t + \varphi(a^* a) \geq 0
\]
By the discriminant lemma (\texttt{discrim\_le\_zero}), we get $\operatorname{Re}(\varphi(b^* a))^2 \leq \varphi(a^* a) \cdot \varphi(b^* b)$. Applying the same argument to $(a + it \cdot b)$ yields the same bound for the imaginary part, giving $|\varphi(b^* a)|^2 \leq 2 \cdot \varphi(a^* a) \cdot \varphi(b^* b)$.

\emph{Tight form}: For the sharp bound, we optimize over complex $\mu$. If $\varphi(b^* b) = 0$, the weak form immediately gives $|\varphi(b^* a)|^2 = 0$. Otherwise, set $\mu = -\varphi(b^* a) / \varphi(b^* b)$. The positivity of $\varphi((a + \mu b)^*(a + \mu b)) \geq 0$ expands to:
\[
\varphi(a^* a) - \frac{|\varphi(b^* a)|^2}{\varphi(b^* b)} \geq 0
\]
which gives the tight bound. The algebraic manipulation uses \texttt{cross\_term\_opt\_identity}.
\end{proof}

\begin{corollary}
\label{cor:null-consequence}
If $\varphi(a^* a) = 0$, then $\varphi(b^* a) = 0$ for all $b \in A$.
\end{corollary}

%----------------------------------------------------------------------
\section{The Null Space}
\label{sec:null-space}

\begin{definition}[GNS Null Space]
\label{def:null-space}
The \emph{GNS null space} $N_\varphi$ is defined as:
\[
N_\varphi = \{a \in A : \varphi(a^* a) = 0\}
\]
\end{definition}

The Lean formalization defines this as an \texttt{AddSubgroup}:
\begin{lstlisting}[language=lean]
def gnsNullSpace : AddSubgroup A where
  carrier := {a : A | phi (star a * a) = 0}
  zero_mem' := by simp [star_zero, map_zero]
  add_mem' := ...  -- uses Cauchy-Schwarz
  neg_mem' := ...
\end{lstlisting}

\begin{theorem}[Null Space is an Additive Subgroup]
\label{thm:null-subgroup}
$N_\varphi$ is closed under $0$, addition, negation, and scalar multiplication.
\end{theorem}

\begin{proof}
Closure under $0$ and negation is straightforward. For addition: if $a, b \in N_\varphi$, we expand
\[
\varphi((a+b)^*(a+b)) = \varphi(a^* a) + \varphi(a^* b) + \varphi(b^* a) + \varphi(b^* b)
\]
By Corollary~\ref{cor:null-consequence}, $\varphi(a^* a) = 0$ implies $\varphi(x^* a) = 0$ for all $x$, and similarly for $b$. Thus all four terms vanish.
\end{proof}

\begin{theorem}[Null Space is a Left Ideal]
\label{thm:null-left-ideal}
If $a \in N_\varphi$, then $ba \in N_\varphi$ for all $b \in A$.
\end{theorem}

\begin{proof}
We need $\varphi((ba)^*(ba)) = 0$. Computing $(ba)^*(ba) = a^* b^* b a$, we have
\[
\varphi(a^* \cdot (b^* b a)) = 0
\]
by the ``swapped'' Cauchy-Schwarz: if $\varphi(a^* a) = 0$, then $\varphi(a^* \cdot x) = 0$ for all $x$. This is \texttt{apply\_mul\_star\_eq\_zero\_of\_apply\_star\_self\_eq\_zero}.
\end{proof}

%----------------------------------------------------------------------
\section{Quotient Construction}
\label{sec:quotient}

The quotient $A / N_\varphi$ is the pre-Hilbert space on which we will define the inner product.

\begin{definition}[GNS Quotient]
\label{def:quotient}
The \emph{GNS quotient space} is $A / N_\varphi$, where $N_\varphi$ is viewed as a $\mathbb{C}$-submodule of $A$.
\end{definition}

\begin{lstlisting}[language=lean]
def gnsNullIdeal : Submodule C A where
  carrier := {a : A | phi (star a * a) = 0}
  add_mem' := fun {_ _} ha hb => phi.gnsNullSpace.add_mem ha hb
  zero_mem' := phi.gnsNullSpace.zero_mem
  smul_mem' := fun c {_} ha => gnsNullSpace_smul_mem phi ha c

abbrev gnsQuotient := A / phi.gnsNullIdeal
\end{lstlisting}

%----------------------------------------------------------------------
\section{Pre-Representation}
\label{sec:pre-rep}

\begin{definition}[Pre-Representation]
\label{def:pre-rep}
The \emph{GNS pre-representation} $\pi_\varphi(a) : A/N_\varphi \to A/N_\varphi$ is defined by left multiplication:
\[
\pi_\varphi(a)[b] = [ab]
\]
\end{definition}

\begin{theorem}
\label{thm:pre-rep-well-defined}
The pre-representation is well-defined (since $N_\varphi$ is a left ideal) and satisfies:
\begin{enumerate}
    \item $\pi_\varphi(ab) = \pi_\varphi(a) \circ \pi_\varphi(b)$ (multiplicative)
    \item $\pi_\varphi(1) = \mathrm{id}$
    \item $\pi_\varphi(a+b) = \pi_\varphi(a) + \pi_\varphi(b)$ (additive)
    \item $\pi_\varphi(c \cdot a) = c \cdot \pi_\varphi(a)$ (respects scalars)
\end{enumerate}
\end{theorem}

The Lean proofs use \texttt{Submodule.liftQ} for the well-definedness lift.

%----------------------------------------------------------------------
\section{Inner Product Structure}
\label{sec:inner-product}

\begin{definition}[GNS Inner Product]
\label{def:gns-inner}
The inner product on $A/N_\varphi$ is defined by:
\[
\langle [a], [b] \rangle = \varphi(b^* a)
\]
\end{definition}

\begin{theorem}[Well-Definedness]
\label{thm:inner-well-defined}
The inner product is well-defined on the quotient.
\end{theorem}

\begin{proof}
If $a_1 - a_2 \in N_\varphi$, then by Corollary~\ref{cor:null-consequence},
\[
\varphi(b^* a_1) - \varphi(b^* a_2) = \varphi(b^* (a_1 - a_2)) = 0
\]
Similarly for the second argument using conjugate symmetry.
\end{proof}

\begin{theorem}[Inner Product Properties]
\label{thm:inner-properties}
The inner product satisfies:
\begin{enumerate}
    \item Conjugate symmetry: $\langle x, y \rangle = \overline{\langle y, x \rangle}$
    \item Linearity in first argument
    \item Non-negativity: $\langle x, x \rangle \geq 0$ and $\langle x, x \rangle \in \mathbb{R}$
    \item Positive definiteness: $\langle x, x \rangle = 0 \Leftrightarrow x = 0$
\end{enumerate}
\end{theorem}

\begin{proof}
Conjugate symmetry follows from \texttt{sesqForm\_conj\_symm}. Non-negativity is inherited from positivity of states. Positive definiteness: $\langle [a], [a] \rangle = \varphi(a^* a) = 0$ iff $a \in N_\varphi$ iff $[a] = 0$.
\end{proof}

The norm on the quotient is $\|[a]\| = \sqrt{\operatorname{Re}(\varphi(a^* a))}$.

%----------------------------------------------------------------------
\section{Boundedness}
\label{sec:boundedness}

\begin{theorem}[Boundedness of Pre-Representation]
\label{thm:pre-rep-bounded}
For any $a \in A$ and $x \in A/N_\varphi$:
\[
\|\pi_\varphi(a) x\| \leq \|a\| \cdot \|x\|
\]
\end{theorem}

\begin{proof}
The key inequality from $C^*$-algebra theory is $a^* a \leq \|a\|^2 \cdot 1$ (spectral ordering). Since states are monotone on positive elements, we get
\[
\varphi(b^* (a^* a) b) \leq \|a\|^2 \cdot \varphi(b^* b)
\]
for any $b \in A$. This is the inequality \texttt{key\_inequality}. For $x = [b]$:
\[
\|\pi_\varphi(a)[b]\|^2 = \varphi((ab)^*(ab)) = \varphi(b^* a^* a b) \leq \|a\|^2 \cdot \varphi(b^* b) = \|a\|^2 \|[b]\|^2
\]
Taking square roots gives the result. The Lean proof uses \texttt{sq\_le\_sq\ensuremath{_0}} to convert from squared to non-squared norms.
\end{proof}

%----------------------------------------------------------------------
\section{Completion to Hilbert Space}
\label{sec:completion}

\begin{definition}[GNS Hilbert Space]
\label{def:gns-hilbert}
The \emph{GNS Hilbert space} $H_\varphi$ is the completion of $A/N_\varphi$ with respect to the norm induced by the inner product.
\end{definition}

\begin{lstlisting}[language=lean]
abbrev gnsHilbertSpace := UniformSpace.Completion phi.gnsQuotient

instance gnsHilbertSpaceCompleteSpace : CompleteSpace phi.gnsHilbertSpace :=
  UniformSpace.Completion.completeSpace phi.gnsQuotient
\end{lstlisting}

The inner product space structure on the quotient induces an inner product space structure on the completion via mathlib's \texttt{InnerProductSpace.Completion}.

%----------------------------------------------------------------------
\section{Extension of Operators}
\label{sec:extension}

\begin{theorem}[Extension to Hilbert Space]
\label{thm:extension}
The pre-representation $\pi_\varphi(a)$ extends uniquely to a bounded linear operator on $H_\varphi$.
\end{theorem}

\begin{proof}
By Theorem~\ref{thm:pre-rep-bounded}, $\pi_\varphi(a)$ is bounded on $A/N_\varphi$. Bounded linear maps on a dense subspace of a complete space extend uniquely by uniform continuity. The extension uses \texttt{UniformSpace.Completion.map}.
\end{proof}

\begin{definition}[GNS Representation]
\label{def:gns-rep}
The \emph{GNS representation} $\pi_\varphi : A \to B(H_\varphi)$ is defined by extending each $\pi_\varphi(a)$ to the completion.
\end{definition}

\begin{lstlisting}[language=lean]
noncomputable def gnsRep (a : A) : phi.gnsHilbertSpace ->L[C] phi.gnsHilbertSpace where
  toLinearMap := {
    toFun := UniformSpace.Completion.map (phi.gnsPreRepContinuous a)
    map_add' := ...
    map_smul' := ...
  }
  cont := UniformSpace.Completion.continuous_map
\end{lstlisting}

\begin{theorem}[Algebraic Properties]
\label{thm:gns-rep-algebra}
The GNS representation satisfies:
\begin{enumerate}
    \item $\pi_\varphi(ab) = \pi_\varphi(a) \circ \pi_\varphi(b)$
    \item $\pi_\varphi(1) = \mathrm{id}$
    \item $\pi_\varphi(a+b) = \pi_\varphi(a) + \pi_\varphi(b)$
\end{enumerate}
\end{theorem}

\begin{proof}
Each property is proven by density: both sides are continuous and agree on the dense quotient.
\end{proof}

%----------------------------------------------------------------------
\section{Star Structure}
\label{sec:star-structure}

\begin{theorem}[Star Preservation]
\label{thm:gns-rep-star}
The GNS representation preserves the star: $\pi_\varphi(a^*) = \pi_\varphi(a)^\dagger$.
\end{theorem}

\begin{proof}
We show $\langle \pi_\varphi(a^*) x, y \rangle = \langle x, \pi_\varphi(a) y \rangle$ for all $x, y$. By density, it suffices to check on quotient elements $x = [b]$, $y = [c]$:
\begin{align*}
\langle \pi_\varphi(a^*)[b], [c] \rangle &= \langle [a^* b], [c] \rangle = \varphi(c^* a^* b) \\
\langle [b], \pi_\varphi(a)[c] \rangle &= \langle [b], [ac] \rangle = \varphi((ac)^* b) = \varphi(c^* a^* b)
\end{align*}
The key calculation is \texttt{gnsPreRep\_inner\_star}.
\end{proof}

\begin{definition}[Star Algebra Homomorphism]
\label{def:star-alg-hom}
The GNS representation is a $*$-algebra homomorphism $A \to_{\ast\mathrm{alg}} B(H_\varphi)$.
\end{definition}

\begin{lstlisting}[language=lean]
noncomputable def gnsStarAlgHom : A ->*a[C] (phi.gnsHilbertSpace ->L[C] phi.gnsHilbertSpace) where
  toFun := phi.gnsRep
  map_one' := gnsRep_one phi
  map_mul' := fun a b => by rw [gnsRep_mul, ContinuousLinearMap.mul_def]
  map_star' := fun a => (gnsRep_star' phi a).symm
  ...
\end{lstlisting}

%----------------------------------------------------------------------
\section{The Cyclic Vector}
\label{sec:cyclic-vector}

\begin{definition}[Cyclic Vector]
\label{def:cyclic-vector}
The \emph{GNS cyclic vector} $\Omega_\varphi \in H_\varphi$ is the image of $[1]$ under the embedding of the quotient into the completion.
\end{definition}

\begin{lstlisting}[language=lean]
noncomputable def gnsCyclicVector : phi.gnsHilbertSpace :=
  (Submodule.Quotient.mk (p := phi.gnsNullIdeal) 1 : phi.gnsQuotient)
\end{lstlisting}

\begin{theorem}[Unit Norm]
\label{thm:cyclic-norm}
$\|\Omega_\varphi\| = 1$.
\end{theorem}

\begin{proof}
$\|\Omega_\varphi\|^2 = \langle [1], [1] \rangle = \varphi(1^* \cdot 1) = \varphi(1) = 1$.
\end{proof}

\begin{theorem}[Vector State Property]
\label{thm:vector-state}
For all $a \in A$:
\[
\varphi(a) = \langle \Omega_\varphi, \pi_\varphi(a) \Omega_\varphi \rangle
\]
\end{theorem}

\begin{proof}
We compute:
\begin{align*}
\langle \Omega_\varphi, \pi_\varphi(a) \Omega_\varphi \rangle &= \langle [1], \pi_\varphi(a)[1] \rangle \\
&= \langle [1], [a] \rangle \\
&= \varphi(1^* \cdot a) = \varphi(a)
\end{align*}
The proof uses \texttt{gnsRep\_cyclicVector} to show $\pi_\varphi(a)\Omega_\varphi = [a]$.
\end{proof}

\begin{theorem}[Cyclicity]
\label{thm:cyclicity}
The orbit $\{\pi_\varphi(a)\Omega_\varphi : a \in A\}$ is dense in $H_\varphi$.
\end{theorem}

\begin{proof}
Since $\pi_\varphi(a)\Omega_\varphi = [a]$ and the quotient map $A \to A/N_\varphi$ is surjective, the orbit equals the embedded image of $A/N_\varphi$, which is dense in the completion by construction.
\end{proof}

%----------------------------------------------------------------------
\section{The GNS Theorem}
\label{sec:gns-theorem}

\begin{theorem}[GNS Construction]
\label{thm:gns}
Let $\varphi$ be a state on a $C^*$-algebra $A$. There exists:
\begin{enumerate}
    \item A Hilbert space $H_\varphi$
    \item A $*$-representation $\pi_\varphi : A \to B(H_\varphi)$
    \item A unit vector $\Omega_\varphi \in H_\varphi$ with $\|\Omega_\varphi\| = 1$
\end{enumerate}
such that:
\begin{enumerate}
    \item[(a)] $\varphi(a) = \langle \Omega_\varphi, \pi_\varphi(a) \Omega_\varphi \rangle$ for all $a \in A$
    \item[(b)] $\{\pi_\varphi(a)\Omega_\varphi : a \in A\}$ is dense in $H_\varphi$
\end{enumerate}
\end{theorem}

\begin{proof}
The construction has been carried out in the preceding sections. The Lean statement is:
\begin{lstlisting}[language=lean]
theorem gns_theorem :
    ||phi.gnsCyclicVector|| = 1 /\
    (forall a : A, phi a = @inner C phi.gnsHilbertSpace _ phi.gnsCyclicVector
                      (phi.gnsRep a phi.gnsCyclicVector)) /\
    DenseRange (fun a : A => phi.gnsRep a phi.gnsCyclicVector) :=
  <gnsCyclicVector_norm phi, gns_vector_state phi, gnsCyclicVector_denseRange phi>
\end{lstlisting}
\end{proof}

%----------------------------------------------------------------------
\section{Uniqueness}
\label{sec:uniqueness}

The GNS representation is unique up to unitary equivalence.

\begin{theorem}[GNS Uniqueness]
\label{thm:gns-uniqueness}
Let $(H, \pi, \xi)$ be a cyclic $*$-representation with:
\begin{itemize}
    \item $\pi : A \to B(H)$ is a $*$-representation
    \item $\|\xi\| = 1$
    \item $\varphi(a) = \langle \xi, \pi(a)\xi \rangle$ for all $a \in A$
    \item $\{\pi(a)\xi : a \in A\}$ is dense in $H$
\end{itemize}
Then there exists a unitary $U : H_\varphi \to H$ such that:
\begin{enumerate}
    \item $U(\Omega_\varphi) = \xi$
    \item $U \circ \pi_\varphi(a) = \pi(a) \circ U$ for all $a \in A$
\end{enumerate}
\end{theorem}

The proof constructs the intertwiner $U$ in several stages.

\subsection{Construction of the Intertwiner}

\begin{definition}[Intertwiner on Quotient]
\label{def:intertwiner-quotient}
Define $U_0 : A/N_\varphi \to H$ by $U_0([a]) = \pi(a)\xi$.
\end{definition}

\begin{lemma}[Well-Definedness]
\label{lem:intertwiner-well-defined}
$U_0$ is well-defined: if $a - b \in N_\varphi$, then $\pi(a)\xi = \pi(b)\xi$.
\end{lemma}

\begin{proof}
If $\varphi((a-b)^*(a-b)) = 0$, then using the state condition:
\[
\|\pi(a-b)\xi\|^2 = \langle \xi, \pi((a-b)^*(a-b))\xi \rangle = \varphi((a-b)^*(a-b)) = 0
\]
so $\pi(a-b)\xi = 0$, giving $\pi(a)\xi = \pi(b)\xi$.
\end{proof}

\begin{lemma}[Isometry]
\label{lem:intertwiner-isometry}
$U_0$ is an isometry: $\|U_0([a])\| = \|[a]\|$.
\end{lemma}

\begin{proof}
\begin{align*}
\|U_0([a])\|^2 = \|\pi(a)\xi\|^2 &= \langle \xi, \pi(a)^\dagger \pi(a)\xi \rangle \\
&= \langle \xi, \pi(a^*)\pi(a)\xi \rangle \\
&= \langle \xi, \pi(a^* a)\xi \rangle \\
&= \varphi(a^* a) = \|[a]\|^2
\end{align*}
\end{proof}

\begin{lemma}[Linearity]
\label{lem:intertwiner-linear}
$U_0$ is linear: $U_0(x + y) = U_0(x) + U_0(y)$ and $U_0(c \cdot x) = c \cdot U_0(x)$.
\end{lemma}

\subsection{Extension to Hilbert Space}

\begin{lemma}[Extension]
\label{lem:intertwiner-extension}
$U_0$ extends uniquely to an isometry $U : H_\varphi \to H$.
\end{lemma}

\begin{proof}
As an isometry, $U_0$ is uniformly continuous. By the universal property of completions (\texttt{UniformSpace.Completion.extension}), it extends uniquely to the completion. The extension preserves the isometry property by \texttt{Isometry.completion\_extension}.
\end{proof}

\subsection{Surjectivity}

\begin{lemma}[Dense Range]
\label{lem:intertwiner-dense-range}
The range of $U$ contains the orbit $\{\pi(a)\xi : a \in A\}$, hence is dense.
\end{lemma}

\begin{proof}
For any $a \in A$, we have $U([a]) = \pi(a)\xi$, so $\pi(a)\xi \in \mathrm{range}(U)$.
\end{proof}

\begin{lemma}[Surjectivity]
\label{lem:intertwiner-surjective}
$U$ is surjective.
\end{lemma}

\begin{proof}
An isometry from a complete space into a complete space with dense range is surjective. The range is complete (image of complete space under uniform inducing map), hence closed. Dense and closed implies the whole space.
\end{proof}

\subsection{Intertwining Property}

\begin{lemma}[Intertwining]
\label{lem:intertwining}
$U \circ \pi_\varphi(a) = \pi(a) \circ U$ for all $a \in A$.
\end{lemma}

\begin{proof}
On quotient elements:
\[
U(\pi_\varphi(a)[b]) = U([ab]) = \pi(ab)\xi = \pi(a)\pi(b)\xi = \pi(a)(U([b]))
\]
Both sides are continuous, so by density the identity extends to all of $H_\varphi$.
\end{proof}

\subsection{Cyclic Vector Mapping}

\begin{lemma}[Cyclic Vector]
\label{lem:cyclic-mapping}
$U(\Omega_\varphi) = \xi$.
\end{lemma}

\begin{proof}
$U(\Omega_\varphi) = U([1]) = \pi(1)\xi = \xi$.
\end{proof}

\begin{proof}[Proof of Theorem~\ref{thm:gns-uniqueness}]
Combining the above lemmas, $U$ is a linear isometry equivalence (bijective isometry) that maps $\Omega_\varphi$ to $\xi$ and intertwines the representations. The Lean statement:
\begin{lstlisting}[language=lean]
theorem gns_uniqueness
    (_h_xi_norm : ||xi|| = 1)
    (h_xi_cyclic : DenseRange (fun a => pi a xi))
    (h_xi_state : forall a : A, @inner C H _ xi (pi a xi) = phi a) :
    exists U : phi.gnsHilbertSpace =~li[C] H,
      U phi.gnsCyclicVector = xi /\
      forall a : A, forall x : phi.gnsHilbertSpace, U (phi.gnsRep a x) = pi a (U x) :=
  <gnsIntertwinerEquiv phi pi xi h_xi_state h_xi_cyclic,
   gnsIntertwinerEquiv_cyclic phi pi xi h_xi_state h_xi_cyclic,
   gnsIntertwiner_intertwines phi pi xi h_xi_state h_xi_cyclic>
\end{lstlisting}
\end{proof}

%----------------------------------------------------------------------
\section{Summary of Key Definitions}
\label{sec:gns-summary}

For reference, the main Lean definitions in the GNS construction:

\begin{center}
\begin{tabular}{ll}
\hline
\textbf{Definition} & \textbf{Lean Name} \\
\hline
State & \texttt{State} \\
Null space & \texttt{State.gnsNullSpace} \\
Null ideal (submodule) & \texttt{State.gnsNullIdeal} \\
Quotient space & \texttt{State.gnsQuotient} \\
Inner product & \texttt{State.gnsInner} \\
Pre-representation & \texttt{State.gnsPreRep} \\
Hilbert space & \texttt{State.gnsHilbertSpace} \\
GNS representation & \texttt{State.gnsRep} \\
Star algebra homomorphism & \texttt{State.gnsStarAlgHom} \\
Cyclic vector & \texttt{State.gnsCyclicVector} \\
Intertwiner equivalence & \texttt{State.gnsIntertwinerEquiv} \\
\hline
\end{tabular}
\end{center}

The complete GNS formalization comprises approximately 2,455 lines of Lean code with zero sorries, providing a fully machine-checked proof of both the existence and uniqueness theorems.
