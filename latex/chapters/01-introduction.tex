% Chapter: Introduction
% Auto-generated from Lean 4 source
% Source files: AfTests/ArchimedeanClosure/Main/Theorem.lean,
%               AfTests/ArchimedeanClosure/Main/DualCharacterization.lean
% Do not edit manually

\chapter{Introduction}
\label{ch:introduction}

%% ============================================================
\section{Overview}
\label{sec:intro-overview}

This document presents a formal verification in Lean~4 of a fundamental
result in the theory of constrained $C^*$-algebras: the characterization
of positivity in constrained $*$-representations in terms of membership
in the closure of a quadratic module.

The core mathematical content concerns the free $*$-algebra $\Azero$
generated by finitely many self-adjoint elements $\gen{1}, \ldots, \gen{n}$,
equipped with a \emph{quadratic module}~$\M$ that encodes positivity
constraints on the generators. The main theorem establishes a precise
duality between two notions of positivity for self-adjoint elements.

%% ============================================================
\section{Main Result}
\label{sec:intro-main-result}

The central result of this formalization is the following characterization
theorem.

\begin{theorem}[Main Theorem]\label{thm:main_theorem}
Let $A \in \Azero$ be a self-adjoint element. Then
\[
  A \in \Mbar \quad\Longleftrightarrow\quad
  \Rep(A) \ge 0 \text{ for all constrained $*$-representations } \Rep.
\]
\end{theorem}

Here:
\begin{itemize}
  \item $\Mbar$ denotes the closure of the quadratic module $\M$ in the
        state seminorm topology;
  \item A \emph{constrained $*$-representation} is a $*$-homomorphism
        $\Rep : \Azero \to \BH$ such that $\Rep(\gen{j}) \ge 0$ for all
        generators~$\gen{j}$;
  \item The inequality $\Rep(A) \ge 0$ means that $\Rep(A)$ is a positive
        operator on the Hilbert space~$\Hilbert$.
\end{itemize}

This theorem provides a dual characterization: an element belongs to
the topological closure of $\M$ if and only if it ``looks positive''
in every representation respecting the generator constraints.

%% ============================================================
\section{Key Concepts}
\label{sec:intro-concepts}

We briefly introduce the main mathematical objects involved.

\begin{definition}[Free $*$-Algebra]\label{def:FreeStarAlgebra}
The \textbf{free $*$-algebra} $\Azero = \R\langle \gen{1}, \ldots, \gen{n} \rangle$
is the free $\R$-algebra on $n$ generators, equipped with the involution
$*$ determined by declaring each generator self-adjoint: $\gen{j}^* = \gen{j}$.
\end{definition}

\begin{definition}[Quadratic Module]\label{def:QuadraticModule}
The \textbf{quadratic module} $\M \subseteq \Azero$ is the set of finite
sums of the form
\[
  \M = \left\{
    \sum_i a_i^* a_i + \sum_{j,k} b_{jk}^* \gen{j} b_{jk}
    : a_i, b_{jk} \in \Azero
  \right\}.
\]
Equivalently, $\M$ is the cone generated by elements $a^* a$ (squares)
and $b^* \gen{j} b$ (generator-weighted squares).
\end{definition}

\begin{definition}[$\M$-Positive State]\label{def:MPositiveState}
An \textbf{$\M$-positive state} is an $\R$-linear functional
$\varphi : \Azero \to \R$ satisfying:
\begin{enumerate}
  \item (Symmetric) $\varphi(a^*) = \varphi(a)$ for all $a \in \Azero$;
  \item (Normalized) $\varphi(1) = 1$;
  \item ($\M$-positive) $\varphi(m) \ge 0$ for all $m \in \M$.
\end{enumerate}
The set of all $\M$-positive states is denoted $\SM$.
\end{definition}

\begin{definition}[State Seminorm]\label{def:stateSeminorm}
The \textbf{state seminorm} on $\Azero$ is defined by
\[
  \seminorm{a} := \sup \{ \abs{\varphi(a)} : \varphi \in \SM \}.
\]
\end{definition}

\begin{definition}[Archimedean Property]\label{def:IsArchimedean}
The quadratic module $\M$ is \textbf{Archimedean} if for every
$a \in \Azero$, there exists $N \in \N$ such that
\[
  N \cdot 1 - a^* a \in \M.
\]
This bounds the ``size'' of elements in terms of $\M$-membership.
\end{definition}

%% ============================================================
\section{Proof Structure}
\label{sec:intro-proof-structure}

The proof of \Cref{thm:main_theorem} proceeds in eight phases,
mirroring the structure of the Lean formalization.

\subsection*{Phase 1: Algebraic Setup}
We construct the free $*$-algebra $\Azero$ over $\R$ with self-adjoint
generators, define the quadratic module $\M$, and establish the
Archimedean property (\Cref{ch:algebra}).

\subsection*{Phase 2: States}
We define $\M$-positive states and prove that the state space $\SM$
is non-empty via the \emph{scalar extraction} functional
(\Cref{ch:states}).

\subsection*{Phase 3: Boundedness}
Using a Cauchy--Schwarz inequality adapted for $\M$-positive states
and the Archimedean property, we prove that states are uniformly
bounded: $\abs{\varphi(a)}^2 \le N_a$ for a constant $N_a$
depending only on $a$ (\Cref{ch:boundedness}).

\subsection*{Phase 4: Topology}
We equip the dual space with the weak-$*$ topology and prove that
$\SM$ is compact, using the Tychonoff theorem together with the
boundedness results (\Cref{ch:topology}).

\subsection*{Phase 5: Seminorm}
We define the state seminorm $\seminorm{\cdot}$ and characterize
the closure $\Mbar$ in this topology (\Cref{ch:seminorm}).

\subsection*{Phase 6: Dual Characterization}
The key intermediate result is:
\begin{theorem}[Dual Characterization]\label{thm:dual_characterization}
For self-adjoint $A \in \Azero$:
\[
  A \in \Mbar \quad\Longleftrightarrow\quad
  \varphi(A) \ge 0 \text{ for all } \varphi \in \SM.
\]
\end{theorem}
The forward direction uses continuity; the backward direction
applies the Riesz extension theorem to construct a separating
functional when $A \notin \Mbar$ (\Cref{ch:dual}).

\subsection*{Phase 7: Representations}
We define constrained $*$-representations and establish the
correspondence between $\M$-positive states and vector states
from constrained representations via the GNS construction
(\Cref{ch:gns,ch:representations}).

\subsection*{Phase 8: Main Theorem}
The main theorem follows by chaining:
\begin{enumerate}
  \item $A \in \Mbar \Leftrightarrow \varphi(A) \ge 0$ for all states
        (dual characterization);
  \item $\varphi(A) \ge 0$ for all states $\Rightarrow \Rep(A) \ge 0$
        for all constrained representations (by vector states);
  \item $\Rep(A) \ge 0$ for all representations $\Rightarrow \varphi(A) \ge 0$
        for all states (by the GNS construction).
\end{enumerate}
See \Cref{ch:main-theorem} for the complete proof.

%% ============================================================
\section{The Lean Formalization}
\label{sec:intro-lean}

The formalization is implemented in Lean~4 using the Mathlib library.
The code comprises approximately 965 lines across 26 files, organized
into modules corresponding to the proof phases above.

\subsection*{Key Mathlib Dependencies}
The formalization relies on several Mathlib components:
\begin{itemize}
  \item \texttt{Algebra.FreeAlgebra} --- free algebras over a ring;
  \item \texttt{Algebra.Star.Free} --- star structure on free algebras;
  \item \texttt{Analysis.Convex.Cone.Extension} --- the Riesz extension
        theorem for convex cones;
  \item \texttt{Topology.Compactness.Compact} --- the Tychonoff theorem;
  \item \texttt{Analysis.Seminorm} --- seminorm infrastructure.
\end{itemize}

\subsection*{Design Decisions}
A critical design decision was to work over $\R$ rather than $\C$.
In Mathlib's \texttt{FreeAlgebra}, the star operation does \emph{not}
conjugate scalars: $(\alpha \cdot 1)^* = \alpha \cdot 1$ rather than
$\bar{\alpha} \cdot 1$. Over $\C$, this leads to $i^* \cdot i = -1$,
breaking positivity arguments. Working over $\R$ ensures that
$c^* \cdot c = c^2 \ge 0$ for real scalars $c$.

\subsection*{Reuse from GNS Infrastructure}
The formalization builds on a prior complete verification of the
GNS (Gelfand--Naimark--Segal) construction, providing:
\begin{itemize}
  \item Cauchy--Schwarz inequality for states;
  \item GNS Hilbert space construction;
  \item Uniqueness up to unitary equivalence.
\end{itemize}

%% ============================================================
\section{Notation}
\label{sec:intro-notation}

Throughout this document, we use the following notation:

\begin{notation}\label{not:conventions}
\begin{itemize}
  \item $\Azero$ --- the free $*$-algebra $\R\langle \gen{1}, \ldots, \gen{n} \rangle$
  \item $\gen{j}$ --- the $j$-th generator (self-adjoint)
  \item $a^*$ --- the $*$-involution (adjoint) of $a$
  \item $\M$ --- the quadratic module
  \item $\Mbar$ --- the closure of $\M$ in the state seminorm topology
  \item $\SM$ --- the set of $\M$-positive states
  \item $\seminorm{a}$ --- the state seminorm of $a$
  \item $\Hilbert$, $\HilbertPhi$ --- Hilbert spaces
  \item $\BH$ --- bounded operators on $\Hilbert$
  \item $\Rep$ --- a $*$-representation
\end{itemize}
\end{notation}

%% ============================================================
\section{Document Structure}
\label{sec:intro-structure}

The remainder of this document is organized as follows:

\begin{description}
  \item[\Cref{ch:algebra}] Algebraic Setup --- free $*$-algebra,
    quadratic module, Archimedean property
  \item[\Cref{ch:states}] States --- $\M$-positive states, non-emptiness
  \item[\Cref{ch:boundedness}] Boundedness --- Cauchy--Schwarz,
    Archimedean bounds
  \item[\Cref{ch:topology}] Topology --- weak-$*$ topology, compactness
  \item[\Cref{ch:seminorm}] Seminorm --- state seminorm, closure
  \item[\Cref{ch:dual}] Dual Characterization --- Riesz extension
  \item[\Cref{ch:gns}] GNS Construction --- Hilbert space construction
  \item[\Cref{ch:representations}] Representations --- constrained
    $*$-representations
  \item[\Cref{ch:main-theorem}] Main Theorem --- final synthesis
  \item[\Cref{app:nonemptiness}] Appendix: Non-emptiness of $\SM$
\end{description}
