% Auto-generated from Lean 4 source
% Source files: StateSeminorm.lean, SeminormProps.lean, Closure.lean
% Do not edit manually

\chapter{The State Seminorm}
\label{ch:seminorm}

This chapter develops the state seminorm $\|{\cdot}\|_M$ on the free $*$-algebra $\mathcal{A}_0$.
The Archimedean property ensures that for any element $a \in \mathcal{A}_0$, the supremum
$\sup\{|\varphi(a)| : \varphi \in S_M\}$ is finite, yielding a well-defined seminorm. We then
define the closure $\bar{M}$ of the quadratic module in this seminorm topology.

The main results are:
\begin{itemize}
  \item \texttt{stateSeminorm} (Definition~\ref{def:stateSeminorm}): the state seminorm $\|a\|_M$
  \item \texttt{stateSeminormSeminorm} (Definition~\ref{def:stateSeminormSeminorm}): the seminorm instance
  \item \texttt{quadraticModuleClosure} (Definition~\ref{def:quadraticModuleClosure}): the closure $\bar{M}$
\end{itemize}

%% ============================================================
\section{Definition of the State Seminorm}
\label{sec:seminorm-def}

The Archimedean property ensures that for any $a \in \mathcal{A}_0$, there exists $N_a \in \mathbb{R}$
such that $N_a \cdot 1 - a^*a \in M$. This bound is used to show that the set of state values
$\{|\varphi(a)| : \varphi \in S_M\}$ is bounded above.

\begin{lemma}[Bounded above]\label{lem:bddAbove_abs_range}
For any $a \in \mathcal{A}_0$, the set $\{|\varphi(a)| : \varphi \in S_M\}$ is bounded above by $\sqrt{N_a}$.
\end{lemma}

\begin{proof}
By the Archimedean bound (Lemma~\ref{lem:ArchimedeanBound}), for each $\varphi \in S_M$ we have
$|\varphi(a)| \le \sqrt{N_a}$. Hence $\sqrt{N_a}$ is an upper bound.
\end{proof}

\begin{definition}[State Seminorm]\label{def:stateSeminorm}
The \textbf{state seminorm} $\|{\cdot}\|_M : \mathcal{A}_0 \to \mathbb{R}$ is defined by
\[
  \|a\|_M = \sup\{|\varphi(a)| : \varphi \in S_M\}.
\]
\end{definition}

The nonemptiness of $S_M$ (guaranteed by the scalar state $\varphi_0$) ensures this supremum is
well-defined, and Lemma~\ref{lem:bddAbove_abs_range} ensures it is finite.

\begin{theorem}[Seminorm Upper Bound]\label{thm:stateSeminorm_le}
For any $a \in \mathcal{A}_0$,
\[
  \|a\|_M \le \sqrt{N_a}.
\]
\end{theorem}

\begin{proof}
Apply the supremum bound: since $|\varphi(a)| \le \sqrt{N_a}$ for all $\varphi \in S_M$,
the supremum is at most $\sqrt{N_a}$.
\end{proof}

\begin{lemma}[Nonnegativity]\label{lem:stateSeminorm_nonneg}
For any $a \in \mathcal{A}_0$, $\|a\|_M \ge 0$.
\end{lemma}

\begin{proof}
The scalar state $\varphi_0 \in S_M$ satisfies $|\varphi_0(a)| \ge 0$.
Since $\|a\|_M$ is the supremum over a set containing $|\varphi_0(a)|$, we have $\|a\|_M \ge 0$.
\end{proof}

\begin{lemma}[State Bound]\label{lem:apply_abs_le_seminorm}
For any $\varphi \in S_M$ and $a \in \mathcal{A}_0$,
\[
  |\varphi(a)| \le \|a\|_M.
\]
\end{lemma}

\begin{proof}
By definition, $\|a\|_M$ is the supremum of the set containing $|\varphi(a)|$.
\end{proof}

\begin{theorem}[Triangle Inequality]\label{thm:stateSeminorm_add}
For any $a, b \in \mathcal{A}_0$,
\[
  \|a + b\|_M \le \|a\|_M + \|b\|_M.
\]
\end{theorem}

\begin{proof}[Proof sketch]
For each $\varphi \in S_M$, linearity gives $\varphi(a+b) = \varphi(a) + \varphi(b)$, so
\[
  |\varphi(a+b)| \le |\varphi(a)| + |\varphi(b)|.
\]
Taking the supremum over $\varphi$ on both sides and using the additivity of suprema yields the result.
\end{proof}

%% ============================================================
\section{Seminorm Properties}
\label{sec:seminorm-props}

We verify that $\|{\cdot}\|_M$ satisfies all the axioms of a seminorm over $\mathbb{R}$.

\begin{lemma}[Zero]\label{lem:stateSeminorm_zero}
$\|0\|_M = 0$.
\end{lemma}

\begin{proof}
For any $\varphi \in S_M$, linearity gives $\varphi(0) = 0$, so $|\varphi(0)| = 0$.
The supremum of a constant function is that constant.
\end{proof}

\begin{lemma}[Negation]\label{lem:stateSeminorm_neg}
For any $a \in \mathcal{A}_0$, $\|-a\|_M = \|a\|_M$.
\end{lemma}

\begin{proof}
Linearity gives $\varphi(-a) = -\varphi(a)$, so $|\varphi(-a)| = |\varphi(a)|$.
The sets over which we take suprema are identical.
\end{proof}

\begin{theorem}[Homogeneity]\label{thm:stateSeminorm_smul}
For any $c \in \mathbb{R}$ and $a \in \mathcal{A}_0$,
\[
  \|c \cdot a\|_M = |c| \cdot \|a\|_M.
\]
\end{theorem}

\begin{proof}[Proof sketch]
If $c = 0$, both sides are zero by Lemma~\ref{lem:stateSeminorm_zero}.
Otherwise, linearity gives $|\varphi(c \cdot a)| = |c| \cdot |\varphi(a)|$.
The supremum factors out the constant $|c| \ge 0$.
\end{proof}

\begin{definition}[Seminorm Instance]\label{def:stateSeminormSeminorm}
The state seminorm $\|{\cdot}\|_M$ forms a \textbf{seminorm} over $\mathbb{R}$ on $\mathcal{A}_0$,
constructed via the triangle inequality (Theorem~\ref{thm:stateSeminorm_add}) and
homogeneity (Theorem~\ref{thm:stateSeminorm_smul}).
\end{definition}

\begin{remark}
The seminorm $\|{\cdot}\|_M$ is not necessarily a norm: we may have $\|a\|_M = 0$ for nonzero $a$.
However, the kernel $\{a : \|a\|_M = 0\}$ is precisely the intersection of all GNS null spaces.
\end{remark}

%% ============================================================
\section{Closure of the Quadratic Module}
\label{sec:closure}

We now define the closure of $M$ in the topology induced by $\|{\cdot}\|_M$.

\begin{definition}[Quadratic Module Closure]\label{def:quadraticModuleClosure}
The \textbf{closure} of $M$, denoted $\bar{M}$, is defined as
\[
  \bar{M} = \{a \in \mathcal{A}_0 : \forall \varepsilon > 0,\, \exists m \in M,\, \|a - m\|_M < \varepsilon\}.
\]
\end{definition}

\begin{theorem}[Module Subset]\label{thm:quadraticModule_subset_closure}
$M \subseteq \bar{M}$.
\end{theorem}

\begin{proof}
Let $m \in M$. For any $\varepsilon > 0$, take $m' = m$. Then $\|m - m'\|_M = \|0\|_M = 0 < \varepsilon$.
\end{proof}

\begin{lemma}[Zero in Closure]\label{lem:zero_mem_closure}
$0 \in \bar{M}$.
\end{lemma}

\begin{proof}
Since $0 = 0^* \cdot 0 \in M$, this follows from Theorem~\ref{thm:quadraticModule_subset_closure}.
\end{proof}

\begin{theorem}[Closure Closed under Addition]\label{thm:closure_add_mem}
If $a, b \in \bar{M}$, then $a + b \in \bar{M}$.
\end{theorem}

\begin{proof}[Proof sketch]
Let $\varepsilon > 0$. By definition, there exist $m_a, m_b \in M$ with
$\|a - m_a\|_M < \varepsilon/2$ and $\|b - m_b\|_M < \varepsilon/2$.
Since $M$ is closed under addition, $m_a + m_b \in M$. Then
\begin{align*}
  \|a + b - (m_a + m_b)\|_M &= \|(a - m_a) + (b - m_b)\|_M \\
  &\le \|a - m_a\|_M + \|b - m_b\|_M \\
  &< \varepsilon/2 + \varepsilon/2 = \varepsilon.
\end{align*}
\end{proof}

\begin{theorem}[Closure Closed under Scalar Multiplication]\label{thm:closure_smul_mem}
If $c \ge 0$ and $a \in \bar{M}$, then $c \cdot a \in \bar{M}$.
\end{theorem}

\begin{proof}[Proof sketch]
If $c = 0$, then $c \cdot a = 0 \in \bar{M}$ by Lemma~\ref{lem:zero_mem_closure}.

Otherwise $c > 0$. Let $\varepsilon > 0$. Choose $m \in M$ with $\|a - m\|_M < \varepsilon/c$.
Since $M$ is closed under nonnegative scalar multiplication, $c \cdot m \in M$. Then
\begin{align*}
  \|c \cdot a - c \cdot m\|_M &= \|c \cdot (a - m)\|_M \\
  &= |c| \cdot \|a - m\|_M \\
  &= c \cdot \|a - m\|_M \\
  &< c \cdot (\varepsilon/c) = \varepsilon.
\end{align*}
\end{proof}

\begin{remark}
The closure $\bar{M}$ forms a cone: it is closed under addition and nonnegative scalar
multiplication. This structure is essential for the dual characterization via the Riesz
extension theorem in Chapter~\ref{ch:dual}.
\end{remark}
