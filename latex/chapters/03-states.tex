% Auto-generated from Lean 4 source
% Source files: MPositiveState.lean, MPositiveStateProps.lean, NonEmptiness.lean
% Do not edit manually

\chapter{$M$-Positive States}
\label{ch:states}

This chapter defines $M$-positive states on the free $*$-algebra and establishes
that the state space $\SM$ is nonempty. An $M$-positive state is an $\R$-linear
functional satisfying symmetry, normalization, and positivity conditions with
respect to the quadratic module $M$.

The main results are:
\begin{itemize}
  \item \texttt{MPositiveState} (Definition~\ref{def:MPositiveState}): the structure
        of an $M$-positive state
  \item \texttt{scalarState} (Definition~\ref{def:scalarState}): a canonical example
  \item \texttt{MPositiveStateSet\_nonempty} (Theorem~\ref{thm:nonempty}): $\SM \neq \emptyset$
\end{itemize}

%% ============================================================
\section{Definition of $M$-Positive States}
\label{sec:mpositive-state}

We work with $\R$-linear functionals rather than $\C$-linear ones. This choice
simplifies the theory: positivity conditions require no separate ``real part''
checks, and symmetry $\varphi(a^*) = \varphi(a)$ is well-defined without
complex conjugation.

\begin{definition}[$M$-Positive State]\label{def:MPositiveState}
An \textbf{$M$-positive state} is an $\R$-linear functional
$\varphi : \mathcal{A}_0 \to \R$ satisfying:
\begin{enumerate}
  \item (Symmetry) $\varphi(a^*) = \varphi(a)$ for all $a \in \mathcal{A}_0$
  \item (Normalization) $\varphi(1) = 1$
  \item ($M$-positivity) $\varphi(m) \ge 0$ for all $m \in M$
\end{enumerate}
\end{definition}

The symmetry condition determines values on all elements from their values on
self-adjoint elements, since any $a$ can be written as $a = \frac{1}{2}(a + a^*)
+ \frac{1}{2}(a - a^*)$ where $(a + a^*)$ is self-adjoint and $(a - a^*)$ is
skew-adjoint.

\begin{definition}[State Space]\label{def:MPositiveStateSet}
The \textbf{state space} $\SM$ is the set of all $M$-positive states:
\[
  \SM = \{ \varphi : \mathcal{A}_0 \to \R \mid
    \varphi \text{ is an } M\text{-positive state} \}
\]
\end{definition}

%% ============================================================
\section{Basic Properties}
\label{sec:state-properties}

The following properties follow directly from the definition of $M$-positive states.

\begin{lemma}[Positivity on Squares]\label{lem:star-mul-self-nonneg}
For any $M$-positive state $\varphi$ and $a \in \mathcal{A}_0$,
\[
  \varphi(a^* a) \ge 0.
\]
\end{lemma}

\begin{proof}
Since $a^* a \in M$ (as an element of the square set generating $M$), this
follows directly from the $M$-positivity condition.
\end{proof}

\begin{lemma}[Self-Adjoint Addition]\label{lem:self-adjoint-add}
For any $M$-positive state $\varphi$ and $a \in \mathcal{A}_0$,
\[
  \varphi(a + a^*) = 2\varphi(a).
\]
\end{lemma}

\begin{proof}
By linearity and symmetry:
$\varphi(a + a^*) = \varphi(a) + \varphi(a^*) = \varphi(a) + \varphi(a) = 2\varphi(a)$.
\end{proof}

\begin{lemma}[Skew-Adjoint Vanishing]\label{lem:self-adjoint-sub}
For any $M$-positive state $\varphi$ and $a \in \mathcal{A}_0$,
\[
  \varphi(a - a^*) = 0.
\]
\end{lemma}

\begin{proof}
By linearity and symmetry:
$\varphi(a - a^*) = \varphi(a) - \varphi(a^*) = \varphi(a) - \varphi(a) = 0$.
\end{proof}

\begin{lemma}[Decomposition]\label{lem:decomposition}
For any $M$-positive state $\varphi$ and $a \in \mathcal{A}_0$,
\[
  \varphi(a) = \frac{1}{2}\varphi(a + a^*).
\]
\end{lemma}

\begin{proof}
Immediate from Lemma~\ref{lem:self-adjoint-add}: dividing both sides by 2 gives
$\varphi(a) = \frac{1}{2}\varphi(a + a^*)$.
\end{proof}

%% ============================================================
\section{Non-Emptiness of $\SM$}
\label{sec:nonemptiness}

A crucial requirement for our theory is that $\SM$ is nonempty. We construct
an explicit $M$-positive state using scalar extraction.

\begin{definition}[Scalar Extraction]\label{def:scalarExtraction}
The \textbf{scalar extraction} functional $\varepsilon : \mathcal{A}_0 \to \R$
extracts the coefficient of the identity element. Formally, it is the algebra
homomorphism satisfying:
\begin{itemize}
  \item $\varepsilon(1) = 1$
  \item $\varepsilon(g_j) = 0$ for all generators $g_j$
\end{itemize}
This is the left inverse of the algebra map $\R \hookrightarrow \mathcal{A}_0$.
\end{definition}

\begin{lemma}[Scalar Extraction Commutes with Star]\label{lem:scalar-star}
For all $a \in \mathcal{A}_0$,
\[
  \varepsilon(a^*) = \varepsilon(a).
\]
\end{lemma}

\begin{proof}
By induction on the structure of $\mathcal{A}_0$:
\begin{itemize}
  \item For scalars: $\varepsilon(c^*) = \varepsilon(c) = c$ since $c \in \R$.
  \item For generators: $\varepsilon(g_j^*) = \varepsilon(g_j) = 0$ since
        generators are self-adjoint.
  \item For sums: follows from linearity.
  \item For products: $\varepsilon((ab)^*) = \varepsilon(b^* a^*)
        = \varepsilon(b^*)\varepsilon(a^*) = \varepsilon(b)\varepsilon(a)
        = \varepsilon(a)\varepsilon(b) = \varepsilon(ab)$ using the induction
        hypothesis and commutativity of $\R$.
\end{itemize}
\end{proof}

\begin{lemma}[Scalar Extraction on Squares]\label{lem:scalar-square}
For all $a \in \mathcal{A}_0$,
\[
  \varepsilon(a^* a) = \varepsilon(a)^2 \ge 0.
\]
\end{lemma}

\begin{proof}
Since $\varepsilon$ is an algebra homomorphism:
$\varepsilon(a^* a) = \varepsilon(a^*)\varepsilon(a) = \varepsilon(a)\varepsilon(a)
= \varepsilon(a)^2 \ge 0$.
\end{proof}

\begin{lemma}[Scalar Extraction on Generator-Weighted Squares]\label{lem:scalar-gen-square}
For all $a \in \mathcal{A}_0$ and generators $g_j$,
\[
  \varepsilon(a^* g_j a) = 0.
\]
\end{lemma}

\begin{proof}
Since $\varepsilon$ is an algebra homomorphism and $\varepsilon(g_j) = 0$:
$\varepsilon(a^* g_j a) = \varepsilon(a^*)\varepsilon(g_j)\varepsilon(a)
= \varepsilon(a^*) \cdot 0 \cdot \varepsilon(a) = 0$.
\end{proof}

\begin{lemma}[Scalar Extraction is $M$-Positive]\label{lem:scalar-m-nonneg}
For all $m \in M$,
\[
  \varepsilon(m) \ge 0.
\]
\end{lemma}

\begin{proof}
By induction on the construction of $m \in M$:
\begin{itemize}
  \item If $m = a^* a$: $\varepsilon(m) = \varepsilon(a)^2 \ge 0$ by
        Lemma~\ref{lem:scalar-square}.
  \item If $m = a^* g_j a$: $\varepsilon(m) = 0 \ge 0$ by
        Lemma~\ref{lem:scalar-gen-square}.
  \item If $m = m_1 + m_2$: $\varepsilon(m) = \varepsilon(m_1) + \varepsilon(m_2)
        \ge 0$ by the induction hypothesis and closure of $\R_{\ge 0}$ under
        addition.
  \item If $m = c \cdot m'$ with $c \ge 0$: $\varepsilon(m) = c \cdot \varepsilon(m')
        \ge 0$ by the induction hypothesis.
\end{itemize}
\end{proof}

\begin{definition}[Scalar State]\label{def:scalarState}
The \textbf{scalar state} $\varphi_0 : \mathcal{A}_0 \to \R$ is the $M$-positive
state defined by scalar extraction:
\[
  \varphi_0(a) = \varepsilon(a).
\]
\end{definition}

The scalar state satisfies all axioms of an $M$-positive state:
\begin{itemize}
  \item Symmetry: $\varphi_0(a^*) = \varepsilon(a^*) = \varepsilon(a) = \varphi_0(a)$
        by Lemma~\ref{lem:scalar-star}.
  \item Normalization: $\varphi_0(1) = \varepsilon(1) = 1$.
  \item $M$-positivity: $\varphi_0(m) = \varepsilon(m) \ge 0$ for $m \in M$ by
        Lemma~\ref{lem:scalar-m-nonneg}.
\end{itemize}

\begin{theorem}[Non-Emptiness of $\SM$]\label{thm:nonempty}
The state space $\SM$ is nonempty.
\end{theorem}

\begin{proof}
The scalar state $\varphi_0$ (Definition~\ref{def:scalarState}) is an
$M$-positive state, hence $\varphi_0 \in \SM$.
\end{proof}

\begin{remark}
The non-emptiness proof crucially relies on working over $\R$ rather than $\C$.
Over $\C$, scalar extraction would fail the positivity condition: for the
imaginary unit $i \in \C$, we would have $\varepsilon((i \cdot 1)^*(i \cdot 1))
= \varepsilon(-1) = -1 < 0$, violating $M$-positivity. This is one reason the
formalization uses $\R$-linear functionals.
\end{remark}
