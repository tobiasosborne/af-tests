% Auto-generated from Lean 4 source
% Source files: Forward.lean, SpanIntersection.lean, SeparatingFunctional.lean,
%               RieszApplication.lean, ComplexExtension.lean, Normalization.lean

\chapter{Dual Characterization}
\label{ch:dual}

This chapter establishes the dual characterization of the Archimedean closure
$\bar{M}$. The main result shows that membership in the closure can be detected
by evaluation against $M$-positive states: an element $A$ belongs to $\bar{M}$
if and only if $\varphi(A) \ge 0$ for all $M$-positive states $\varphi$.

%% ============================================================
\section{Forward Direction}
\label{sec:forward}

This section proves the forward direction of the dual characterization:
if $A$ is in the closure of the quadratic module $M$ (in the $\|\cdot\|_M$
topology), then $\varphi(A) \ge 0$ for all $M$-positive states $\varphi$.

The main result is:
\begin{itemize}
  \item \texttt{closure\_implies\_nonneg} (Theorem~\ref{thm:closure_implies_nonneg}):
        $A \in \bar{M}$ implies $\varphi(A) \ge 0$ for all $\varphi \in S_M$
\end{itemize}

\begin{lemma}[Linearity of States]\label{lem:apply_sub}
For any $M$-positive state $\varphi$ and elements $a, m \in \mathcal{A}_0$,
\[
  \varphi(a - m) = \varphi(a) - \varphi(m).
\]
\end{lemma}

\begin{proof}
Immediate from the linearity of $\varphi$.
\end{proof}

\begin{theorem}[Closure Implies Nonnegativity]\label{thm:closure_implies_nonneg}
If $A \in \bar{M}$, then $\varphi(A) \ge 0$ for all $M$-positive states $\varphi$.
\end{theorem}

\begin{proof}[Proof sketch]
Suppose, for contradiction, that $\varphi(A) < 0$. Set $\varepsilon = -\varphi(A) > 0$.
By definition of closure, there exists $m \in M$ with $\|A - m\|_M < \varepsilon$.

Since $|\varphi(A - m)| \le \|A - m\|_M$ (by the seminorm bound on states) and
$\varphi(m) \ge 0$ (by $M$-positivity), we have:
\begin{align*}
  |\varphi(A - m)| &< -\varphi(A) = \varepsilon \\
  \Rightarrow\quad \varphi(A) &< \varphi(A) - \varphi(m) + \varepsilon
\end{align*}
This yields $\varphi(m) < \varepsilon = -\varphi(A)$. Combined with
$\varphi(A) - \varphi(m) < -\varphi(A)$ from the absolute value bound,
we obtain a contradiction via linear arithmetic.
\end{proof}

%% ============================================================
\section{Span Intersection Lemma}
\label{sec:span-intersection}

This section proves that if $A \notin \bar{M}$, then positive scalar multiples
of $A$ cannot belong to $M$. This is the key result needed for constructing
the separating functional in the Riesz extension argument.

The main results are:
\begin{itemize}
  \item \texttt{positive\_smul\_not\_in\_M} (Lemma~\ref{lem:positive_smul_not_in_M}):
        If $A \notin \bar{M}$ and $c > 0$, then $c \cdot A \notin M$
  \item \texttt{separating\_nonneg\_on\_span\_cap\_M}
        (Theorem~\ref{thm:separating_nonneg_on_span_cap_M}):
        A separating functional $\psi_0(\lambda A) = -\lambda\varepsilon$
        is automatically nonnegative on $M \cap \mathrm{span}\{A\}$
\end{itemize}

\begin{lemma}[Positive Scalar Multiples]\label{lem:positive_smul_not_in_M}
If $A \notin \bar{M}$ and $c > 0$, then $c \cdot A \notin M$.
\end{lemma}

\begin{proof}[Proof sketch]
Suppose $c \cdot A \in M$. Since $M$ is a cone and $c^{-1} > 0$, we have
\[
  A = c^{-1} \cdot (c \cdot A) \in M.
\]
But $M \subseteq \bar{M}$, so $A \in \bar{M}$, contradicting $A \notin \bar{M}$.
\end{proof}

\begin{corollary}\label{cor:self_not_in_M}
If $A \notin \bar{M}$, then $A \notin M$.
\end{corollary}

\begin{proof}
Apply Lemma~\ref{lem:positive_smul_not_in_M} with $c = 1$.
\end{proof}

\begin{theorem}[Separating Functional Nonnegativity]\label{thm:separating_nonneg_on_span_cap_M}
For $A \notin \bar{M}$ and $\varepsilon > 0$, the functional
$\psi_0(\lambda A) = -\lambda \varepsilon$ is nonnegative on $M \cap \mathrm{span}\{A\}$.
\end{theorem}

\begin{proof}[Proof sketch]
Elements of $M \cap \mathrm{span}\{A\}$ have the form $\lambda A$ for some
$\lambda \in \mathbb{R}$ with $\lambda A \in M$. We consider two cases:
\begin{itemize}
  \item If $\lambda > 0$: By Lemma~\ref{lem:positive_smul_not_in_M},
        $\lambda A \notin M$, so this case is impossible.
  \item If $\lambda \le 0$: Then $\psi_0(\lambda A) = -\lambda \varepsilon \ge 0$.
\end{itemize}
Thus $\psi_0$ is nonnegative on the intersection.
\end{proof}

\begin{lemma}[Coefficient Sign]\label{lem:span_cap_M_nonpos_coeff}
If $A \notin \bar{M}$ and $c \cdot A \in M$ for some $c \in \mathbb{R}$,
then $c \le 0$.
\end{lemma}

\begin{proof}
If $c > 0$, then by Lemma~\ref{lem:positive_smul_not_in_M}, $c \cdot A \notin M$,
a contradiction.
\end{proof}

%% ============================================================
\section{Constructing the Separating Functional}
\label{sec:separating-functional}

This section constructs a linear functional $\psi_0$ on $\mathrm{span}\{A\}$
that separates $A$ from $M$. When $A \notin \bar{M}$, we define
$\psi_0(\lambda A) = -\lambda$, achieving $\psi_0(A) = -1 < 0$ while
$\psi_0 \ge 0$ on $M \cap \mathrm{span}\{A\}$.

The main results are:
\begin{itemize}
  \item \texttt{not\_in\_closure\_ne\_zero} (Lemma~\ref{lem:not_in_closure_ne_zero}):
        $A \notin \bar{M}$ implies $A \ne 0$
  \item \texttt{separatingOnSpan} (Definition~\ref{def:separatingOnSpan}):
        The linear map $\psi_0 : \mathrm{span}\{A\} \to \mathbb{R}$ with
        $\psi_0(A) = -1$
  \item \texttt{separatingOnSpan\_nonneg\_on\_M\_cap\_span}
        (Theorem~\ref{thm:separatingOnSpan_nonneg}):
        $\psi_0 \ge 0$ on $M \cap \mathrm{span}\{A\}$
\end{itemize}

\begin{lemma}[Non-Zero Outside Closure]\label{lem:not_in_closure_ne_zero}
If $A \notin \bar{M}$, then $A \ne 0$.
\end{lemma}

\begin{proof}
We have $0 = 0^* \cdot 0 \in M \subseteq \bar{M}$. Thus if $A \notin \bar{M}$,
then $A \ne 0$.
\end{proof}

\begin{definition}[Separating Functional on Span]\label{def:separatingOnSpan}
For $A \notin \bar{M}$, the \textbf{separating functional}
$\psi_0 : \mathrm{span}\{A\} \to \mathbb{R}$ is defined by
\[
  \psi_0(\lambda A) = -\lambda.
\]
This is well-defined since $A \ne 0$ by Lemma~\ref{lem:not_in_closure_ne_zero}.
\end{definition}

\begin{remark}
The construction uses \texttt{LinearPMap.mkSpanSingleton}, which defines
a linear map on the span of a single non-zero vector by specifying its
value on that vector.
\end{remark}

\begin{lemma}[Value at $A$]\label{lem:separatingOnSpan_apply_A}
For $A \notin \bar{M}$, $\psi_0(A) = -1$.
\end{lemma}

\begin{proof}
By definition of $\psi_0$, taking $\lambda = 1$.
\end{proof}

\begin{corollary}\label{cor:separatingOnSpan_apply_A_neg}
For $A \notin \bar{M}$, $\psi_0(A) < 0$.
\end{corollary}

\begin{proof}
Immediate from $\psi_0(A) = -1 < 0$.
\end{proof}

\begin{lemma}[Value at Scalar Multiples]\label{lem:separatingOnSpan_apply_smul}
For $A \notin \bar{M}$ and $c \in \mathbb{R}$, $\psi_0(c \cdot A) = -c$.
\end{lemma}

\begin{proof}
By linearity of $\psi_0$ and the fact that $\psi_0(A) = -1$:
\[
  \psi_0(c \cdot A) = c \cdot \psi_0(A) = c \cdot (-1) = -c.
\]
\end{proof}

\begin{theorem}[Nonnegativity on Intersection]\label{thm:separatingOnSpan_nonneg}
For $A \notin \bar{M}$, the functional $\psi_0$ is nonnegative on
$M \cap \mathrm{span}\{A\}$.
\end{theorem}

\begin{proof}[Proof sketch]
Let $x \in M \cap \mathrm{span}\{A\}$. Then $x = c \cdot A$ for some
$c \in \mathbb{R}$, and $\psi_0(x) = -c$.

By Lemma~\ref{lem:span_cap_M_nonpos_coeff}, since $x = c \cdot A \in M$
and $A \notin \bar{M}$, we have $c \le 0$.

Therefore $\psi_0(x) = -c \ge 0$.
\end{proof}

%% ============================================================
\section{Separating Functional via Geometric Hahn--Banach}
\label{sec:riesz-application}

This section constructs a separating functional for the dual characterization theorem
using geometric separation in locally convex spaces.

The key insight is to use \texttt{ProperCone.hyperplane\_separation\_point} from mathlib,
which provides the geometric Hahn--Banach separation theorem for locally convex spaces.
This approach avoids the ``generating condition'' difficulties of the classical Riesz
extension theorem.

\subsection{Setup}

The construction proceeds in four steps:
\begin{enumerate}
  \item \textbf{Topology}: $\mathcal{A}_0$ is equipped with the topology from the state
        seminorm $\|\cdot\|_M$
  \item \textbf{Locally Convex Space}: Seminorm topologies are always locally convex
  \item \textbf{Proper Cone}: The closure $\bar{M}$ forms a proper cone:
        \begin{itemize}
          \item Nonempty ($0 \in \bar{M}$)
          \item Closed (by definition of closure)
          \item Cone (closed under addition and $\mathbb{R}_{\ge 0}$ scaling)
        \end{itemize}
  \item \textbf{Separation}: For $A \notin \bar{M}$, there exists a continuous $f$
        with $f \ge 0$ on $\bar{M}$ and $f(A) < 0$
\end{enumerate}

\begin{definition}[Quadratic Module Closure as Proper Cone]\label{def:quadraticModuleClosureProperCone}
The closure $\bar{M}$ forms a \textbf{proper cone} in $\mathcal{A}_0$, i.e., a nonempty
closed convex cone. The cone structure is inherited from the quadratic module:
\begin{itemize}
  \item $0 \in \bar{M}$ (from closure of $M$ containing $0$)
  \item $a, b \in \bar{M} \Rightarrow a + b \in \bar{M}$ (closure preserves addition)
  \item $c \ge 0, a \in \bar{M} \Rightarrow c \cdot a \in \bar{M}$ (closure preserves
        nonnegative scaling)
\end{itemize}
\end{definition}

\begin{lemma}\label{lem:mem_quadraticModuleClosureProperCone}
Membership in the proper cone is equivalent to membership in $\bar{M}$:
\[
  a \in \mathsf{quadraticModuleClosureProperCone} \iff a \in \bar{M}.
\]
\end{lemma}

\begin{theorem}[Existence of Separating Functional]\label{thm:riesz_extension_exists}
Let $A \in \mathcal{A}_0$ be self-adjoint with $A \notin \bar{M}$. Then there exists
a linear functional $\psi : \mathcal{A}_0 \to \mathbb{R}$ such that:
\begin{enumerate}
  \item $\psi(m) \ge 0$ for all $m \in M$
  \item $\psi(A) < 0$
\end{enumerate}
\end{theorem}

\begin{proof}[Proof sketch]
Apply \texttt{ProperCone.hyperplane\_separation\_point} to the proper cone $\bar{M}$
and the point $A \notin \bar{M}$. This yields a continuous linear functional $f$
with $f \ge 0$ on $\bar{M}$ and $f(A) < 0$. Since $M \subseteq \bar{M}$, we have
$f \ge 0$ on $M$. The underlying linear map gives the desired $\psi$.
\end{proof}

%% ============================================================
\section{Symmetrization of Real Functional}
\label{sec:complex-extension}

The separation theorem yields $\psi : \mathcal{A}_0 \to \mathbb{R}$ with $\psi \ge 0$
on $M$ and $\psi(A) < 0$. However, an $M$-positive state requires symmetry:
$\varphi(a^*) = \varphi(a)$. This section constructs the symmetrization.

\begin{definition}[Star as Linear Map]\label{def:starAsLinearMap}
The star operation $a \mapsto a^*$ defines an $\mathbb{R}$-linear map
$\mathsf{star} : \mathcal{A}_0 \to \mathcal{A}_0$. This uses that for $r \in \mathbb{R}$,
we have $(\mathsf{algebraMap}(r))^* = \mathsf{algebraMap}(r)$.
\end{definition}

\begin{definition}[Symmetrization]\label{def:symmetrize}
Given $\psi : \mathcal{A}_0 \to \mathbb{R}$ linear, the \textbf{symmetrization} is
\[
  \varphi(a) = \frac{\psi(a) + \psi(a^*)}{2}.
\]
\end{definition}

\begin{lemma}[Symmetrization is Symmetric]\label{lem:symmetrize_map_star}
For any linear $\psi$ and all $a \in \mathcal{A}_0$:
\[
  \varphi(a^*) = \varphi(a).
\]
\end{lemma}

\begin{proof}
Direct computation:
$\varphi(a^*) = \frac{\psi(a^*) + \psi((a^*)^*)}{2} = \frac{\psi(a^*) + \psi(a)}{2} = \varphi(a)$.
\end{proof}

\begin{lemma}[Symmetrization on Self-Adjoint Elements]\label{lem:symmetrize_eq_of_selfAdjoint}
If $a = a^*$, then $\varphi(a) = \psi(a)$.
\end{lemma}

\begin{proof}
When $a^* = a$: $\varphi(a) = \frac{\psi(a) + \psi(a)}{2} = \psi(a)$.
\end{proof}

\begin{lemma}[Elements of $M$ are Self-Adjoint]\label{lem:isSelfAdjoint_of_mem_quadraticModule}
Every element $m \in M$ satisfies $m^* = m$.
\end{lemma}

\begin{proof}[Proof sketch]
By induction on the structure of $M$:
\begin{itemize}
  \item Squares $a^* a$ are self-adjoint: $(a^* a)^* = a^* (a^*)^* = a^* a$
  \item Generator-weighted squares $b^* g_j b$ are self-adjoint (since $g_j^* = g_j$)
  \item Self-adjointness is preserved by addition and nonnegative scaling
\end{itemize}
\end{proof}

\begin{lemma}[Symmetrization Preserves $M$-Nonnegativity]\label{lem:symmetrize_nonneg_on_M}
If $\psi(m) \ge 0$ for all $m \in M$, then $\varphi(m) \ge 0$ for all $m \in M$.
\end{lemma}

\begin{proof}
For $m \in M$, we have $m^* = m$ by Lemma~\ref{lem:isSelfAdjoint_of_mem_quadraticModule}.
Thus $\varphi(m) = \psi(m) \ge 0$ by Lemma~\ref{lem:symmetrize_eq_of_selfAdjoint}.
\end{proof}

\begin{lemma}[Symmetrization Preserves Negativity]\label{lem:symmetrize_neg_of_selfAdjoint}
If $A^* = A$ and $\psi(A) < 0$, then $\varphi(A) < 0$.
\end{lemma}

\begin{proof}
By Lemma~\ref{lem:symmetrize_eq_of_selfAdjoint}, $\varphi(A) = \psi(A) < 0$.
\end{proof}

\begin{theorem}[Symmetric Separating Functional]\label{thm:symmetrize_separation}
Let $A \in \mathcal{A}_0$ be self-adjoint with $A \notin \bar{M}$. Then there exists
a linear functional $\varphi : \mathcal{A}_0 \to \mathbb{R}$ such that:
\begin{enumerate}
  \item $\varphi(a^*) = \varphi(a)$ for all $a$
  \item $\varphi(m) \ge 0$ for all $m \in M$
  \item $\varphi(A) < 0$
\end{enumerate}
\end{theorem}

\begin{proof}[Proof sketch]
Apply Theorem~\ref{thm:riesz_extension_exists} to obtain $\psi$, then symmetrize.
\end{proof}

%% ============================================================
\section{Normalization to $M$-Positive State}
\label{sec:normalization}

This section normalizes the symmetric separating functional to obtain a true
$M$-positive state with $\varphi(1) = 1$.

\subsection{Cauchy--Schwarz for Symmetric Functionals}

\begin{theorem}[Cauchy--Schwarz for Symmetric Functionals]\label{thm:cauchy_schwarz_general}
Let $\varphi : \mathcal{A}_0 \to \mathbb{R}$ be linear with $\varphi(a^*) = \varphi(a)$
and $\varphi(m) \ge 0$ for $m \in M$. Then for all $a \in \mathcal{A}_0$:
\[
  \varphi(a)^2 \le \varphi(a^* a) \cdot \varphi(1).
\]
\end{theorem}

\begin{proof}[Proof sketch]
Consider the quadratic $q(t) = \varphi((a + t \cdot 1)^*(a + t \cdot 1)) \ge 0$ for
$t \in \mathbb{R}$. Expanding yields
\[
  q(t) = \varphi(a^* a) + 2t \cdot \varphi(a) + t^2 \cdot \varphi(1).
\]
If $\varphi(1) = 0$, the quadratic reduces to a linear function, and nonnegativity
for all $t$ forces $\varphi(a) = 0$. If $\varphi(1) \ne 0$, the discriminant condition
$4\varphi(a)^2 - 4\varphi(a^* a)\varphi(1) \le 0$ gives the result.
\end{proof}

\subsection{Positivity of $\varphi(1)$}

\begin{theorem}[$\varphi(1) > 0$]\label{thm:phi_one_pos}
Let $\varphi : \mathcal{A}_0 \to \mathbb{R}$ be symmetric with $\varphi \ge 0$ on $M$.
If $A^* = A$ and $\varphi(A) < 0$, then $\varphi(1) > 0$.
\end{theorem}

\begin{proof}[Proof sketch]
We rule out the cases $\varphi(1) = 0$ and $\varphi(1) < 0$:

\textbf{Case $\varphi(1) = 0$:} By Cauchy--Schwarz, $\varphi(a)^2 \le \varphi(a^* a) \cdot 0 = 0$
for all $a$. Thus $\varphi \equiv 0$, contradicting $\varphi(A) < 0$.

\textbf{Case $\varphi(1) < 0$:} By the Archimedean property, there exists $N \in \mathbb{N}$
with $N \cdot 1 - A^* A \in M$. Then $N \cdot \varphi(1) \ge \varphi(A^* A) \ge 0$.
Since $\varphi(1) < 0$ and $N \ge 0$, this forces $\varphi(A^* A) = 0$.
By Cauchy--Schwarz, $\varphi(A)^2 \le 0$, so $\varphi(A) = 0$. Contradiction.
\end{proof}

\subsection{Normalized State}

\begin{definition}[Normalized $M$-Positive State]\label{def:normalizedMPositiveState}
Given $\varphi$ symmetric with $\varphi \ge 0$ on $M$ and $\varphi(1) > 0$, define
\[
  \tilde{\varphi}(a) = \frac{\varphi(a)}{\varphi(1)}.
\]
Then $\tilde{\varphi}$ is an $M$-positive state: $\tilde{\varphi}(a^*) = \tilde{\varphi}(a)$,
$\tilde{\varphi}(1) = 1$, and $\tilde{\varphi}(m) \ge 0$ for $m \in M$.
\end{definition}

\begin{lemma}[Normalization Preserves Sign]\label{lem:normalizedMPositiveState_negative}
If $\varphi(A) < 0$ and $\varphi(1) > 0$, then $\tilde{\varphi}(A) < 0$.
\end{lemma}

\begin{proof}
$\tilde{\varphi}(A) = \varphi(A) / \varphi(1) < 0$ since the numerator is negative
and the denominator is positive.
\end{proof}

\subsection{Main Theorem}

\begin{theorem}[Existence of Negative $M$-Positive State]\label{thm:exists_MPositiveState_negative}
Let $A \in \mathcal{A}_0$ be self-adjoint with $A \notin \bar{M}$. Then there exists
an $M$-positive state $\varphi \in S_M$ with $\varphi(A) < 0$.
\end{theorem}

\begin{proof}[Proof sketch]
Combine the previous results:
\begin{enumerate}
  \item By Theorem~\ref{thm:symmetrize_separation}, obtain symmetric $\psi$ with
        $\psi \ge 0$ on $M$ and $\psi(A) < 0$
  \item By Theorem~\ref{thm:phi_one_pos}, $\psi(1) > 0$
  \item By Definition~\ref{def:normalizedMPositiveState}, normalize to get
        $\varphi \in S_M$ with $\varphi(A) < 0$
\end{enumerate}
\end{proof}

\begin{remark}
This theorem is the key ingredient for the backward direction of the main dual
characterization: if $A \notin \bar{M}$, there exists a state witnessing this via
$\varphi(A) < 0$. Combined with the forward direction (states are nonnegative on
$\bar{M}$), this yields the equivalence
\[
  A \in \bar{M} \iff \varphi(A) \ge 0 \text{ for all } \varphi \in S_M.
\]
\end{remark}

