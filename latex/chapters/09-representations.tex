% Auto-generated from Lean 4 source
% Source files: Constrained.lean, VectorState.lean, GNSConstrained.lean
% Do not edit manually

\chapter{Constrained Representations}
\label{ch:representations}

This chapter develops the theory of constrained $*$-representations, which are
the representation-theoretic counterpart to $M$-positive states. The generators
$g_j$ of the free $*$-algebra represent physical observables (such as position),
which should be positive operators. A constrained representation respects this
requirement: each $\pi(g_j)$ is a positive operator on the Hilbert space.

The main results establish the fundamental equivalence between states and
representations:
\begin{itemize}
  \item \texttt{ConstrainedStarRep} (Definition~\ref{def:constrained-rep}): constrained $*$-representations
  \item \texttt{vectorState} (Theorem~\ref{thm:vector-state-mpositive}): vector states from constrained reps are $M$-positive
  \item \texttt{state\_nonneg\_implies\_rep\_positive} (Theorem~\ref{thm:states-to-reps}): forward direction
  \item \texttt{gns\_constrained\_implies\_state\_nonneg} (Theorem~\ref{thm:reps-to-states}): backward direction
\end{itemize}

%% ============================================================
\section{Constrained $*$-Representations}
\label{sec:constrained-rep}

A constrained $*$-representation is a $*$-algebra homomorphism to bounded
operators on a Hilbert space, where the generators map to positive operators.

\begin{definition}[Constrained $*$-Representation]\label{def:constrained-rep}
A \textbf{constrained $*$-representation} of the free $*$-algebra $\mathcal{A}_0$
with $n$ generators consists of:
\begin{enumerate}
  \item A complex Hilbert space $H$;
  \item A $*$-algebra homomorphism $\pi : \mathcal{A}_0 \to \mathcal{B}(H)$ to bounded operators on $H$;
  \item The \emph{constraint}: for each generator $g_j$ ($j = 1, \ldots, n$), the operator $\pi(g_j)$ is positive, i.e., $\pi(g_j) \ge 0$.
\end{enumerate}
\end{definition}

\begin{remark}
The constraint $\pi(g_j) \ge 0$ means that $\langle \xi, \pi(g_j)\xi \rangle \ge 0$
for all $\xi \in H$. This reflects the physical interpretation of generators as
positive observables.
\end{remark}

\begin{lemma}[Representation Properties]\label{lem:rep-properties}
Let $\pi : \mathcal{A}_0 \to \mathcal{B}(H)$ be a constrained $*$-representation.
Then for all $a, b \in \mathcal{A}_0$:
\begin{enumerate}
  \item $\pi(1) = \mathrm{id}_H$ (preserves identity);
  \item $\pi(ab) = \pi(a)\pi(b)$ (preserves multiplication);
  \item $\pi(a^*) = \pi(a)^*$ (preserves adjoint).
\end{enumerate}
\end{lemma}

\begin{proof}
These follow directly from the definition of a $*$-algebra homomorphism.
\end{proof}

%% ============================================================
\section{Vector States}
\label{sec:vector-states}

Given a constrained representation $\pi$ and a unit vector $\xi \in H$, we
construct the associated vector state and prove it is $M$-positive.

\begin{definition}[Vector State Functional]\label{def:vector-state-fun}
Let $\pi : \mathcal{A}_0 \to \mathcal{B}(H)$ be a constrained $*$-representation
and $\xi \in H$ a unit vector ($\|\xi\| = 1$). The \textbf{vector state functional}
is
\[
  \varphi_\xi : \mathcal{A}_0 \to \mathbb{R}, \quad \varphi_\xi(a) = \mathrm{Re}\langle \xi, \pi(a)\xi \rangle.
\]
\end{definition}

\begin{lemma}[Linearity]\label{lem:vector-state-linear}
The vector state functional $\varphi_\xi$ is $\mathbb{R}$-linear:
\begin{enumerate}
  \item $\varphi_\xi(a + b) = \varphi_\xi(a) + \varphi_\xi(b)$;
  \item $\varphi_\xi(c \cdot a) = c \cdot \varphi_\xi(a)$ for $c \in \mathbb{R}$.
\end{enumerate}
\end{lemma}

\begin{proof}
Both properties follow from the linearity of the inner product in the second
argument and the linearity of taking the real part.
\end{proof}

\begin{lemma}[Symmetry]\label{lem:vector-state-star}
For any $a \in \mathcal{A}_0$,
\[
  \varphi_\xi(a^*) = \varphi_\xi(a).
\]
\end{lemma}

\begin{proof}
Using $\pi(a^*) = \pi(a)^*$:
\begin{align*}
  \varphi_\xi(a^*) &= \mathrm{Re}\langle \xi, \pi(a^*)\xi \rangle \\
  &= \mathrm{Re}\langle \xi, \pi(a)^*\xi \rangle \\
  &= \mathrm{Re}\langle \pi(a)\xi, \xi \rangle \\
  &= \mathrm{Re}\overline{\langle \xi, \pi(a)\xi \rangle} \\
  &= \mathrm{Re}\langle \xi, \pi(a)\xi \rangle = \varphi_\xi(a).
\end{align*}
\end{proof}

\begin{lemma}[Normalization]\label{lem:vector-state-one}
If $\|\xi\| = 1$, then $\varphi_\xi(1) = 1$.
\end{lemma}

\begin{proof}
$\varphi_\xi(1) = \mathrm{Re}\langle \xi, \pi(1)\xi \rangle = \mathrm{Re}\langle \xi, \xi \rangle = \|\xi\|^2 = 1$.
\end{proof}

\begin{lemma}[Positivity on Sums of Squares]\label{lem:vector-state-sos}
For any $a \in \mathcal{A}_0$,
\[
  \varphi_\xi(a^* a) \ge 0.
\]
\end{lemma}

\begin{proof}
Using $\pi(a^*) = \pi(a)^*$:
\begin{align*}
  \varphi_\xi(a^* a) &= \mathrm{Re}\langle \xi, \pi(a^* a)\xi \rangle \\
  &= \mathrm{Re}\langle \xi, \pi(a)^*\pi(a)\xi \rangle \\
  &= \mathrm{Re}\langle \pi(a)\xi, \pi(a)\xi \rangle \\
  &= \|\pi(a)\xi\|^2 \ge 0.
\end{align*}
\end{proof}

\begin{lemma}[Positivity on Generator Terms]\label{lem:vector-state-gen}
For any generator $g_j$ and $b \in \mathcal{A}_0$,
\[
  \varphi_\xi(b^* g_j b) \ge 0.
\]
\end{lemma}

\begin{proof}
Let $v = \pi(b)\xi$. Since $\pi(g_j) \ge 0$ is a positive operator:
\begin{align*}
  \varphi_\xi(b^* g_j b) &= \mathrm{Re}\langle \xi, \pi(b^* g_j b)\xi \rangle \\
  &= \mathrm{Re}\langle \xi, \pi(b)^*\pi(g_j)\pi(b)\xi \rangle \\
  &= \mathrm{Re}\langle v, \pi(g_j)v \rangle \ge 0.
\end{align*}
The last inequality uses the positivity of $\pi(g_j)$.
\end{proof}

\begin{theorem}[Vector States are $M$-Positive]\label{thm:vector-state-mpositive}
Let $\pi$ be a constrained $*$-representation and $\xi \in H$ a unit vector.
Then the vector state $\varphi_\xi$ is an $M$-positive state, i.e., $\varphi_\xi \in S_M$.
\end{theorem}

\begin{proof}
We verify the defining properties of an $M$-positive state:
\begin{enumerate}
  \item \emph{Linearity}: Lemma~\ref{lem:vector-state-linear}.
  \item \emph{Symmetry}: Lemma~\ref{lem:vector-state-star}.
  \item \emph{Normalization}: Lemma~\ref{lem:vector-state-one}.
  \item \emph{$M$-positivity}: For $m \in M$, we need $\varphi_\xi(m) \ge 0$.
    The quadratic module $M$ is generated by elements of the form $a^* a$ and
    $b^* g_j b$, and is closed under sums and positive scalar multiplication.
    By Lemmas~\ref{lem:vector-state-sos} and~\ref{lem:vector-state-gen},
    $\varphi_\xi$ is nonnegative on generators. By linearity, $\varphi_\xi(m) \ge 0$
    for all $m \in M$.
\end{enumerate}
\end{proof}

%% ============================================================
\section{GNS Representations are Constrained}
\label{sec:gns-constrained}

We now establish the key equivalence between positivity in states and positivity
in constrained representations. This requires the GNS construction developed
in Chapter~\ref{ch:gns}.

\begin{theorem}[GNS Representation Exists]\label{thm:gns-exists}
Let $\varphi$ be an $M$-positive state. Under the Archimedean assumption, there
exists a constrained $*$-representation $\pi_\varphi$ and a cyclic unit vector
$\Omega \in H_\varphi$ such that for all $a \in \mathcal{A}_0$:
\[
  \varphi(a) = \mathrm{Re}\langle \Omega, \pi_\varphi(a)\Omega \rangle.
\]
\end{theorem}

\begin{proof}[Proof sketch]
The construction follows the standard GNS procedure:
\begin{enumerate}
  \item Form the null space $\mathcal{N}_\varphi = \{a : \varphi(a^* a) = 0\}$.
  \item Quotient by $\mathcal{N}_\varphi$ and complete to obtain the Hilbert space $H_\varphi$.
  \item The left multiplication action extends to a $*$-representation $\pi_\varphi$.
  \item The image of $1 \in \mathcal{A}_0$ gives the cyclic vector $\Omega$.
\end{enumerate}
The representation is constrained because $\varphi(g_j) \ge 0$ (taking $b = 1$ in
$b^* g_j b \in M$) implies $\pi_\varphi(g_j)$ is a positive operator. The inner
product reconstruction follows from the cyclic vector identity.
\end{proof}

\begin{theorem}[Forward Direction: States to Representations]\label{thm:states-to-reps}
Let $A \in \mathcal{A}_0$ be self-adjoint. If $\varphi(A) \ge 0$ for all $M$-positive
states $\varphi \in S_M$, then $\pi(A) \ge 0$ for all constrained $*$-representations $\pi$.
\end{theorem}

\begin{proof}
Let $\pi : \mathcal{A}_0 \to \mathcal{B}(H)$ be a constrained representation.
To show $\pi(A) \ge 0$, we verify $\langle v, \pi(A)v \rangle \ge 0$ for all $v \in H$.

For $v = 0$, this is trivial. For $v \ne 0$, let $u = \|v\|^{-1} v$ be the
normalization. By Theorem~\ref{thm:vector-state-mpositive}, the vector state
$\varphi_u$ is $M$-positive. By hypothesis:
\[
  \varphi_u(A) = \mathrm{Re}\langle u, \pi(A)u \rangle \ge 0.
\]
Since $A$ is self-adjoint, $\pi(A)$ is self-adjoint, so $\langle u, \pi(A)u \rangle$
is real. Thus $\langle u, \pi(A)u \rangle \ge 0$.

Finally, $v = \|v\| \cdot u$, so:
\[
  \langle v, \pi(A)v \rangle = \|v\|^2 \langle u, \pi(A)u \rangle \ge 0.
\]
\end{proof}

\begin{theorem}[Backward Direction: Representations to States]\label{thm:reps-to-states}
Under the Archimedean assumption, let $A \in \mathcal{A}_0$ be self-adjoint.
If $\pi(A) \ge 0$ for all constrained $*$-representations $\pi$, then $\varphi(A) \ge 0$
for all $M$-positive states $\varphi \in S_M$.
\end{theorem}

\begin{proof}
Let $\varphi$ be an $M$-positive state. By Theorem~\ref{thm:gns-exists}, there
exists a constrained representation $\pi_\varphi$ and cyclic vector $\Omega$ with
$\|\Omega\| = 1$ such that:
\[
  \varphi(A) = \mathrm{Re}\langle \Omega, \pi_\varphi(A)\Omega \rangle.
\]
By hypothesis, $\pi_\varphi(A) \ge 0$. For a positive operator $T$ and any vector $\xi$:
\[
  \langle \xi, T\xi \rangle \ge 0.
\]
Therefore:
\[
  \varphi(A) = \mathrm{Re}\langle \Omega, \pi_\varphi(A)\Omega \rangle = \langle \Omega, \pi_\varphi(A)\Omega \rangle \ge 0.
\]
(The real part equals the full inner product since $\pi_\varphi(A)$ is self-adjoint.)
\end{proof}

\begin{corollary}[State-Representation Equivalence]\label{cor:state-rep-equiv}
Under the Archimedean assumption, for self-adjoint $A \in (\mathcal{A}_0)_{\mathrm{sa}}$:
\[
  \bigl(\forall \varphi \in S_M,\, \varphi(A) \ge 0\bigr)
  \iff
  \bigl(\forall \pi \text{ constrained},\, \pi(A) \ge 0\bigr).
\]
\end{corollary}

\begin{proof}
Combine Theorems~\ref{thm:states-to-reps} and~\ref{thm:reps-to-states}.
\end{proof}

\begin{remark}
Corollary~\ref{cor:state-rep-equiv} is the key bridge between the algebraic
characterization of positivity (via states) and the representation-theoretic
characterization (via constrained representations). This equivalence is
essential for the main theorem in Chapter~\ref{ch:main-theorem}.
\end{remark}
