\documentclass[11pt]{article}

\usepackage[margin=1in]{geometry}
\usepackage{amsmath,amsthm,amssymb}
\usepackage{hyperref}

% Theorem environments
\newtheorem{theorem}{Theorem}
\newtheorem{proposition}[theorem]{Proposition}
\newtheorem{lemma}[theorem]{Lemma}
\theoremstyle{definition}
\newtheorem{definition}[theorem]{Definition}
\newtheorem{assumption}[theorem]{Assumption}
\theoremstyle{remark}
\newtheorem{remark}[theorem]{Remark}

% Notation shortcuts
\newcommand{\Tr}{\operatorname{Tr}}
\newcommand{\supp}{\operatorname{supp}}
\newcommand{\id}{\mathbb{I}}
\newcommand{\R}{\mathbb{R}}
\newcommand{\C}{\mathbb{C}}
\newcommand{\pos}{_{+}}

\title{Path-Integral Hockey-Stick Representation\\of Conditional Entropy Differences}
\author{Adversarial Proof Framework (\texttt{af})\\[4pt]
\small Verified by independent prover and verifier agents\\
\small 36 nodes, all validated and clean}
\date{February 2026}

\begin{document}
\maketitle

\begin{abstract}
We present a complete proof of the path-integral hockey-stick representation
for the difference of conditional entropies of two bipartite quantum states.
Under a full-rank assumption, the conditional entropy difference
$H(A|B)_\sigma - H(A|B)_\rho$ is expressed as a double integral involving
the spectral projector of an interpolating operator and the hockey-stick
divergence kernel.  The proof was constructed and adversarially verified
within the \texttt{af} (Adversarial Proof Framework), in which independent
prover agents propose proof steps and independent verifier agents challenge
them until all nodes reach a validated state.
\end{abstract}

\tableofcontents

%% ========================================================================
\section{Notation and Definitions}\label{sec:notation}
%% ========================================================================

\paragraph{Hilbert spaces.}
Let $\mathcal{H}_A$ and $\mathcal{H}_B$ be finite-dimensional complex Hilbert spaces
with $d_A := \dim \mathcal{H}_A$ and $d_B := \dim \mathcal{H}_B$.
Write $\mathcal{H}_{AB} := \mathcal{H}_A \otimes \mathcal{H}_B$ with
$\dim \mathcal{H}_{AB} = d_A d_B$, and let $\id_X$ denote the identity
operator on $\mathcal{H}_X$.

\paragraph{States and operators.}
A \emph{density operator} is a positive semidefinite operator $\rho \ge 0$
with $\Tr[\rho] = 1$.  We write $X > 0$ for strictly positive definite
operators.  The \emph{partial trace} is $\rho_B := \Tr_A[\rho_{AB}]$.
The \emph{trace norm} is $\|X\|_1 := \Tr\!\sqrt{X^\dagger X}$.

\paragraph{Entropy and divergence.}
The \emph{von Neumann entropy} is $S(\rho) := -\Tr[\rho \log \rho]$
(natural logarithm, with $0 \log 0 := 0$).
The \emph{conditional entropy} is
\[
  H(A|B)_\rho \;:=\; S(\rho_{AB}) - S(\rho_B).
\]
The \emph{(Umegaki) relative entropy} is
\[
  D(\rho \| \sigma) \;:=\; \Tr[\rho(\log \rho - \log \sigma)]
\]
when $\supp(\rho) \subseteq \supp(\sigma)$, and $+\infty$ otherwise.
A key identity (\textbf{CE-RE}) is
\begin{equation}\label{eq:CE-RE}
  H(A|B)_\rho \;=\; -D(\rho_{AB} \| \id_A \otimes \rho_B).
\end{equation}

\paragraph{Hockey-stick divergence.}
For positive semidefinite operators $\rho, \sigma$ and $\gamma > 0$,
\[
  E_\gamma(\rho \| \sigma) \;:=\; \Tr(\rho - \gamma \sigma)\pos
  \;=\; \max_{0 \le M \le \id}\; \Tr[M(\rho - \gamma\sigma)].
\]
The maximiser is $P_\gamma = \mathbf{1}\{\rho - \gamma\sigma > 0\}$,
the spectral projector onto the strictly positive eigenspace.

\paragraph{Max-relative entropy.}
$M(\rho,\sigma) := \inf\{\lambda \ge 0 : \rho \le \lambda\sigma\}$.
For bipartite states, $M(\rho_{AB}, \id_A \otimes \rho_B) \le d_A$.

\paragraph{Frenkel integral representation.}
For a density operator $\rho$ and an unnormalised reference $\tau$ with
$\Tr[\tau] = c > 0$ and $\supp(\rho) \subseteq \supp(\tau)$ (\textbf{FR-bip}):
\begin{equation}\label{eq:frenkel}
  D(\rho \| \tau) \;=\;
  \int_0^\infty
    \frac{E_\gamma(\rho\|\tau) - (1 - c\gamma)\pos}
         {\gamma(1 + c\gamma)}
  \,d\gamma \;-\; \log c.
\end{equation}

\paragraph{Interpolating path.}
Given density operators $\rho_{AB}, \sigma_{AB}$ on $\mathcal{H}_{AB}$, define
\begin{align*}
  \rho(t) &:= (1-t)\,\rho_{AB} + t\,\sigma_{AB}, \qquad t \in [0,1], \\
  \delta_{AB} &:= \sigma_{AB} - \rho_{AB}, \qquad
  \delta_B := \sigma_B - \rho_B = \Tr_A[\delta_{AB}], \\
  \rho(t)_B &:= \Tr_A[\rho(t)] = (1-t)\,\rho_B + t\,\sigma_B, \\
  \tau(t) &:= \id_A \otimes \rho(t)_B, \qquad
  \Tr[\tau(t)] = d_A.
\end{align*}
We also define the shifted difference operator
$\delta_{AB}^{(\gamma)} := \delta_{AB} - \gamma\,\id_A \otimes \delta_B$,
the spectral projector
$P_\gamma(t) := \mathbf{1}\{\rho(t) - \gamma\,\tau(t) > 0\}$,
and the $t$-dependent upper cutoff
$M(t) := M(\rho(t), \tau(t)) \le d_A$.

%% ========================================================================
\section{Statement of the Main Theorem}\label{sec:theorem}
%% ========================================================================

\begin{theorem}[Path-Integral Hockey-Stick Representation]\label{thm:main}
Let $\rho_{AB}, \sigma_{AB} > 0$ be strictly positive density operators on
$\mathcal{H}_A \otimes \mathcal{H}_B$.  Then
\begin{equation}\label{eq:main}
  H(A|B)_\sigma - H(A|B)_\rho
  \;=\;
  -\int_0^1 \!\int_0^{M(t)}
    \frac{\Tr\!\big[P_\gamma(t)\,\delta_{AB}^{(\gamma)}\big]}
         {\gamma\,(1 + d_A\gamma)}
  \,d\gamma\,dt.
\end{equation}
\end{theorem}

The proof assembles five propositions, which we now develop in turn.

%% ========================================================================
\section{Proof}\label{sec:proof}
%% ========================================================================

\begin{assumption}[Full Rank]\label{ass:fullrank}
Assume $\rho_{AB}, \sigma_{AB} > 0$.  Since the set of positive-definite
operators is convex, $\rho(t) = (1-t)\rho_{AB} + t\,\sigma_{AB} > 0$ for all
$t \in [0,1]$, and consequently $\rho(t)_B = \Tr_A[\rho(t)] > 0$ as well.
\end{assumption}

%% ------ Proposition 1 ------
\begin{proposition}[CE-RE Identity]\label{prop:CE-RE}
For all $t \in [0,1]$,
\[
  H(A|B)_{\rho(t)} \;=\; -D\!\big(\rho(t)_{AB} \,\big\|\, \id_A \otimes \rho(t)_B\big).
\]
\end{proposition}

\begin{proof}
By definition of relative entropy,
\[
  D(\rho_{AB} \| \id_A \otimes \rho_B)
  \;=\; \Tr[\rho_{AB} \log \rho_{AB}]
        \;-\; \Tr[\rho_{AB} \log(\id_A \otimes \rho_B)].
\]
By the spectral mapping theorem for tensor products,
$\log(\id_A \otimes \rho_B) = \id_A \otimes \log \rho_B$.
Using the partial-trace identity
$\Tr[\rho_{AB}(\id_A \otimes X_B)] = \Tr[\rho_B X_B]$,
\[
  \Tr[\rho_{AB}(\id_A \otimes \log \rho_B)]
  \;=\; \Tr[\rho_B \log \rho_B]
  \;=\; -S(\rho_B).
\]
Therefore
\[
  D(\rho_{AB} \| \id_A \otimes \rho_B)
  \;=\; -S(\rho_{AB}) + S(\rho_B)
  \;=\; -\big(S(\rho_{AB}) - S(\rho_B)\big)
  \;=\; -H(A|B)_\rho,
\]
giving $H(A|B)_\rho = -D(\rho_{AB} \| \id_A \otimes \rho_B)$.
The same argument applies at every point $\rho(t)$ along the interpolating path.
\end{proof}

%% ------ Proposition 2 ------
\begin{proposition}[Fundamental Theorem of Calculus]\label{prop:FTC}
\begin{equation}\label{eq:FTC}
  H(A|B)_\sigma - H(A|B)_\rho
  \;=\; \int_0^1 \frac{d}{dt}\, H(A|B)_{\rho(t)}\, dt.
\end{equation}
\end{proposition}

\begin{proof}
Under the full-rank assumption (Assumption~\ref{ass:fullrank}), $\rho(t)$
and $\rho(t)_B$ are strictly positive for all $t \in [0,1]$.  The eigenvalues
of $\rho(t)$, being eigenvalues of a matrix depending affinely on $t$, are smooth
and strictly positive; hence $S(\rho(t))$ and $S(\rho(t)_B)$ are $C^\infty$
functions of $t$.  Since $H(A|B)_{\rho(t)} = S(\rho(t)_{AB}) - S(\rho(t)_B)$,
the map $t \mapsto H(A|B)_{\rho(t)}$ is $C^1$ on $[0,1]$.  The fundamental
theorem of calculus gives
\[
  H(A|B)_{\rho(1)} - H(A|B)_{\rho(0)}
  \;=\; \int_0^1 \frac{d}{dt}\, H(A|B)_{\rho(t)}\, dt.
\]
Since $\rho(0) = \rho_{AB}$ and $\rho(1) = \sigma_{AB}$, the result follows.
\end{proof}

%% ------ Proposition 3 ------
\begin{proposition}[Derivative of Conditional Entropy]\label{prop:DER}
\begin{equation}\label{eq:DER}
  \frac{d}{dt}\, H(A|B)_{\rho(t)}
  \;=\; \Tr\!\big[\delta_{AB}\big(\id_A \otimes \log \rho(t)_B - \log \rho(t)\big)\big].
\end{equation}
\end{proposition}

\begin{proof}
By Proposition~\ref{prop:CE-RE},
$\frac{d}{dt} H(A|B)_{\rho(t)} = -\frac{d}{dt} D(\rho(t) \| \tau(t))$.
Write
\[
  D(\rho(t)\|\tau(t))
  \;=\; \Tr[\rho(t) \log \rho(t)] - \Tr[\rho(t) \log \tau(t)].
\]

\emph{First term.}
Using $\dot{\rho}(t) = \delta_{AB}$,
\[
  \frac{d}{dt}\,\Tr[\rho(t)\log\rho(t)]
  \;=\;
  \Tr[\delta_{AB}\log\rho(t)]
  + \Tr\!\big[\rho(t)\, D_{\!\log}(\rho(t))[\delta_{AB}]\big].
\]
Here $D_{\!\log}(B)[C] = \int_0^\infty (B+s\id)^{-1}\, C\, (B+s\id)^{-1}\, ds$
denotes the Fr\'echet derivative of the matrix logarithm.  The identity
$\Tr[B\, D_{\!\log}(B)[C]] = \Tr[C]$ for $B > 0$
(which follows from the integral representation and cyclicity of trace)
gives, with $B = \rho(t) > 0$ and $C = \delta_{AB}$,
\[
  \Tr\!\big[\rho(t)\, D_{\!\log}(\rho(t))[\delta_{AB}]\big]
  \;=\; \Tr[\delta_{AB}]
  \;=\; 0,
\]
since $\delta_{AB} = \sigma_{AB} - \rho_{AB}$ is traceless (both states have
unit trace).  Therefore
$\frac{d}{dt}\,\Tr[\rho(t)\log\rho(t)] = \Tr[\delta_{AB}\log\rho(t)]$.

\emph{Second term.}
Since $\tau(t) = \id_A \otimes \rho(t)_B$, we have
$\log\tau(t) = \id_A \otimes \log\rho(t)_B$
and $\dot\tau(t) = \id_A \otimes \delta_B$.  Therefore
\[
  \frac{d}{dt}\,\Tr[\rho(t)\log\tau(t)]
  \;=\;
  \Tr[\delta_{AB}\log\tau(t)]
  + \Tr\!\big[\rho(t)\, D_{\!\log}(\tau(t))[\dot\tau(t)]\big].
\]
The Fr\'echet derivative term reduces (via the partial-trace identity
and $\rho(t)_B > 0$) to
\[
  \Tr\!\big[\rho(t)_B\,\delta_B\,\rho(t)_B^{-1}\big] = \Tr[\delta_B] = 0.
\]

\emph{Combining.}
\[
  \frac{d}{dt}\, D(\rho(t)\|\tau(t))
  \;=\;
  \Tr[\delta_{AB}\log\rho(t)]
  - \Tr[\delta_{AB}(\id_A \otimes \log\rho(t)_B)]
  \;=\;
  \Tr\!\big[\delta_{AB}\big(\log\rho(t) - \id_A \otimes \log\rho(t)_B\big)\big].
\]
Negating yields \eqref{eq:DER}.
\end{proof}

%% ------ Proposition 4 ------
\begin{proposition}[Hockey-Stick Derivative]\label{prop:HS-DER}
At generic $\gamma$ (i.e., when no eigenvalue of $\rho(t) - \gamma\,\tau(t)$
equals zero),
\begin{equation}\label{eq:HS-DER}
  \frac{d}{dt}\, E_\gamma(\rho(t)\|\tau(t))
  \;=\;
  \Tr\!\big[P_\gamma(t)\,\delta_{AB}^{(\gamma)}\big].
\end{equation}
\end{proposition}

\begin{proof}
Let $A(t) := \rho(t) - \gamma\,\tau(t)$.  By spectral decomposition,
\[
  A(t) = \sum_i \lambda_i(t)\, |e_i(t)\rangle\langle e_i(t)|, \qquad
  A(t)\pos = \sum_{\lambda_i > 0} \lambda_i(t)\, |e_i(t)\rangle\langle e_i(t)|
  = P_\gamma(t)\, A(t)\, P_\gamma(t).
\]
At generic $\gamma$, no eigenvalue of $A(t)$ is zero, so the rank of
$A(t)\pos$ is locally constant in $t$, and the eigenvalues and
eigenprojections are smooth.  Hence $\Tr A(t)\pos$ is differentiable.

Since $A(t)\pos = P_\gamma(t)\, A(t)$ (using $[P_\gamma(t), A(t)] = 0$),
\[
  \frac{d}{dt}\, \Tr A(t)\pos
  = \Tr\!\Big[\frac{dP_\gamma}{dt}\, A(t)\Big]
  + \Tr\!\big[P_\gamma(t)\, \dot A(t)\big].
\]
We claim $\Tr[P' A] = 0$ where $P = P_\gamma(t)$ and $P' = dP_\gamma/dt$.
Since $P$ is a spectral projector of $A$, we have $[P, A] = 0$; both operators
are block-diagonal in the decomposition $\mathcal{H} = P\mathcal{H} \oplus
(\id - P)\mathcal{H}$.  Differentiating $P^2 = P$ gives
$P'P + PP' = P'$, from which $PP'P = 0$ and $(\id - P)P'(\id - P) = 0$.
Thus $P'$ is \emph{off-block-diagonal}:
\[
  P' = PP'(\id - P) + (\id - P)P'P.
\]
Since $A$ is block-diagonal and $P'$ is off-block-diagonal, their product
$P'A$ is off-block-diagonal.  Any off-block-diagonal operator has zero trace:
\begin{align*}
  \Tr[P'A]
  &= \Tr\!\big[PP'(\id-P)A\big] + \Tr\!\big[(\id-P)P'PA\big] \\
  &= \Tr\!\big[(\id-P)A\,PP'\big] + \Tr\!\big[PA\,(\id-P)P'\big] \\
  &= \Tr\!\big[(\id-P)P\cdot AP'\big] + \Tr\!\big[P(\id-P)\cdot AP'\big]
  = 0,
\end{align*}
using $P(\id - P) = 0$, $[A, P] = 0$, and cyclicity of trace.
Therefore
\[
  \frac{d}{dt}\, E_\gamma(\rho(t)\|\tau(t))
  = \Tr\!\big[P_\gamma(t)\,\dot A(t)\big]
  = \Tr\!\big[P_\gamma(t)\,(\delta_{AB} - \gamma\,\id_A \otimes \delta_B)\big]
  = \Tr\!\big[P_\gamma(t)\,\delta_{AB}^{(\gamma)}\big].
  \qedhere
\]
\end{proof}

%% ------ Proposition 5 ------
\begin{proposition}[Derivative via Hockey-Stick]\label{prop:DER-HS}
\begin{equation}\label{eq:DER-HS}
  \frac{d}{dt}\, D(\rho(t)\|\tau(t))
  \;=\;
  \int_0^{M(t)}
    \frac{\Tr\!\big[P_\gamma(t)\,\delta_{AB}^{(\gamma)}\big]}
         {\gamma\,(1 + d_A\gamma)}
  \,d\gamma.
\end{equation}
\end{proposition}

\begin{proof}
By the Frenkel bipartite formula~\eqref{eq:frenkel} with $c = d_A$,
\[
  D(\rho(t)\|\tau(t))
  = \int_0^\infty
      \frac{E_\gamma(\rho(t)\|\tau(t)) - (1-d_A\gamma)\pos}
           {\gamma(1+d_A\gamma)}
    \,d\gamma \;-\; \log d_A.
\]
The term $(1 - d_A\gamma)\pos$ and the constant $\log d_A$ are independent of $t$.

\emph{Differentiation under the integral sign.}
We justify the exchange of $\frac{d}{dt}$ and $\int$ via the dominated
convergence theorem.  Under the full-rank assumption, $\rho(t) > 0$ for all
$t \in [0,1]$.  By compactness of $[0,1]$ and continuity of the minimum
eigenvalue $\lambda_{\min}(\rho(t))$, there exists $\gamma_0 > 0$ (uniform in $t$)
such that for all $\gamma < \gamma_0$ and all $t \in [0,1]$,
$\rho(t) - \gamma\,\tau(t) > 0$.  For such $\gamma$,
$E_\gamma(\rho(t)\|\tau(t)) = \Tr[\rho(t) - \gamma\tau(t)] = 1 - \gamma d_A$,
which is independent of $t$, so $\frac{d}{dt} E_\gamma = 0$.
Thus the integrand vanishes on $(0, \gamma_0)$.
On $[\gamma_0, \infty)$, the bound
$\big|\frac{d}{dt} E_\gamma\big| \le \|\delta_{AB}\|_1 + \gamma\,\|\id_A \otimes \delta_B\|_1$
is finite, and $[\gamma(1+d_A\gamma)]^{-1} \le [\gamma_0(1+d_A\gamma_0)]^{-1}$
provides a uniform bound.  Dominated convergence applies.

\emph{Substituting the hockey-stick derivative.}
By Proposition~\ref{prop:HS-DER},
$\frac{d}{dt} E_\gamma(\rho(t)\|\tau(t)) = \Tr[P_\gamma(t)\,\delta_{AB}^{(\gamma)}]$
at generic $\gamma$.  The set of non-generic $\gamma$ has measure zero and does
not affect the integral.

\emph{Truncation of the upper limit.}
For $\gamma > M(t)$, we have $\rho(t) \le \gamma\,\tau(t)$, so
$E_\gamma(\rho(t)\|\tau(t)) = 0$ identically in $t$ and its $t$-derivative
vanishes.  The effective upper limit is $M(t) \le d_A$.

\emph{Combining} yields \eqref{eq:DER-HS}.
\end{proof}

%% ------ Main assembly ------
\begin{proof}[Proof of Theorem~\ref{thm:main}]
We assemble the five propositions.

\emph{Step 1} (FTC).
By Proposition~\ref{prop:FTC},
\[
  H(A|B)_\sigma - H(A|B)_\rho
  = \int_0^1 \frac{d}{dt}\, H(A|B)_{\rho(t)}\, dt.
\]

\emph{Step 2} (CE-RE).
By Proposition~\ref{prop:CE-RE},
$\frac{d}{dt} H(A|B)_{\rho(t)} = -\frac{d}{dt} D(\rho(t)\|\tau(t))$.

\emph{Step 3} (DER-HS).
By Proposition~\ref{prop:DER-HS},
\[
  \frac{d}{dt}\, D(\rho(t)\|\tau(t))
  = \int_0^{M(t)}
      \frac{\Tr\!\big[P_\gamma(t)\,\delta_{AB}^{(\gamma)}\big]}
           {\gamma(1+d_A\gamma)}
    \,d\gamma.
\]

\emph{Step 4} (Substitution).
Combining Steps 1--3:
\begin{equation}\label{eq:iterated}
  H(A|B)_\sigma - H(A|B)_\rho
  = -\int_0^1 \int_0^{M(t)}
      \frac{\Tr\!\big[P_\gamma(t)\,\delta_{AB}^{(\gamma)}\big]}
           {\gamma(1+d_A\gamma)}
    \,d\gamma\, dt.
\end{equation}

\emph{Step 5} (Fubini).
We verify that the iterated integral in \eqref{eq:iterated} is well-defined
via the Fubini--Tonelli theorem.  Under the full-rank assumption, $\rho(t) > 0$
for all $t \in [0,1]$.  By compactness and continuity of $\lambda_{\min}(\rho(t))$,
there exists $c > 0$ (uniform in $t$) such that for all $\gamma \in (0,c)$
and all $t \in [0,1]$, $\rho(t) - \gamma\,\tau(t) > 0$, hence $P_\gamma(t) = \id$.
Then
\[
  \Tr\!\big[P_\gamma(t)\,\delta_{AB}^{(\gamma)}\big]
  = \Tr[\delta_{AB}] - \gamma\,\Tr[\id_A \otimes \delta_B]
  = 0 - \gamma\, d_A\, \Tr[\delta_B]
  = 0,
\]
since $\Tr[\delta_{AB}] = 0$ (both states have unit trace) and
$\Tr[\delta_B] = 0$ (partial traces preserve trace).
The integrand therefore vanishes on $[0,1] \times (0,c)$.
On the remaining domain $[0,1] \times [c, d_A]$, the factor
$[\gamma(1+d_A\gamma)]^{-1}$ is bounded above by $[c(1+d_A c)]^{-1}$,
and the numerator satisfies
$|\Tr[P_\gamma(t)\,\delta_{AB}^{(\gamma)}]| \le \|\delta_{AB}\|_1 + d_A^2\,\|\delta_B\|_1$.
The integrand is thus bounded on this compact domain, and Fubini--Tonelli applies.

\medskip
The iterated integral in \eqref{eq:iterated} is therefore well-defined and
equals the claimed double integral.  This completes the proof.
\end{proof}

%% ========================================================================
\section{Concluding Remarks}
%% ========================================================================

\begin{remark}
The full-rank assumption ($\rho_{AB}, \sigma_{AB} > 0$) is used in three
essential places: (i) ensuring $C^1$ regularity of $t \mapsto H(A|B)_{\rho(t)}$
for the fundamental theorem of calculus; (ii) justifying differentiation under
the integral sign via the vanishing of the integrand near $\gamma = 0$; and
(iii) the Fubini argument eliminating the $1/\gamma$ singularity.
The representation may extend to the general case via a limiting argument,
but this requires additional care with the $\gamma \to 0$ singularity
when $\rho(t)$ has a non-trivial kernel.
\end{remark}

\begin{remark}
The proof relies on the Frenkel integral representation~\eqref{eq:frenkel},
which expresses relative entropy as an integral of hockey-stick divergences.
This representation, adapted here to the bipartite setting with unnormalised
reference $\tau(t) = \id_A \otimes \rho(t)_B$ (trace $d_A$), is the key
structural ingredient enabling the path-integral form.
\end{remark}

\begin{remark}
This proof was constructed within the Adversarial Proof Framework (\texttt{af}),
a system in which independent prover agents propose proof steps and independent
verifier agents challenge them.  The proof tree consists of 36 nodes, all of
which reached the \emph{validated} epistemic state with \emph{clean} taint.
Three challenges were raised against the main assembly node (concerning
dependency bookkeeping, structural soundness, and the Fubini justification),
all of which were resolved through amendments.
\end{remark}

\end{document}
